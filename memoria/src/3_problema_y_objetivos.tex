\chapter{Formulación del problema y objetivos}
\label{ch:problema_y_objetivos}

En este capítulo se describirá el contexto en de la investigación y los retos que plantea, así como los problemas que surgen de esta situación. Después, definiremos los objetivos específicos que orientan el estudio, estableciendo las metas a alcanzar y el alcance del proyecto. Al mismo tiempo, se abordarán con más detalle las situaciones que han provocado estos problemas y se expondrán distintos objetivos con la intención de mitigarlos.

\section{Contexto y motivación}
\label{sec:contexto}

La detección de \textit{malware} es uno de los grandes desafíos de la ciberseguridad por su rápida y constante evolución. Cada día aparecen nuevas amenazas capaces de evadir las técnicas tradicionales explicadas en la sección \ref{subsec:deteccion_malware} y no es posible depender de reglas predefinidas. Esto ha puesto de manifiesto la necesidad de evolucionar al mismo ritmo las técnicas de detección y ha hecho que las técnicas tradicionales resulten insuficientes por si solas. Con el rápido crecimiento que está teniendo el aprendizaje automático, podría ser un gran aliado si se usa de la forma adecuada, ya que es una herramienta muy potente capaz de detectar amenazas aún no conocidas.

\vspace{1em}

El uso de algoritmos de aprendizaje automático permite automatizar el análisis de grandes volúmenes de datos y adaptarse a la evolución de las amenazas. Además, existe una gran variedad de modelos, lo que posibilita evaluar diferentes estrategias e identificar los métodos más precisos, eficientes y escalables. Por otro lado, es posible volver a entrenar usando nuevos datos y adaptarse rápidamente a los cambios que surjan \cite{campus}.

\section{Definición del problema}
\label{sec:definicion}

A continuación, se definen los principales problemas y preguntas que nos hemos encontrado:

\begin{itemize}
	\item Detectar nuevas amenazas sin necesidad de conocerlas previamente.
	\item Dificultad de seleccionar el algoritmo más adecuado.
	\item Limitaciones de recursos computacionales.
	\item ¿Cómo afectan las características de cada conjunto al rendimiento de los modelos?
	\item ¿Qué variables son más influyentes en la clasificación?
\end{itemize}

\section{Objetivos}
\label{sec:objetivos}

A partir de los problemas presentados en la sección \ref{sec:definicion}, podemos establecer una serie de objetivos que definirán el desarrollo del estudio del que trata este proyecto. Los objetivos se pueden dividir en dos tipos: generales y específicos. El primero es la columna vertebral del proyecto, el tema central sobre el que gira el estudio que se realizará. Los objetivos específicos dividen el objetivo principal en otros más concretos y deseables. A continuación, se expondrán ambos tipos.

\newpage
\subsection{Objetivo general}
\label{subsec:obj_gen}

El objetivo principal para este estudio es comparar distintos algoritmos de aprendizaje automático haciendo uso de conjuntos de datos de \textit{malware}. Este objetivo se centra en los dos primeros problemas comentados en la sección \ref{sec:definicion}, tratando de conseguir detectar nuevas amenazas y tener una idea general de qué algoritmos se adaptan mejor a este propósito. Para ello, evaluaremos su eficiencia, precisión y viabilidad computacional.

\subsection{Objetivos específicos}
\label{subsec:obj_esp}

A partir del objetivo principal y del resto de problemas planteados, podemos concretar una serie propósitos más concretos:

\begin{itemize}
	\item Estudio teórico de distintos algoritmos en la detección de \textit{malware}.
	\item Obtención y análisis de bases de datos públicas de \textit{malware}.
	\item Implementación de metodologías de detección de \textit{malware} y su adaptación para uso en las bases de datos anteriores.
	\item Evaluar el rendimiento, eficiencia, precisión y viabilidad computacional de estos algoritmos.
	\item Identificar y analizar los métodos que se adaptan mejor al problema, destacando las ventajas e inconvenientes de cada uno de los algoritmos.
	\item Identificación de las variables e información más influyentes en la detección de \textit{malware}, particularmente para cada base de datos.
\end{itemize}

