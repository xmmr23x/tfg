\chapter{Metodología de trabajo}
\label{ch:metodologia}

\section{Enfoque metodológico}

% TODO: Explica el tipo de estudio realizado (comparativo, experimental, descriptivo) y justifica por qué este enfoque es el más adecuado para los objetivos del proyecto. (experimental, descriptivo, comparativo, etc.).

\section{Procedimiento seguido}

% TODO: Detalla la secuencia de actividades, desde la recolección y preparación de datos hasta la obtención y análisis de resultados, especificando el orden lógico de ejecución.

\section{Técnicas y herramientas empleadas}

% TODO: Describe el software, librerías, lenguajes de programación y hardware utilizados, así como su función específica dentro del desarrollo del proyecto.

\section{Criterios de selección de datos y modelos.}

\subsection{Selección del conjunto de datos}
\label{subsec:select_dataset}

En lo que a \textit{malware} se refiere, \textit{BODMAS} \cite{bodmas} es uno de los conjuntos de datos más completos en la actualidad, con la ventaja para este proyecto de ya estar procesado y tener una amplia bibliografía. Otra opción interesante puede ser \textit{VirusShare} \cite{virusshare}, ya que cuenta con más de 99 millones de muestras de \textit{malware} actualizadas pero tiene varios inconvenientes para este proyecto. El primero, es que no incluye muestras de \textit{software} no malicioso y el segundo, que necesita un procesamiento previo para extraer las características. Todo esto conlleva un aumento de tiempo considerable para la realización del proyecto. Otra de las opciones estudiadas ha sido \textit{theZoo} \cite{thezoo}. En cuanto a este repositorio hemos podido observar que tiene los mismos inconvenientes que \textit{VirusShare} y no tiene sus ventajas. Por último tenemos \textit{Microsoft Malware Classification} \cite{malware-classification}. En este caso tenemos un conjunto de datos muy amplio con casi medio \textit{terabyte}, pero además de los inconvenientes ya comentados en los anteriores conjuntos, solo incluye \textit{malware} que afecta a equipos \textit{Windows}, lo que limitaría considerablemente el alcance del estudio.

\vspace{1em}

Teniendo en cuenta todo lo comentado hasta ahora sobre los distintos conjuntos de datos considerados, hemos decidido usar \textit{BODMAS}, ya que es el que mejor se adapta a las necesidades del estudio

\subsection{Selección de los modelos}
\label{subsec:select_model}


