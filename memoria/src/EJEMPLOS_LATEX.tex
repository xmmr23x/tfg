\chapter{Ejemplos de código Latex}
\label{ch:codigos_ejemplo}
Aquí también se puede poner texto.

\section{Acrónimos}
Ejemplo de como poner un acrónimo \acrshort{iot}. Estos están definidos en el archivo ``anexo\_acronimos.text''.


\section{Comillas}
Mira el código para que veas como se hace bien el ``entrecomillado''. No se usan las comillas situadas en la tecla del dígito 2.

\section{Listas}
En esta sección se muestra como crear dos tipos de listas.

\subsection{Lista Enumerada}
\begin{enumerate}
\item Color rojo
\item Color verde
\item Color amarillo
\end{enumerate}



\subsection{Lista NO-Enumerada}
\begin{itemize}
\item Color rojo
\item Color verde
\item Color amarillo
\end{itemize}

\section{Como incluir FIGURAS}

\subsection{Figura única}
\label{sec:figura_unica}

\begin{figure}[H]
  \centering
  % include first image
  \includegraphics[width=.8\linewidth]{Imagenes/logo_uco}  
  \caption[Titulo de la figura para el índice de figuras. En este no se deben poner referencias a citas]{Título de la figura para el texto}
  \label{fig:figura_unica}
\end{figure}

\subsection{Figura con subfiguras}

En la Figura \ref{figT:subfiguras} se puede aprecia una figura compuesta por dos subfiguras.

\begin{figure}[H]
\begin{subfigure}{.5\textwidth}
  \centering
  % include first image
  \includegraphics[width=.8\linewidth]{Imagenes/logo_uco}  
  \caption{Título subfigura 1}
  \label{fig:sub-first}
\end{subfigure}
\begin{subfigure}{.5\textwidth}
  \centering
  % include second image
  \includegraphics[width=.8\linewidth]{Imagenes/logo_uco}  
  \caption{Título subfigura 2}
  \label{fig:sub-second}
\end{subfigure}
\caption{Titulo principal figura}
\label{figT:subfiguras}
\end{figure}

\subsection{Figura con subfiguras en distintas líneas}
\begin{figure}[H]
\begin{subfigure}{.5\textwidth}
  \centering
  % include first image
  \includegraphics[width=.8\linewidth]{Imagenes/logo_uco}  
  \caption{Título subfigura 1}
  \label{fig:sub-first}
\end{subfigure}
\begin{subfigure}{.5\textwidth}
  \centering
  % include second image
  \includegraphics[width=.8\linewidth]{Imagenes/logo_uco}  
  \caption{Título subfigura 2}
  \label{fig:sub-second}
\end{subfigure}

\begin{subfigure}{.5\textwidth}
  \centering
  % include first image
  \includegraphics[width=.8\linewidth]{Imagenes/logo_uco}  
  \caption{Título subfigura 3}
  \label{fig:sub-first}
\end{subfigure}
\begin{subfigure}{.5\textwidth}
  \centering
  % include second image
  \includegraphics[width=.8\linewidth]{Imagenes/logo_uco}  
  \caption{Título subfigura 4}
  \label{fig:sub-second}
\end{subfigure}

\caption{Figuras en multiples lineas}
\label{figT:subfiguras2}
\end{figure}


\section{Fórmulas Matemáticas}

\subsection{Formulas referenciables}
\begin{equation} \label{eq:erl} a = bq + r \end{equation} donde \eqref{eq:erl} es verdadera si $a$ y $b$ son enteros con $b \neq c$. 

\subsection{Fórmulas que no se van a referenciar}
\begin{align*} 
X = 0,42 \ \tau_{625} + 0.35 \ \tau_{550} + 0.21 \ \tau_{445} \\
Y = 0,20 \ \tau_{625} + 0.63 \ \tau_{550} + 0.17 \ \tau_{495} \\
Z = 0,24 \ \tau_{495} + 0.94 \ \tau_{445} 
\end{align*}

Y con estos valores se calculan las coordenadas \textit{x} e \textit{y}:

\begin{align*} 
x = X / (X + Y + Z) \quad y = Y / (X + Y + Z)
\end{align*}

\section{Diagramas usando TikZ}
\url{https://es.overleaf.com/learn/latex/LaTeX_Graphics_using_TikZ%3A_A_Tutorial_for_Beginners_(Part_1)%E2%80%94Basic_Drawing}

\subsection{Ejemplo diagrama de flujo}

\begin{diagrama}
\centering
 
    \begin{tikzpicture}[node distance=2cm]

    \node (start) [startstop] {Start};
    \node (in1) [io, below of=start] {Input};
    \node (pro1) [process, below of=in1] {Process 1};
    \node (dec1) [decision, below of=pro1, yshift=-0.5cm] {Decision 1};

    \node (pro2a) [process, below of=dec1, yshift=-0.5cm] {Process 2a text text text text text text text};

    \node (pro2b) [process, right of=dec1, xshift=2cm] {Process 2b};
    \node (out1) [io, below of=pro2a] {Output};
    \node (stop) [startstop, below of=out1] {Stop};

    \draw [arrow] (start) -- (in1);
    \draw [arrow] (in1) -- (pro1);
    \draw [arrow] (pro1) -- (dec1);
    \draw [arrow] (dec1) -- node[anchor=east] {yes} (pro2a);
    \draw [arrow] (dec1) -- node[anchor=south] {no} (pro2b);
    \draw [arrow] (pro2b) |- (pro1);
    \draw [arrow] (pro2a) -- (out1);
    \draw [arrow] (out1) -- (stop);

    \end{tikzpicture}

	\caption{Ejemplo diagrama}
	\label{diagrama:ejemplo_diagrama}
\end{diagrama}

\section{Referencia a distintas secciones y figuras}
En la sección \ref{sec:figura_unica} se puede ver la Figura \ref{fig:figura_unica}.

En el capítulo \ref{ch:presupuesto} se puede ver el código para crear una tabla. La Tabla \ref{tabla:presupuesto} muestra un presupuesto.