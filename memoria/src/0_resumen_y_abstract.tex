\chapter*{Resumen}
\addcontentsline{toc}{chapter}{Resumen} % si queremos que aparezca en el índice
\markboth{Resumen}{Resumen} % encabezado

Este proyecto se centra en la aplicación de técnicas de aprendizaje automático para la detección de \textit{malware}. El objetivo principal es comparar distintos algoritmos, tanto los estudiados en el plan de estudios de Ingeniería Informática como otros que no forman parte de la formación académica, con el fin de determinar cuáles se adaptan mejor a este problema.

\vspace{1em}

Para ello, se utilizan bases de datos públicas de \textit{malware}, adaptadas mediante técnicas de preprocesamiento, balanceo y reducción de la dimensionalidad. Posteriormente, se implementa un protocolo experimental que incluye la optimización de hiperparámetros, la validación cruzada estratificada y la repetición de experimentos con diferentes semillas para garantizar la reproducibilidad.

\vspace{1em}

Finalmente, se analizan los resultados obtenidos en términos de precisión, eficiencia y viabilidad computacional, destacando las ventajas e inconvenientes de cada enfoque y aportando una visión comparativa que pueda servir de referencia para futuros estudios en la detección automática de \textit{malware}.

\vspace{2em}

\textbf{Palabras clave:} Aprendizaje automático, Detección de \textit{malware}, Clasificación, Ciberseguridad

\chapter*{Abstract}
\addcontentsline{toc}{chapter}{Abstract} % si queremos que aparezca en el índice
\markboth{Abstract}{Abstract} % encabezado

This project focusses on the application of machine learning techniques for malware detection. The main objective is to compare different algorithms, both those studied in the Computer Engineering curriculum and others not included in the formal academic training, to determine which are better suited to this problem.

\vspace{1em}

For this purpose, public malware datasets are used, adapted through preprocessing, data balancing and dimensionality reduction techniques. Afterwards, an experimental protocol including hyperparameter optimization, stratified cross-validation and the repetition of experiments with different random seeds, is implemented to guarantee reproducibility.

\vspace{1em}

Finally, the results are analyzed in terms of accuracy, efficiency and computational viability, with special focus on the advantages and disadvantages of each approach and providing a comparative perspective to serve as a reference for future research on automated malware detection.

\vspace{2em}

\textbf{Keywords:} Machine Learnig, Malware detection, Clasification, Cibersecurity.