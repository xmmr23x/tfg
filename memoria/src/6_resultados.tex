\chapter{Resultados y discusión}
\label{ch:resultados}

% TODO: ESCRIBIR INTRODUCCION

\section{Clasificación binaria}
\label{sec:clas_binaria}

% TODO: escribir introducción para clasificaicon binaria, este texto es de experimentacion
En esta fase se lleva a cabo la implementación práctica del estudio, haciendo uso de los modelos de aprendizaje automático implementados principalmente en la librería \textit{Scikit-Learn} de \textit{python} y descritos en el capítulo \ref{ch:metodologia}. Para ello se configuran los entornos necesarios para su entrenamiento y evaluación, se establecen las métricas de rendimiento, los procedimientos de prueba y los escenarios de experimentación que permitirán obtener resultados consistentes y comparables. El objetivo es verificar, mediante pruebas controladas, la efectividad de cada método en la detección de \textit{malware}.

La parte experimental de este proyecto se estudiará desde dos enfoques distintos. Por un lado se evaluarán los modelos seleccionados en la detección de \textit{malware}, es decir, se realizarán pruebas de clasificación binaria donde se estudiará sin un un patrón corresponde a un programa malicioso o no. Por otro, se estudiará si, para estos mismos patrones, es posible realizar una clasificación más exhaustiva y reconocer con que tipo de \textit{malware} se corresponde cada patrón.

\subsection{Árboles de decisión}
\label{subsec:dt_bin}

\begin{table}[th]
	\centering
	\begin{tabular}{ |c|c|c|c|c|c|c| }
		\hline
		\rowcolor{LightCyan}
		 & \multicolumn{3}{c|}{Entrenamiento} & \multicolumn{3}{c|}{Test} \\
		\hline
		\rowcolor{LightCyan}
		 Estado aleatorio & Acc & MS & F1 & Acc & MS & F1 \\
		\hline
		0 & 1.000 & 1.000 & 1.000 & 0.951 & 0.942 & 1.000 \\
		1 & 1.000 & 1.000 & 1.000 & 0.944 & 0.935 & 1.000 \\
		2 & 1.000 & 1.000 & 1.000 & 0.942 & 0.935 & 1.000 \\
		3 & 1.000 & 1.000 & 1.000 & 0.948 & 0.938 & 1.000 \\
		4 & 1.000 & 1.000 & 1.000 & 0.953 & 0.946 & 1.000 \\
		5 & 0.998 & 0.997 & 1.000 & 0.947 & 0.936 & 1.000 \\
		6 & 0.998 & 0.997 & 1.000 & 0.947 & 0.941 & 1.000 \\
		7 & 1.000 & 1.000 & 1.000 & 0.949 & 0.946 & 1.000 \\
		8 & 0.999 & 0.998 & 1.000 & 0.948 & 0.937 & 1.000 \\
		9 & 1.000 & 1.000 & 1.000 & 0.950 & 0.940 & 1.000 \\
		Mean & 0.999 & 0.999 & 1.000 & 0.948 & 0.940 & 1.000 \\
		STD & 0.001 & 0.001 & 0.000 & 0.003 & 0.004 & 0.000 \\
		\hline
	\end{tabular}
	\caption{Clasificación binaria con \textit{DecisionTreeClassifier}}
	\label{tabla:dt_bin}
\end{table}


\subsection{\textit{Random forest}}
\label{subsec:rf_bin}

\begin{table}[th]
	\centering
	\begin{tabular}{ |c|c|c|c|c|c|c| }
		\hline
		\rowcolor{LightCyan}
		 & \multicolumn{3}{c|}{Entrenamiento} & \multicolumn{3}{c|}{Test} \\
		\hline
		\rowcolor{LightCyan}
		 Estado aleatorio & Acc & MS & F1 & Acc & MS & F1 \\
		\hline
		0 & 0.980 & 0.928 & 0.989 & 0.939 & 0.524 & 0.972 \\
		1 & 0.980 & 0.929 & 0.990 & 0.939 & 0.500 & 0.973 \\
		2 & 0.980 & 0.927 & 0.989 & 0.941 & 0.429 & 0.974 \\
		3 & 0.979 & 0.923 & 0.989 & 0.939 & 0.444 & 0.973 \\
		4 & 0.982 & 0.931 & 0.990 & 0.939 & 0.500 & 0.974 \\
		5 & 0.980 & 0.929 & 0.990 & 0.937 & 0.609 & 0.971 \\
		6 & 0.978 & 0.920 & 0.988 & 0.938 & 0.550 & 0.971 \\
		7 & 0.980 & 0.929 & 0.990 & 0.939 & 0.562 & 0.972 \\
		8 & 0.982 & 0.934 & 0.991 & 0.938 & 0.489 & 0.971 \\
		9 & 0.989 & 0.956 & 0.992 & 0.936 & 0.400 & 0.972 \\
		Mean & 0.981 & 0.931 & 0.990 & 0.939 & 0.501 & 0.972 \\
		STD & 0.003 & 0.010 & 0.001 & 0.001 & 0.064 & 0.001 \\
		\hline
	\end{tabular}
	\caption{Clasificación binaria con \textit{RandomForestClassifier}}
	\label{tabla:rf_bin}
\end{table}


\subsection{\textit{K-NN}}
\label{subsec:knn_bin}

\begin{table}[th]
	\centering
	\begin{tabular}{ |c|c|c|c|c|c|c| }
		\hline
		\rowcolor{LightCyan}
		 & \multicolumn{3}{c|}{Entrenamiento} & \multicolumn{3}{c|}{Test} \\
		\hline
		\rowcolor{LightCyan}
		 Estado aleatorio & Acc & MS & F1 & Acc & MS & F1 \\
		\hline
		0 & 1.000 & 1.000 & 1.000 & 0.947 & 0.935 & 1.000 \\
		1 & 1.000 & 1.000 & 1.000 & 0.949 & 0.938 & 1.000 \\
		2 & 1.000 & 1.000 & 1.000 & 0.939 & 0.926 & 1.000 \\
		3 & 1.000 & 1.000 & 1.000 & 0.949 & 0.937 & 1.000 \\
		4 & 1.000 & 1.000 & 1.000 & 0.949 & 0.936 & 1.000 \\
		5 & 1.000 & 1.000 & 1.000 & 0.946 & 0.932 & 1.000 \\
		6 & 1.000 & 1.000 & 1.000 & 0.946 & 0.931 & 1.000 \\
		7 & 1.000 & 1.000 & 1.000 & 0.944 & 0.934 & 1.000 \\
		8 & 1.000 & 1.000 & 1.000 & 0.947 & 0.935 & 1.000 \\
		9 & 1.000 & 1.000 & 1.000 & 0.949 & 0.940 & 1.000 \\
		Mean & 1.000 & 1.000 & 1.000 & 0.947 & 0.934 & 1.000 \\
		STD & 0.000 & 0.000 & 0.000 & 0.003 & 0.004 & 0.000 \\
		\hline
	\end{tabular}
	\caption{Clasificación binara con \textit{KNeighborscClassifier}}
	\label{tabla:knn_bin}
\end{table}


\subsection{Máquinas de vectores de soporte}
\label{subsec:svm_bin}

\begin{table}[th]
	\centering
	\begin{tabular}{ |c|c|c|c|c|c|c| }
		\hline
		\rowcolor{LightCyan}
		 & \multicolumn{3}{c|}{Entrenamiento} & \multicolumn{3}{c|}{Test} \\
		\hline
		\rowcolor{LightCyan}
		 Estado aleatorio & Acc & MS & F1 & Acc & MS & F1 \\
		\hline
		0 & 0.757 & 0.656 & 1.000 & 0.764 & 0.672 & 1.000 \\
		1 & 0.761 & 0.671 & 1.000 & 0.769 & 0.681 & 1.000 \\
		2 & 0.766 & 0.702 & 1.000 & 0.757 & 0.699 & 1.000 \\
		3 & 0.762 & 0.700 & 1.000 & 0.766 & 0.704 & 1.000 \\
		4 & 0.760 & 0.684 & 1.000 & 0.768 & 0.696 & 1.000 \\
		5 & 0.761 & 0.663 & 1.000 & 0.753 & 0.655 & 1.000 \\
		6 & 0.762 & 0.702 & 1.000 & 0.762 & 0.683 & 1.000 \\
		7 & 0.763 & 0.699 & 1.000 & 0.759 & 0.697 & 1.000 \\
		8 & 0.766 & 0.704 & 1.000 & 0.758 & 0.693 & 1.000 \\
		9 & 0.760 & 0.662 & 1.000 & 0.758 & 0.666 & 1.000 \\
		Mean & 0.762 & 0.684 & 1.000 & 0.762 & 0.685 & 1.000 \\
		STD & 0.003 & 0.020 & 0.000 & 0.005 & 0.016 & 0.000 \\
		\hline
	\end{tabular}
	\caption{Clasificación binaria con \textit{SVC}}
	\label{tabla:svm_bin}
\end{table}


\subsection{\textit{Ridge}}
\label{subsec:ridge_bin}

\begin{table}[th]
	\centering
	\begin{tabular}{ |c|c|c|c|c|c|c| }
		\hline
		\rowcolor{LightCyan}
		 & \multicolumn{3}{c|}{Entrenamiento} & \multicolumn{3}{c|}{Test} \\
		\hline
		\rowcolor{LightCyan}
		 Estado aleatorio & Acc & MS & F1 & Acc & MS & F1 \\
		\hline
		0 & 0.649 & 0.549 & 1.000 & 0.648 & 0.530 & 1.000 \\
		1 & 0.645 & 0.558 & 1.000 & 0.655 & 0.569 & 1.000 \\
		2 & 0.652 & 0.573 & 1.000 & 0.645 & 0.564 & 1.000 \\
		3 & 0.649 & 0.567 & 1.000 & 0.653 & 0.570 & 1.000 \\
		4 & 0.651 & 0.573 & 1.000 & 0.651 & 0.573 & 1.000 \\
		5 & 0.647 & 0.562 & 1.000 & 0.648 & 0.558 & 1.000 \\
		6 & 0.648 & 0.556 & 1.000 & 0.650 & 0.573 & 1.000 \\
		7 & 0.651 & 0.571 & 1.000 & 0.650 & 0.573 & 1.000 \\
		8 & 0.651 & 0.564 & 1.000 & 0.639 & 0.551 & 1.000 \\
		9 & 0.650 & 0.563 & 1.000 & 0.645 & 0.551 & 1.000 \\
		Mean & 0.649 & 0.564 & 1.000 & 0.648 & 0.561 & 1.000 \\
		STD & 0.002 & 0.008 & 0.000 & 0.005 & 0.014 & 0.000 \\
		\hline
	\end{tabular}
	\caption{Clasificación binaria con \textit{RidgeClassifier}}
	\label{tabla:ridge_bin}
\end{table}


\subsection{Redes neuronales: Perceptrón multicapa}
\label{subsec:mlp_bin}

\begin{table}[th]
	\centering
	\begin{tabular}{ |c|c|c|c|c|c|c| }
		\hline
		\rowcolor{LightCyan}
		 & \multicolumn{3}{c|}{Entrenamiento} & \multicolumn{3}{c|}{Test} \\
		\hline
		\rowcolor{LightCyan}
		 Estado aleatorio & Acc & MS & F1 & Acc & MS & F1 \\
		\hline
		0 & 0.783 & 0.771 & 1.000 & 0.789 & 0.778 & 1.000 \\
		1 & 0.788 & 0.736 & 1.000 & 0.792 & 0.740 & 1.000 \\
		2 & 0.788 & 0.750 & 1.000 & 0.782 & 0.739 & 1.000 \\
		3 & 0.733 & 0.605 & 1.000 & 0.737 & 0.609 & 1.000 \\
		4 & 0.767 & 0.759 & 1.000 & 0.769 & 0.760 & 1.000 \\
		5 & 0.790 & 0.736 & 1.000 & 0.783 & 0.730 & 1.000 \\
		6 & 0.777 & 0.772 & 1.000 & 0.783 & 0.781 & 1.000 \\
		7 & 0.774 & 0.767 & 1.000 & 0.770 & 0.763 & 1.000 \\
		8 & 0.778 & 0.704 & 1.000 & 0.772 & 0.705 & 1.000 \\
		9 & 0.788 & 0.762 & 1.000 & 0.784 & 0.751 & 1.000 \\
		Mean & 0.776 & 0.736 & 1.000 & 0.776 & 0.736 & 1.000 \\
		STD & 0.017 & 0.051 & 0.000 & 0.016 & 0.050 & 0.000 \\
		\hline
	\end{tabular}
	\caption{Clasificación binaria con \textit{MLPClassifier}}
	\label{tabla:mlp_bin}
\end{table}


\subsection{\textit{Light Gradient Boosting Machine}}
\label{subsec:lgbm_bin}

\begin{table}[th]
	\centering
	\begin{tabular}{ |c|c|c|c|c|c|c| }
		\hline
		\rowcolor{LightCyan}
		 & \multicolumn{3}{c|}{Entrenamiento} & \multicolumn{3}{c|}{Test} \\
		\hline
		\rowcolor{LightCyan}
		 Estado aleatorio & Acc & MS & F1 & Acc & MS & F1 \\
		\hline
		0 & 0.984 & 0.981 & 1.000 & 0.953 & 0.952 & 1.000 \\
		1 & 0.984 & 0.980 & 1.000 & 0.951 & 0.947 & 1.000 \\
		2 & 0.985 & 0.983 & 1.000 & 0.949 & 0.946 & 1.000 \\
		3 & 0.985 & 0.982 & 1.000 & 0.952 & 0.951 & 1.000 \\
		4 & 0.984 & 0.981 & 1.000 & 0.950 & 0.945 & 1.000 \\
		5 & 0.985 & 0.981 & 1.000 & 0.949 & 0.948 & 1.000 \\
		6 & 0.985 & 0.982 & 1.000 & 0.952 & 0.949 & 1.000 \\
		7 & 0.986 & 0.984 & 1.000 & 0.948 & 0.947 & 1.000 \\
		8 & 0.984 & 0.979 & 1.000 & 0.953 & 0.952 & 1.000 \\
		9 & 0.989 & 0.989 & 1.000 & 0.953 & 0.950 & 1.000 \\
		Mean & 0.985 & 0.982 & 1.000 & 0.951 & 0.949 & 1.000 \\
		STD & 0.002 & 0.003 & 0.000 & 0.002 & 0.002 & 0.000 \\
		\hline
	\end{tabular}
	\caption{Clasificación binaria con \textit{LGBMClassifier}}
	\label{tabla:lgbm_bin}
\end{table}


\section{Clasificación multiclase}
\label{sec:clas_multi}

\subsection{Árboles de decisión}
\label{subsec:dt_multi}

\begin{table}[th]
	\centering
	\begin{tabular}{ |c|c|c|c|c|c|c| }
		\hline
		\rowcolor{LightCyan}
		 & \multicolumn{3}{c|}{Entrenamiento} & \multicolumn{3}{c|}{Test} \\
		\hline
		\rowcolor{LightCyan}
		 Estado aleatorio & Acc & MS & F1 & Acc & MS & F1 \\
		\hline
		0 & 0.980 & 0.928 & 0.989 & 0.939 & 0.524 & 0.972 \\
		1 & 0.980 & 0.929 & 0.990 & 0.939 & 0.500 & 0.973 \\
		2 & 0.980 & 0.927 & 0.989 & 0.941 & 0.429 & 0.974 \\
		3 & 0.979 & 0.923 & 0.989 & 0.939 & 0.444 & 0.973 \\
		4 & 0.982 & 0.931 & 0.990 & 0.939 & 0.500 & 0.974 \\
		5 & 0.980 & 0.929 & 0.990 & 0.937 & 0.609 & 0.971 \\
		6 & 0.978 & 0.920 & 0.988 & 0.938 & 0.550 & 0.971 \\
		7 & 0.980 & 0.929 & 0.990 & 0.939 & 0.562 & 0.972 \\
		8 & 0.982 & 0.934 & 0.991 & 0.938 & 0.489 & 0.971 \\
		9 & 0.989 & 0.956 & 0.992 & 0.936 & 0.400 & 0.972 \\
		Mean & 0.981 & 0.931 & 0.990 & 0.939 & 0.501 & 0.972 \\
		STD & 0.003 & 0.010 & 0.001 & 0.001 & 0.064 & 0.001 \\
		\hline
	\end{tabular}
	\caption{Clasificación multiclase con \textit{DecisionTreeClassifier}}
	\label{tabla:dt_multi}
\end{table}


\subsection{\textit{Random forest}}
\label{subsec:rf_multi}

\begin{table}[th]
	\centering
	\begin{tabular}{ |c|c|c|c|c|c|c| }
		\hline
		\rowcolor{LightCyan}
		 & \multicolumn{3}{c|}{Entrenamiento} & \multicolumn{3}{c|}{Test} \\
		\hline
		\rowcolor{LightCyan}
		 Estado aleatorio & Acc & MS & F1 & Acc & MS & F1 \\
		\hline
		0 & 0.981 & 0.926 & 0.990 & 0.951 & 0.524 & 0.977 \\
		1 & 0.981 & 0.926 & 0.990 & 0.953 & 0.735 & 0.978 \\
		2 & 0.981 & 0.926 & 0.990 & 0.954 & 0.429 & 0.978 \\
		3 & 0.980 & 0.923 & 0.990 & 0.954 & 0.500 & 0.978 \\
		4 & 0.981 & 0.926 & 0.990 & 0.954 & 0.500 & 0.979 \\
		5 & 0.981 & 0.927 & 0.990 & 0.952 & 0.638 & 0.977 \\
		6 & 0.980 & 0.923 & 0.990 & 0.952 & 0.550 & 0.977 \\
		7 & 0.981 & 0.927 & 0.991 & 0.954 & 0.500 & 0.978 \\
		8 & 0.981 & 0.926 & 0.990 & 0.953 & 0.471 & 0.978 \\
		9 & 0.981 & 0.926 & 0.990 & 0.953 & 0.400 & 0.978 \\
		Mean & 0.981 & 0.926 & 0.990 & 0.953 & 0.525 & 0.978 \\
		STD & 0.000 & 0.002 & 0.000 & 0.001 & 0.098 & 0.001 \\
		\hline
	\end{tabular}
	\caption{Clasificación multiclase con \textit{RandomForestClassifier}}
	\label{tabla:rf_multi}
\end{table}


\subsection{\textit{K-NN}}
\label{subsec:knn_multi}

\begin{table}[th]
	\centering
	\begin{tabular}{ |c|c|c|c|c|c|c| }
		\hline
		\rowcolor{LightCyan}
		 & \multicolumn{3}{c|}{Entrenamiento} & \multicolumn{3}{c|}{Test} \\
		\hline
		\rowcolor{LightCyan}
		 Estado aleatorio & Acc & MS & F1 & Acc & MS & F1 \\
		\hline
		0 & 0.994 & 0.811 & 0.997 & 0.940 & 0.524 & 0.976 \\
		1 & 0.994 & 0.794 & 0.996 & 0.943 & 0.500 & 0.977 \\
		2 & 0.994 & 0.811 & 0.997 & 0.940 & 0.357 & 0.977 \\
		3 & 0.994 & 0.815 & 0.996 & 0.942 & 0.389 & 0.978 \\
		4 & 0.994 & 0.810 & 0.997 & 0.941 & 0.375 & 0.976 \\
		5 & 0.994 & 0.807 & 0.996 & 0.940 & 0.435 & 0.976 \\
		6 & 0.994 & 0.802 & 0.996 & 0.941 & 0.500 & 0.976 \\
		7 & 0.994 & 0.849 & 0.996 & 0.940 & 0.500 & 0.976 \\
		8 & 0.994 & 0.817 & 0.996 & 0.939 & 0.529 & 0.976 \\
		9 & 0.994 & 0.834 & 0.996 & 0.940 & 0.400 & 0.976 \\
		Mean & 0.994 & 0.815 & 0.996 & 0.941 & 0.451 & 0.976 \\
		STD & 0.000 & 0.016 & 0.000 & 0.001 & 0.067 & 0.001 \\
		\hline
	\end{tabular}
	\caption{Clasificación multiclase con \textit{KNeighborsClassifier}}
	\label{tabla:knn_multi}
\end{table}


\subsection{Máquinas de vectores de soporte}
\label{subsec:svm_multi}

En este caso, el proceso de entrenamiento presentó una mayor complejidad y dificultad para obtener resultados comparables con los de otros modelos evaluados, principalmente debido a las limitaciones del equipo utilizado. El elevado tiempo requerido para el entrenamiento sin ajuste de parámetros, junto con los resultados poco satisfactorios obtenidos para las dos semillas empleadas ---con una precisión aproximada del 20\%---, motivaron la decisión de no continuar con las máquinas de vectores de soporte para la clasificación multiclase. No obstante, estos resultados no indican que el modelo sea inadecuado para el problema planteado, sino que tiene una mayor exigencia en cuanto a los recursos necesarios para su entrenamiento.

\subsection{\textit{Ridge}}
\label{subsec:ridge_multi}

\begin{table}[th]
	\centering
	\begin{tabular}{ |c|c|c|c|c|c|c| }
		\hline
		\rowcolor{LightCyan}
		 & \multicolumn{3}{c|}{Entrenamiento} & \multicolumn{3}{c|}{Test} \\
		\hline
		\rowcolor{LightCyan}
		 Estado aleatorio & Acc & MS & F1 & Acc & MS & F1 \\
		\hline
		0 & 0.189 & 0.000 & 0.301 & 0.186 & 0.000 & 0.299 \\
		1 & 0.195 & 0.000 & 0.308 & 0.191 & 0.000 & 0.307 \\
		2 & 0.173 & 0.000 & 0.284 & 0.176 & 0.000 & 0.287 \\
		3 & 0.172 & 0.000 & 0.283 & 0.172 & 0.000 & 0.280 \\
		4 & 0.185 & 0.000 & 0.297 & 0.189 & 0.000 & 0.305 \\
		5 & 0.187 & 0.000 & 0.300 & 0.189 & 0.000 & 0.303 \\
		6 & 0.170 & 0.000 & 0.280 & 0.166 & 0.000 & 0.274 \\
		7 & 0.186 & 0.000 & 0.299 & 0.191 & 0.000 & 0.303 \\
		8 & 0.187 & 0.000 & 0.300 & 0.187 & 0.000 & 0.301 \\
		9 & 0.171 & 0.000 & 0.282 & 0.173 & 0.000 & 0.284 \\
		Mean & 0.182 & 0.000 & 0.293 & 0.182 & 0.000 & 0.294 \\
		STD & 0.009 & 0.000 & 0.010 & 0.009 & 0.000 & 0.012 \\
		\hline
	\end{tabular}
	\caption{Clasificación multiclase con \textit{RidgeClassifier}}
	\label{tabla:ridge_multi}
\end{table}


\subsection{Redes neuronales: Perceptrón multicapa}
\label{subsec:mlp_multi}

\begin{table}[th]
	\centering
	\begin{tabular}{ |c|c|c|c|c|c|c| }
		\hline
		\rowcolor{LightCyan}
		 & \multicolumn{3}{c|}{Entrenamiento} & \multicolumn{3}{c|}{Test} \\
		\hline
		\rowcolor{LightCyan}
		 Estado aleatorio & Acc & MS & F1 & Acc & MS & F1 \\
		\hline
		0 & 0.725 & 0.000 & 0.885 & 0.722 & 0.000 & 0.883 \\
		1 & 0.724 & 0.000 & 0.901 & 0.724 & 0.000 & 0.900 \\
		2 & 0.724 & 0.000 & 0.885 & 0.723 & 0.000 & 0.885 \\
		3 & 0.679 & 0.000 & 0.885 & 0.681 & 0.000 & 0.888 \\
		4 & 0.730 & 0.000 & 0.904 & 0.735 & 0.000 & 0.902 \\
		5 & 0.721 & 0.000 & 0.888 & 0.717 & 0.000 & 0.884 \\
		6 & 0.724 & 0.000 & 0.889 & 0.723 & 0.000 & 0.888 \\
		7 & 0.711 & 0.000 & 0.885 & 0.711 & 0.000 & 0.884 \\
		8 & 0.719 & 0.000 & 0.910 & 0.720 & 0.000 & 0.910 \\
		9 & 0.716 & 0.000 & 0.886 & 0.718 & 0.000 & 0.885 \\
		Mean & 0.717 & 0.000 & 0.892 & 0.718 & 0.000 & 0.891 \\
		STD & 0.014 & 0.000 & 0.009 & 0.014 & 0.000 & 0.009 \\
		\hline
	\end{tabular}
	\caption{Clasificación multiclase con \textit{MLPClassifier}}
	\label{tabla:mlp_multi}
\end{table}


\subsection{\textit{Light Gradient Boosting Machine}}
\label{subsec:lgbm_multi}

\begin{table}[th]
	\centering
	\begin{tabular}{ |c|c|c|c|c|c|c| }
		\hline
		\rowcolor{LightCyan}
		 & \multicolumn{3}{c|}{Entrenamiento} & \multicolumn{3}{c|}{Test} \\
		\hline
		\rowcolor{LightCyan}
		 Estado aleatorio & Acc & MS & F1 & Acc & MS & F1 \\
		\hline
		0 & 0.938 & 0.821 & 0.965 & 0.916 & 0.600 & 0.953 \\
		1 & 0.936 & 0.820 & 0.964 & 0.916 & 0.735 & 0.953 \\
		2 & 0.890 & 0.749 & 0.941 & 0.884 & 0.357 & 0.936 \\
		3 & 0.323 & 0.000 & 0.460 & 0.327 & 0.000 & 0.460 \\
		4 & 0.888 & 0.747 & 0.940 & 0.880 & 0.500 & 0.936 \\
		5 & 0.938 & 0.828 & 0.967 & 0.917 & 0.565 & 0.955 \\
		6 & 0.893 & 0.758 & 0.943 & 0.881 & 0.550 & 0.935 \\
		7 & 0.936 & 0.821 & 0.964 & 0.917 & 0.562 & 0.952 \\
		8 & 0.891 & 0.750 & 0.941 & 0.880 & 0.588 & 0.933 \\
		9 & 0.893 & 0.760 & 0.942 & 0.883 & 0.467 & 0.936 \\
		Mean & 0.853 & 0.706 & 0.903 & 0.840 & 0.492 & 0.895 \\
		STD & 0.187 & 0.250 & 0.156 & 0.181 & 0.198 & 0.153 \\
		\hline
	\end{tabular}
	\caption{Clasificación multiclase con \textit{LGBMClassifier}}
	\label{tabla:lgbm_multi}
\end{table}
