\documentclass[a4paper, 12pt]{book}
\usepackage[spanish,es-tabla]{babel}
\usepackage[utf8]{inputenc}

\usepackage[left = 35mm, right = 25mm, top = 25mm, height = 206mm]{geometry}
\usepackage{fancyhdr}
\usepackage{graphicx}
\usepackage{lipsum}
\usepackage[square,numbers]{natbib}
\usepackage[skip=12pt,font=small]{caption}  %Tamaño de la leyenda
\usepackage{subcaption}
\usepackage{amsmath}
\usepackage{afterpage}
\usepackage{titlesec}
\usepackage{color, colortbl}
\usepackage{listings}
\usepackage{xcolor}
\usepackage{float}
\usepackage{eurosym}
\usepackage{enumitem}
\usepackage{amssymb}
\usepackage{array}
\usepackage{wrapfig}
\usepackage{multirow}
\usepackage{tabularx}
\usepackage{url}
\usepackage{eurosym}
\usepackage{appendix}
\usepackage{pdfpages}
\usepackage{pdflscape} %poner pdf horizontal
\usepackage[hidelinks]{hyperref}%para el hiperlink de las secciones
\usepackage[newfloat]{minted}
\usepackage[nottoc]{tocbibind}
\usepackage{glossaries}


\usepackage{tikz} % Diagramas de flujo
\usetikzlibrary{shapes.geometric, arrows} %Libería de flechas

\usepackage{datetime}
\newdateformat{monthyeardate}{\monthname, \THEYEAR}

% Entorno para código
\newenvironment{code}{\captionsetup{type=listing}}{}

% Cambiamos el caption de los listing a español
\renewcommand{\listingname}{Código}

 % Cambiamos el idioma del índice de códigos
\renewcommand*{\listlistingname}{Índice de Códigos}

\newfloat{diagrama}{htbp}{loc}
\floatname{diagrama}{Diagrama}
\newcommand{\listofdiagrams}{\listof{diagrama}{Índice de Diagramas de Flujo}}
\floatstyle{plaintop}
\restylefloat{table}

\graphicspath{ {Imagenes/} }


\input{Configuracion/confg_colores}
\input{Configuracion/confg_codigo}
\input{Configuracion/confg_diagrama_flujo}
%----------------------------------------------------------------------------------------
%	ENCABEZADOS Y PIE DE PAGINA
%----------------------------------------------------------------------------------------

\pagestyle{fancy}
	\fancypagestyle{cuerpo}{
		\fancyfoot{}
		\fancyfoot[L]{Trabajo Fin de Grado} %Autor
		\fancyfoot[C]{}
		\fancyfoot[R]{- \thepage \ -}
		\fancyhead{}
		\fancyhead[L]{\leftmark} %Nombre del documento
		\fancyhead[R]{\includegraphics[height=1.8cm]{logo_uco}} %Logo Universidad
		\renewcommand{\headrulewidth}{0.4pt}
		\renewcommand{\footrulewidth}{0.4pt}
		\setlength{\headheight}{65pt}
	}
	\fancypagestyle{indice}{
		\fancyfoot{}
		\fancyfoot[L]{\AutorTFG} %Autor
		\fancyfoot[R]{- \thepage \ -} %Asignatura
		\fancyhead{}
		\fancyhead[L]{Trabajo Fin de Grado} %Nombre del documento
		\fancyhead[R]{\includegraphics[height=1.8cm]{logo_uco}} %Logo
		\renewcommand{\headrulewidth}{0.4pt}
		\renewcommand{\footrulewidth}{0.4pt}
		\setlength{\headheight}{65pt}
	}

	\fancypagestyle{plain}{% % <-- this is new
		\fancyfoot{}
		\fancyfoot[L]{Trabajo Fin de Grado} %Autor
		\fancyfoot[C]{} %ecc
		\fancyfoot[R]{- \thepage \ -}
		\fancyhead{}
		\fancyhead[L]{\leftmark} %Nombre del documento
		\fancyhead[R]{\includegraphics[height=1.8cm]{logo_uco}} %Logo Universidad
		\renewcommand{\headrulewidth}{0.4pt}
		\renewcommand{\footrulewidth}{0.4pt}
		\setlength{\headheight}{65pt}
	}

\setcounter{secnumdepth}{4}

\pagenumbering{empty}


%%%%%%%%%
% ANEXO %
%%%%%%%%%

% Para que ponga ANEXO en vez de APENDICE
 \addto\captionsspanish{\renewcommand{\appendixname}{Anexo}} %para que ponga anexo y no apéndice.

\makeglossaries
\raggedbottom  % Quita huecos grandes en blanco
\begin{document}

%%%%%%%%%%%
% PORTADA %
%%%%%%%%%%%
\newcommand{\TituloTFG}{Estudio comparativo de métodos de aprendizaje automático en la detección de malware}
\newcommand{\AutorTFG}{Manuel Jesús Mariscal Romero}
\newcommand{\DirectorUno}{D. David Guijo Rubio}
\newcommand{\DirectorDos}{D. Víctor Manuel Vargas Yun}
\newcommand{\Grado}{Grado en Ingeniería Informática}



% Definición de colores usados en la portada
\definecolor{epsc:oscuro}{HTML}{280091}
\definecolor{epsc:medio}{HTML}{4C5CC5}
\definecolor{epsc:verde}{HTML}{00B299}
\definecolor{epsc:claro}{HTML}{3FCFD5}

\definecolor{PORTADA_0}{cmyk}{1.00,0.00,0.07,0.28,1.00}

\definecolor{PORTADA_1}{cmyk}{0.58,0.55,0.00,0.44,1.00}

\definecolor{friendlyGray}{HTML}{f0f0f0}

\thispagestyle{empty}

\begin{tikzpicture}[remember picture, overlay]
  % Top
  \node [anchor=north east, inner sep=0pt]  at (current page.north east)%
    {\includegraphics[height=6cm]{Portada/Img/topRightCorner.pdf}};
  % Bottom
  \node [anchor=south west, inner sep=0pt]  at (current page.south west)%
    {\includegraphics[height=6cm]{Portada/Img/bottomLeftCorner.pdf}};
  \node (uco) [anchor=south east, inner sep=0pt, xshift=-10mm, yshift=10mm]  at (current page.south east)%
    {\includegraphics[height=2cm]{Portada/Img/uco_escudo}};
  \node [anchor=south east, inner sep=0pt, xshift=-10mm]  at (uco.south west)%
    {\includegraphics[height=2cm]{Portada/Img/emblema-ing-industrial.pdf}};
    %{\includegraphics[height=2cm]{Portada/Img/emblema-ing-informatica.png}};
\end{tikzpicture}

% Título del trabajo, autor y directores
\begin{center}
\vspace*{2cm}
\vfill
\vfill
\vfill
\vfill
\includegraphics[width=12.5cm]{Portada/Img/LogotipoEPSC}
\vfill
\vfill

{\color{epsc:verde}
\large

\textbf{TRABAJO FIN DE GRADO}

\textit{\Grado}
} \vspace{1mm}

{\color{epsc:oscuro}

\textbf{\large \TituloTFG}

\vspace{14.35mm}
\begin{tabular}{rl}
\textbf{Autor:} & \AutorTFG\\
\textbf{Directores:} & \DirectorUno\\
                     & \DirectorDos\\
\end{tabular}

\vspace{6.79mm}
\textbf{\monthyeardate{\today}}        % Fecha de compilación en formato mes, año

}
\end{center}
\endinput

\newpage
\thispagestyle{empty}   % Página en blanco entre la portada y el comienzo del documento
\mbox{}
$\ $
\thispagestyle{empty} % para que no se numere esta pagina
\newpage
$\ $
\thispagestyle{empty} % para que no se numere esta pagina

%%%%%%%%%%%%%%%
% DEDICATORIA %
%%%%%%%%%%%%%%%
\pagenumbering{Roman} % para comenzar la numeracion de paginas en numeros romanos
% \pagestyle{indice}
% \thispagestyle{indice}
% \begin{flushright}
% \textit{Dedicado a \\
% mi familia}
% \end{flushright}

%%%%%%%%%%%%%%%%%%%
% AGRADECIMIENTOS %
%%%%%%%%%%%%%%%%%%%
% \chapter*{Agradecimientos}

\addcontentsline{toc}{chapter}{Agradecimientos} % si queremos que aparezca en el índice

\markboth{Agradecimientos}{Agradecimientos} % encabezado

RELLENAR



%%%%%%%%%%%%%%%%%%%%%%
% RESUMEN Y ABSTRACT %
%%%%%%%%%%%%%%%%%%%%%%
\chapter*{Resumen}
\addcontentsline{toc}{chapter}{Resumen} % si queremos que aparezca en el índice
\markboth{Resumen}{Resumen} % encabezado

rellenar

\textbf{Palabras clave:} Palabra-1, Palabra-2, Palabra-3, Palabra-4.

\chapter*{Abstract}
\addcontentsline{toc}{chapter}{Abstract} % si queremos que aparezca en el índice
\markboth{Abstract}{Abstract} % encabezado

rellenar


\textbf{Keywords:} Palabra-1, Palabra-2, Palabra-3, Palabra-4.

%%%%%%%%%%%%%%%%%%
% INDICE GENERAL %
%%%%%%%%%%%%%%%%%%
\newpage
\thispagestyle{fancy}

\tableofcontents

\cleardoublepage
\listoffigures

\cleardoublepage
\listoftables

% Descomentar las siguientes líneas si tienes pensado añadir diagramas de flujo.
% \cleardoublepage
% \phantomsection
% \addcontentsline{toc}{chapter}{Índice de diagramas de flujo}
% \listofdiagrams

\cleardoublepage
%\phantomsection
\addcontentsline{toc}{chapter}{Lista de Acrónimos}

\include{lista_acronimos}
\printglossary[type=\acronymtype, title={Acrónimos}]

%---------------------%
%	CUERPO DE TEXTO   %
%---------------------%

\afterpage{\null\newpage}

\newpage
\newpage

\pagestyle{cuerpo}
\thispagestyle{cuerpo}

\pagenumbering{arabic}

%%%%%%%%%%%%%%%%%%%%%%%%%%%%%%%%%%%%%%%%%%%%%%%%%%%%%%%%%%%%%%%%%%%%%
%                         ESTRUCTURA CONTENIDO                      %
%%%%%%%%%%%%%%%%%%%%%%%%%%%%%%%%%%%%%%%%%%%%%%%%%%%%%%%%%%%%%%%%%%%%%

%------------------%
%	CAPÍTULOS      %
%------------------%

\chapter{Introducción}
\label{ch:introduccion}

El surgimiento de nuevas herramientas y tecnologías ha hecho posible mejorar las técnicas de ciberseguridad, tanto las técnicas para proteger la información, como las que explotan las brechas de seguridad. Este mismo desarrollo tecnológico hace que se incrementen de forma exponencial los ataques de \textit{malware}. La mejora de las técnicas de ciberseguridad ha provocado que los ciberdelincuentes se esfuercen aún más por conseguir su objetivo.

\vspace{1em}

En 1971, Creeper \cite{creeper}, el primer \textit{malware} de la historia, fue desarrollado por Bob Thomas Morris como un experimento y no causaba daño en los sistemas. Creeper era un gusano que se autorreplicaba y propagaba a través de ARPANET, y mostraba el mensaje <<I'm the creeper, catch me if you can!>>. El primer \textit{malware} con impacto mundial, afectando a un 10\% de los 60000 servidores que había en ARPANET, fue el gusano Morris \cite{morris}, desarrollado en 1988 por Robert Tappan Morris. Morris intentaba obtener la contraseña de los equipos en los que se ejecutaba mediante fuerza bruta, es decir, permutaba los nombres de usuarios conocidos y una lista de las contraseñas más comunes. Creeper y el gusano Morris provocaron la aparición de Reaper, el primer antivirus de la historia y la creación del Equipo de Respuesta ante Emergencias Informáticas (CERT, por sus siglas en inglés) \cite{cert}.

\vspace{1em}

Uno de los casos recientes más sonados fue WannaCry \cite{wannacry} en 2018, \textit{ramsomware} \cite{ransomware} que bloquea el acceso a partes del sistema y pide un rescate. Este \textit{malware} causó un gran impacto a nivel mundial, afectando en España a empresas como Telefónica o Iberdrola \cite{noticia_wannacry}.

\vspace{1em}

Este aumento en la complejidad de las tecnicas usadas tanto para dañar los sistemas como para evitar su detección, ha hecho que los métodos tradicionales basados en firmas \cite{firmas} para detectar patrones únicos en el código queden obsoletos. A día de hoy se combina con la heurística y sigue siendo una de las técnicas más usadas, pero son insuficientes para enfrentar vectores desconocidos.


\vspace{1em}

En los últimos años, uno de los campos más estudiados en  informática y con mayor avance y previsión de futuro es el aprendizaje automático \cite{ml}. Es un campo de estudio dentro de la inteligencia artificial que se centra en aprender patrones a partir de datos, en lugar de seguir reglas programadas explícitamente. El sistema entrena con ejemplos y luego generaliza para hacer predicciones o tomar decisiones. El aprendizaje automático podría ser una solución eficiente y escalable para la detección y clasificación de \textit{malware}.

\vspace{1em}

A lo largo de este proyecto se tratará de evaluar la efectividad de diferentes modelos de aprendizaje automático para identificar este tipo de programas. Para ello tendremos en cuenta precisión, velocidad y capacidad de adaptación ante nuevas variantes de amenazas tratando de identificar las ventajas y limitaciones de cada método.

\chapter{Estado de la técnica}
\label{ch:estado_tecnica}

% TODO: 

\section{Aprendizaje automático}
\label{sec:aprend_auto}

% TODO: 

\subsection{Balanceo de datos}
\label{subsec:balanceo}

% TODO: 

\subsubsection{Sobremuestreo}
\label{subsubsec:oversampling}

% TODO: 

\subsubsection{Submuestreo}
\label{subsubsec:undersampling}

% TODO: 

\subsection{Reducción de la dimensionalidad}
\label{subsec:red_dim}

% TODO: 

\subsubsection{Análisis de componentes principales}
\label{subsubsec:pca}

% TODO: 

\subsubsection{Análisis factorial}
\label{subsubsec:fa}

% TODO: 

\subsubsection{Descomposición en valores singulares}
	\label{subsubsec:svd}

% TODO: 

\subsection{Métricas de evaluación}
\label{subsec:2_metricas}

La elección de métricas de evaluación adecuadas es esencial para valorar de forma precisa el rendimiento de los modelos. No todas las métricas ofrecen la misma información. En esta sección se revisan las métricas más empleadas en la literatura especializada, destacando su utilidad, limitaciones y el tipo de información que aportan para la comparación de modelos.

\subsubsection{Exactitud}
\label{subsubsec:acc}

La exactitud o \textit{Accuracy} se corresponde con el porcentaje de aciertos que se han producido, es decir, los patrones clasificados correctamente respecto al total. Se calcula como la suma de verdaderos positivos (TP) y verdaderos negativos (TN) respecto al número total de patrones de entrada (N) \cite{metrics}.

\begin{equation}
	\label{eq:accuracy}
	\text{CCR} = \frac{TP+TN}{N}
\end{equation}

\subsubsection{Precisión}
\label{subsubsec:prec}

La precisión es una métrica que evalúa la proporción de patrones clasificados como positivas que realmente pertenecen a la clase positiva, es decir, mide como de confiable es el modelo cuando predice un positivo. Es muy relevante cuando el coste de clasificar erróneamente un negativo como positivo es alto.

\begin{equation}
	\text{Precisión} = \frac{TP}{TP + FP}
	\label{eq:precision}
\end{equation}

Donde \(TP\) representa el número de verdaderos positivos, y \(FP\) corresponde al número de falsos positivos.

\subsubsection{Sensibilidad}
\label{subsubsec:sens}

También conocida como exhaustividad o \textit{recall} en inglés, mide la capacidad del modelo para detectar correctamente los positivos de un conjunto de datos. Como se muestra en la ecuación \ref{eq:recall}, se calcula como la proporción entre el número de verdaderos positivos (TP) y la suma de verdaderos positivos y falsos negativos (FN) \cite{metrics}. Un valor alto de sensibilidad indica que se han obtenido pocos falsos negativos.

\begin{equation}
	\label{eq:recall}
	\text{Sensibilidad} = \frac{TP}{TP + FN}
\end{equation}

\subsubsection{Mínima sensibilidad}
\label{subsubsec:ms}

La mínima sensibilidad mide cómo de bien se clasifica la clase peor clasificada. Es útil en clasificación multiclase o con conjuntos de datos desbalanceados, ya que permite identificar si existe alguna clase que el modelo no está clasificando correctamente. Un valor alto indica que el modelo mantiene un buen rendimiento en todas las clases, mientras que un valor bajo revela que, al menos, una de ellas presenta un bajo grado de acierto. Si el modelo se deja una clase sin clasificar, el valor será 0.

\vspace{1em}

Sea \( S_i \) la sensibilidad de la clase \( i \), con \( n \) el número total de clases, la mínima sensibilidad se calcula como se muestra en la ecuación \ref{eq:ms}.

\begin{equation}
	MS = \min_{i \in \{1, 2, \dots, n\}} S_i
	\label{eq:ms}
\end{equation}

Donde la sensibilidad de cada clase \( S_i \) se obtiene mediante la ecuación \ref{eq:recall}

\subsubsection{Valor-F}
\label{subsubsec:f1}

El valor-F o \textit{F1-score} mide el equilibrio entre la precisión y la sensibilidad \cite{metrics}. Se calcula como la media armónica entre ambas, lo que penaliza de forma más severa los valores extremos y proporciona una medida equilibrada del rendimiento del modelo. Es especialmente útil en problemas con clases desbalanceadas, ya que evita que un alto rendimiento en una sola métrica distorsione la evaluación global.

\subsubsection{Matriz de confusión}
\label{subsubsec:matrix}
% TODO: 



\subsection{Técnicas de validación}
\label{subsec:validacion}
% TODO: 



\subsubsection{Validación cruzada}
\label{subsubsec:cv}
% TODO: 



\subsubsection{Validación estratificada}
\label{subsubsec:ev}
% TODO: 



\subsubsection{Problemas de entrenamiento}
\label{subsubsec:overunderfit}
% TODO:  overfitting y underfitting

\subsection{Preprocesamiento de datos}
%\label{subsec:}
% TODO: 



\subsubsection{}
%\label{subsubsec:}
% TODO: 




\subsection{Algoritmos de clasificación}
%\label{subsec:}
% TODO: 



\subsubsection{}
%\label{subsubsec:}
% TODO: 



\subsection{}
%\label{subsec:}
% TODO: 



\subsubsection{}
%\label{subsubsec:}
% TODO: 








\section{Ciberseguridad}
\label{sec:ciberseguridad}

La ciberseguridad es la protección de la infraestructura informática y la información que hay en ella, abarcando \textit{software}, \textit{hardware} y redes. Para garantizar la seguridad, es esencial combinar estrategias de prevención con métodos de protección efectivos. Las estrategias de prevención, como el uso de \textit{firewalls}, \textit{software} antivirus actualizado y educación en ciberseguridad para los usuarios, se centra en identificar y mitigar posibles amenazas antes de que ocurran. Por otro lado, la protección se enfoca en responder a los incidentes y minimizar sus efectos, mediante herramientas como los sistemas de detección de intrusiones. Con esto, podemos llegar a la conclusión de que el objetivo de la seguridad es minimizar los riesgos de recibir un ataque y reducir el impacto en caso recibirlo \cite{ciberseguridad_def}. En esta sección nos centraremos en la ciberseguridad \textit{software}, concretamente en los aspectos relacionados con la detección y clasificación de malware.

\subsection{Conceptos generales}
\label{subsec:ciberseguridad_general}
% TODO: Breve introducción a la ciberseguridad, importancia en la sociedad digital, principales objetivos (confidencialidad, integridad, disponibilidad) y amenazas comunes.

\subsection{\textit{Malware}}
\label{subsec:malware}
% TODO: Definición de malware, evolución histórica, relevancia actual.

\subsubsection{Tipos de \textit{Malware}}
\label{subsubsec:tipos_malware}
El \textit{software} malicioso o \textit{malware} es cualquier tipo de \textit{software} que se introduce de manera encubierta con el objetivo de comprometer la confidencialidad, integridad o disponibilidad de la información o el sistema \cite{def_malware}. El \textit{malware} se ha convertido en una de las amenazas externas más relevantes debido al daño que puede llegar a causar en una organización. Podemos clasificar el \textit{malware} en diferentes categorías \cite{categoriamw} según su propósito:

\begin{itemize}
	\item Virus. Tienen como objetivo infectar archivos y sistemas informáticos. Se propagan cuando los usuarios comparten archivos o ejecutan programas infectados.
	\item Gusanos. Se propagan a través de las redes sin que tenga que intervenir el usuario.
	\item Troyanos. Se presentan como un \textit{software} legítimo. De esta forma intentan engañar al usuario para que lo descargue, instale y ejecute.
	\item \textit{Adware}. Muestra anuncios de forma intrusiva. Puede ser incrustada en una página web mediante gráficos, carteles, ventanas flotantes, o durante la instalación de algún programa al usuario, con el fin de generar lucro a sus autores.
	\item \textit{Spyware}. Trata de conseguir información de un equipo sin conocimiento ni consentimiento del usuario. Después transmite esta información a una entidad externa.
	\item \textit{Ransomware}. Conocido como secuestro de datos en español. Está diseñado para restringir el acceso a archivos o partes de un sistema y pedir un rescate para quitar la restricción.
	\item \textit{Rootkit}. Es un conjunto de \textit{software} que permite al atacante un acceso de privilegio a un ordenador, manteniendo presencia inicialmente oculta al control de los administradores.
	\item \textit{Keylogger}. Se encarga de registrar las pulsaciones que se realizan en el teclado, para memorizarlas en un fichero o enviarlas a través de Internet.
	\item \textit{Exploit}. Aprovecha un error o una vulnerabilidad de una aplicación o sistema para provocar un comportamiento involuntario.
	\item \textit{Backdoor}. Puerta trasera en español. Este tipo de \textit{software} permite un acceso no autorizado al sistema, evitando pasar por los métodos de autenticación.
\end{itemize}

\subsection{Técnicas de detección de \textit{Malware}}
\label{subsec:deteccion_malware}
% TODO: 

Ningún método de detección es infalible y los principales antivirus comerciales pueden combinar distintas técnicas en función de las necesidades. La detección basada en firmas siguen siendo el método más usado en términos absolutos porque son rápidas, eficientes y fáciles de implementar. Este método consiste en comparar archivos con una base de datos de patrones conocidos. Otros mecanismos son: la detección heurística, por comportamiento, \textit{sandbox} e inteligencia artificial \cite{antivirus}.

\vspace{1em}

Existen varias limitaciones de los métodos tradicionales frente a nuevas amenazas. Por ejemplo, para evadir la detección basada en firmas se generaba una cadena de bits única cada vez que se codificaba. Esto se denomina polimorfismo. Gracias a la heurística no era necesaria una coincidencia exacta con las firmas almacenadas, pero debido a la gran cantidad de variaciones que surgen a diario, su efectividad y la de otros mecanismos se ve comprometida \cite{limitaciones}. A continuación se estudiarán algunas de las técnicas más usadas.

\subsubsection{Detección basada en firmas}
\label{subsubsec:firmas}
% TODO: 

\subsubsection{Detección heurística y análisis estático}
\label{subsubsec:heuristica}
% TODO: 

\subsubsection{Detección basada en comportamiento (análisis dinámico)}
\label{subsubsec:comportamiento}
% TODO: 

\subsubsection{Métodos híbridos}
\label{subsubsec:hibridos}
% TODO: 

\subsubsection{Detección mediante aprendizaje automático}
\label{subsubsec:ml}
% TODO: Algoritmos de clasificación, extracción de características y métricas más usadas.


\subsection{Retos y tendencias}
\label{subsec:retos_tendencias}
% TODO: Limitaciones de las técnicas clásicas, evasión mediante ofuscación, necesidad de datasets representativos y uso creciente de IA y aprendizaje profundo en ciberseguridad.



\chapter{Formulación del problema y objetivos}
\label{ch:problema_y_objetivos}

En este capítulo se describirá el contexto en de la investigación y los retos que plantea, así como los problemas que surgen de esta situación. Después, definiremos los objetivos específicos que orientan el estudio, estableciendo las metas a alcanzar y el alcance del proyecto. Al mismo tiempo, se abordarán con más detalle las situaciones que han provocado estos problemas y se expondrán distintos objetivos con la intención de mitigarlos.

\section{Contexto y motivación}
\label{sec:contexto}

La detección de \textit{malware} es uno de los grandes desafíos de la ciberseguridad por su rápida y constante evolución. Cada día aparecen nuevas amenazas capaces de evadir las técnicas tradicionales explicadas en la sección \ref{subsec:deteccion_malware} y no es posible depender de reglas predefinidas. Esto ha puesto de manifiesto la necesidad de evolucionar al mismo ritmo las técnicas de detección y ha hecho que las técnicas tradicionales resulten insuficientes por si solas. Con el rápido crecimiento que está teniendo el aprendizaje automático, podría ser un gran aliado si se usa de la forma adecuada, ya que es una herramienta muy potente capaz de detectar amenazas aun no conocidas.

\vspace{1em}

El uso de algoritmos de aprendizaje automático permite automatizar el análisis de grandes volúmenes de datos y adaptarse a la evolución de las amenazas. Además, existe una gran variedad de modelos, lo que posibilita evaluar diferentes estrategias e identificar los métodos más precisos, eficientes y escalables. Por otro lado, es posible volver a entrenar usando nuevos datos y adaptarse rápidamente a los cambios que surjan \cite{campus}.

\section{Definición del problema}
\label{sec:definicion}

A continuación, se definen los principales problemas y preguntas que nos hemos encontrado:

\begin{itemize}
	\item Detectar nuevas amenazas sin necesidad de conocerlas previamente.
	\item Dificultad de seleccionar el algoritmo más adecuado.
	\item Limitaciones de recursos computacionales.
	\item ¿Cómo afectan las características de cada conjunto al rendimiento de los modelos?
	\item ¿Qué variables son más influyentes en la clasificación?
\end{itemize}

\section{Objetivos}
\label{sec:objetivos}

A partir de los problemas presentados en la sección \ref{sec:definicion}, podemos establecer una serie de objetivos que definirán el desarrollo del estudio del que trata este proyecto. Los objetivos se pueden dividir en dos tipos: general y específicos. El primero es la columna vertebral del proyecto, el tema central sobre el que gira el estudio que se realizará. Los objetivos específicos dividen el objetivo principal en otros más concretos y deseables. A continuación, se expondrán ambos tipos.

\subsection{Objetivo general}
\label{subsec:obj_gen}

El objetivo principal para este estudio es comparar distintos algoritmos de aprendizaje automático haciendo uso de conjuntos de datos de \textit{malware}. Este objetivo se centra en los dos primeros problemas comentados en la sección \ref{sec:definicion}, tratando de conseguir detectar nuevas amenazas y tener una idea general de qué algoritmos se adaptan mejor a este propósito. Para ello, evaluaremos su eficiencia, precisión y viabilidad computacional.

\subsection{Objetivos específicos}
\label{subsec:obj_esp}

A partir del objetivo principal y del resto de problemas planteados, podemos concretar una serie propósitos más concretos:

\begin{itemize}
	\item Estudio teórico de distintos algoritmos en la detección de \textit{malware}.
	\item Obtención y análisis de bases de datos públicas de \textit{malware}.
	\item Implementación de metodologías de detección de \textit{malware} y su adaptación para uso en las bases de datos anteriores.
	\item Evaluar el rendimiento, eficiencia, precisión y viabilidad computacional de estos algoritmos.
	\item Identificar y analizar los métodos que se adaptan mejor al problema, destacando las ventajas e inconvenientes de cada uno de los algoritmos.
	\item Identificación de las variables e información más influyentes en la detección de \textit{malware}, particularmente para cada base de datos.
\end{itemize}


\chapter{Metodología de trabajo}
\label{ch:metodologia}

Este capítulo describe la metodología seguida para el desarrollo del proyecto. Se explican los enfoques, técnicas y herramientas utilizadas para alcanzar los objetivos. Además, se expone el tipo de estudio realizado y la selección del conjunto de datos y los modelos. El propósito es ofrecer una guía clara del proceso seguido.

\section{Enfoque metodológico}
\label{sec:enfoque}

Existen varios enfoques aplicables a este tipo de proyecto, pero dado el carácter planteado inicialmente en los objetivos, se centrará en un estudio comparativo y experimental de distintos algoritmos de aprendizaje automático en la detección de \textit{malware}. Se combina la experimentación práctica sobre conjuntos de datos reales con análisis estadísticos sobre los resultado obtenidos en las distintas pruebas realizadas. Cada modelo se somete a pruebas controladas en escenarios de clasificación binaria y multiclase, utilizando conjuntos de datos públicos y representativos.

\vspace{1em}

Esta metodología permite identificar los algoritmos con mejor equilibrio y adaptación a nuevos patrones. También se podrán detectar posibles limitaciones y áreas de mejora para futuras investigaciones. Este enfoque proporciona un marco sistemático para el análisis comparativo de modelos, facilitando la interpretación de resultados y la toma de decisiones fundamentadas sobre el rendimiento de cada algoritmo.

\section{Preparación del entorno}
\label{sec:prep_entorno}

En esta sección se describe el entorno de trabajo utilizado por el alumno para la implementación de los modelos y la realización de las pruebas. El entorno se debe preparar de forma correcta, ya que puede afectar a la ejecución de los algoritmos y a la reproducibilidad de los experimentos. A continuación, se explican elementos del entorno como el lenguaje de programación, las bibliotecas y las características del equipo.

\subsection{Herramientas y bibliotecas}
\label{subsec:herramientas}

El desarrollo y la experimentación de este proyecto se han llevado a cabo empleando un conjunto de herramientas y bibliotecas muy utilizadas en la ciencia de datos. \textit{Python} ha sido el lenguaje de programación de este trabajo, ya que ofrece una fácil implementación de modelos, manipulación de datos y visualización de resultados. Su popularidad en se debe a su sintaxis sencilla, escalabilidad y amplia variedad de herramientas y bibliotecas \cite{python_ml}.

\vspace{1em}

En este proyecto se ha utilizado la versión 3.12 de \textit{Python}, elegida principalmente por su compatibilidad con las bibliotecas empleadas, en particular, con \textit{GridSearchCV}, que aprovechan la paralelización de procesos para mejorar el rendimiento. El problema encontrado es que los hilos no se cierran correctamente, es un comportamiento típico asociado a lo que en programación concurrente se denomina \textit{thread leakage} o hilos huérfanos. provoca que la memoria \textit{RAM} y la \textit{CPU} sigan siendo consumidas incluso después de que la ejecución haya terminado. En teoría, este problema no afecta al rendimiento de los clasificadores, pero puede afectar al tiempo de ejecución.

\newpage
\subsubsection{\texttt{Scikit-learn}}
\label{subsubsec:sklearn}

\texttt{Scikit-learn} es un paquete de código abierto en \texttt{Python} que ofrece una gran variedad de métodos de aprendizaje automático rápidos y eficientes, gracias a que usan bibliotecas compiladas en lenguajes como \texttt{C++}, \texttt{C} o \texttt{Fortran}. Tiene detrás una comunidad activa que mantiene la documentación, corrige errores y asegura la calidad. Aunque no incluye todos los algoritmos usados en este proyecto, es una herramienta muy recomendable si necesitamos: transformación de datos, aprendizaje supervisado o evaluación de modelos \cite{hao2019scikit}.

\subsubsection{\texttt{DLOrdinal}}
\label{subsubsec:dlordinal}

La biblioteca \texttt{dlordinal} incluye muchas de las metodologías más recientes de clasificación ordinal usando técnicas avanzadas de aprendizaje profundo. El enfoque ordinal de esta herramienta tiene el objetivo de aprovechar la información de orden presente en la variable objetivo usando funciones de pérdida, diversas capas de salida y otras estrategias \cite{dlordinal}. El módulo de \texttt{dlordinal} que nos ha resultado de utilidad para este proyecto ha sido la el conjunto de métricas que incluye para evaluar los modelos utilizados, ya que cuenta con una de las métricas que finalmente hemos usado: mínima sensibilidad.

\subsubsection{\texttt{Matplotlib}}
\label{subsubsec:matplotlib}

Para una mejor visualización de los datos obtenidos en los modelos utilizados, se ha usado la biblioteca \texttt{Matplotlib}, ya que incluye una gran cantidad de recursos para la representación gráfica de la información \cite{matplotlib}. Se ha usado, en combinación con \texttt{Seaborn}, descrita en la sección \ref{subsubsec:seaborn}.

\newpage
\subsubsection{\texttt{NumPy}}
\label{subsubsec:numpy}

La biblioteca \texttt{NumPy} tiene como objetivo principal dar soporte a la creación de vectores y matrices de grandes dimensiones, junto con una colección de funciones matemáticas con las que operar \cite{numpy}. Ha sido de gran utilidad en el desarrollo del proyecto, ya que el conjunto de datos con el que se ha trabajado es de un tamaño considerable, aunque no ha sido necesario hacer uso de las funciones que proporciona porque la mayor parte de los cálculos necesarios se hacen de manera interna en los modelos utilizados.

\subsubsection{\texttt{Pandas}}
\label{subsubsec:pandas}

\texttt{Pandas} es una herramienta muy potente para el manejo, análisis y manipulación de datos. Incluye una amplia variedad de herramientas para: leer y escribir datos, reestructuración y segmentación, inserción y eliminación de columnas, mezcla y unión de datos y muchas funcionalidades más \cite{pandas}. Varias de ellas se han utilizado durante el desarrollo y la preparación del conjunto de datos.

\subsubsection{\texttt{LightGBM}}
\label{subsubsec:lightgbm}

La biblioteca \textit{Light Gradient-Boosting Machine} por su nombre en inglés, es una infraestructura de aprendizaje automático basada en modelos de árboles de decisión \cite{lgbm}. Se puede usar en diferentes tareas, pero la importante para el análisis realizado el la de clasificación. Los principales algoritmos soportados son: \textit{Gradient Boosting Decision Trees (GBDT)}, el cual utiliza \texttt{LGBMClassifier}, clasificador usado durante la experimentación, \textit{Dropouts meet Multiple Additive Regression Trees (Dart)} y \textit{Gradient-based One-Side Sampling (Goss)} \cite{lgbm_alg}.

\newpage
\subsubsection{\texttt{Seaborn}}
\label{subsubsec:seaborn}

Basada en \texttt{Matplotlib}, \texttt{Seaborn} proporciona una interfaz de alto nivel para generar gráficos estadísticos \cite{seaborn}. Es posible usar ambas bibliotecas de forma combinada para una mayor capacidad de visualización. Mientras que \texttt{Matplotlib} ofrece un control detallado sobre cada elemento de la figura, \texttt{Seaborn} simplifica la creación de visualizaciones complejas, incorporando estilos predefinidos y funciones específicas para el análisis de datos.

\subsection{\textit{Hardware}}
\label{subsec:hw_usado}

El entrenamiento y evaluación de los modelos se ha realizado en el equipo del estudiante con las siguientes características: procesador \textit{Intel Core i7-4712MQ}, tarjeta gráfica \textit{NVIDIA GeForce 920M}, 16 GB de \textit{RAM} DDR3 y almacenamiento compuesto por un \textit{SSD Crucial MX500} de 250 GB y un \textit{HDD} de 1 TB. Este hardware permite la paralelización de los algoritmos en múltiples núcleos del procesador, lo que reduce significativamente los tiempos de entrenamiento, pero se encuentra muy limitado respecto al conjunto utilizado para clasificación multiclase y modelos más costosos como puede ser \textit{SVM}.

\subsection{Conjunto de datos}
\label{subsec:select_dataset}

En lo que a \textit{malware} se refiere, \textit{BODMAS} \cite{bodmas} es uno de los conjuntos de datos más completos en la actualidad, con la ventaja para este proyecto de ya estar procesado y tener una amplia bibliografía. Otra opción interesante puede ser \textit{VirusShare} \cite{virusshare}, ya que cuenta con más de 99 millones de muestras de \textit{malware} actualizadas pero tiene varios inconvenientes para este proyecto. El primero, es que no incluye muestras de \textit{software} no malicioso y el segundo, que necesita un procesamiento previo para extraer las características. Todo esto conlleva un aumento de tiempo considerable para la realización del proyecto. Otra de las opciones estudiadas ha sido \textit{theZoo} \cite{thezoo}. En cuanto a este repositorio hemos podido observar que tiene los mismos inconvenientes que \textit{VirusShare} y no tiene sus ventajas. Por último tenemos \textit{Microsoft Malware Classification} \cite{malware-classification}. En este caso tenemos un conjunto de datos muy amplio con casi medio \textit{terabyte} de información, pero además de los inconvenientes ya comentados en los anteriores conjuntos, solo incluye \textit{malware} que afecta a equipos \textit{Windows}, lo que limitaría considerablemente el alcance del estudio.

\vspace{1em}

Teniendo en cuenta todo lo comentado hasta ahora sobre los distintos conjuntos de datos considerados, hemos decidido usar \textit{BODMAS}, ya que es el que mejor se adapta a las necesidades del estudio.

\subsection{Modelos}
\label{subsec:select_model}

Existe una gran variedad de modelos de aprendizaje automático implementados en las diferentes bibliotecas de \textit{Python}. Aunque habría sido interesante hacer una comparación con el mayor número posible de ellos, por limitaciones de equipo y tiempo se ha hecho una pequeña selección siguiendo los siguientes criterios:

\begin{itemize}
	\item Diversidad en los enfoques de aprendizaje: Modelos rápidos como los árboles, otros más robustos, modelos lineales como referencia, métodos basados en distancia, redes neuronales y máquinas de vectores soporte por su capacidad de trabajar la optimización de márgenes.
	\item Equilibrio entre interpretabilidad y complejidad, usando modelos simples y algunos complejos pero que suelen ofrecer mejores resultados.
	\item Se han incluido tanto modelos que toleran bien el desbalanceo como otros más sensibles.
	\item Escalabilidad y coste computacional, usando desde algunos modelos ligeros a otros más costosos. Esto permite evaluar la viabilidad práctica de cada modelo en escenarios reales
\end{itemize}

\newpage
En este proyecto se han empleado diversos algoritmos de aprendizaje automático, seleccionados en función de los criterios mencionados en el capitulo \ref{ch:metodologia} y con el objetivo de representar diferentes enfoques. Se han utilizado los siguientes modelos implementados en \textit{scikit-learn} y \textit{LightGBM}:

\begin{spacing}{1}
	\begin{itemize}
		\item \textit{DecisionTreeClassifier}
		\item \textit{RandomForestClassifier}
		\item \textit{KNeighborsClassifier}
		\item \textit{RidgeClassifier}
		\item \textit{MLPClassifier}
		\item \textit{SVC}
		\item \textit{LGBMClassifier} (de \textit{LightGBM})
	\end{itemize}
\end{spacing}
Todos los modelos se han ajustado y evaluado utilizando \textit{GridSearchCV}, lo que permite explorar sistemáticamente distintas combinaciones de hiperparámetros y asegurar comparaciones consistentes entre los distintos métodos de clasificación. La descripción teórica de estos modelos se presenta en el capítulo \ref{ch:estado_tecnica}.

\subsection{Criterios de evaluación}
\label{subsec:evaluacion}

Para evaluar la efectividad de los modelos implementados, de las métricas comentadas en la sección \ref{subsec:2_metricas}, se han utilizado las siguientes métricas:

\begin{itemize}
	\item \textbf{\textit{Accuracy}}: proporción de predicciones correctas sobre el total de patrones.
	\item \textbf{Mínima Sensibilidad}: sensibilidad de la clase peor clasificada.
\end{itemize}

Estas métricas han sido seleccionadas porque permiten evaluar de forma equilibrada tanto el rendimiento global del modelo como su capacidad para identificar correctamente las clases menos representadas, evitando que los resultados se vean sesgados por un posible desbalanceo en los datos. Esta combinación ofrece una visión complementaria que facilita la comparación objetiva entre diferentes enfoques.
\chapter{Desarrollo y experimentación}
\label{ch:desarrollo}

En esta fase se lleva a cabo la implementación práctica del estudio, haciendo uso de los modelos de aprendizaje automático implementados principalmente en la biblioteca \textit{Scikit-Learn} de \textit{python}. Para ello se realiza un procesamiento de los datos, necesario para obtener un conjunto reducido y otro apto para la clasificación multiclase. Además, se configuran los entornos necesarios para su entrenamiento y evaluación, se establecen las métricas de rendimiento, los procedimientos de prueba y los escenarios de experimentación que permitirán obtener resultados consistentes y comparables. El objetivo es verificar, mediante pruebas controladas, la efectividad de cada método en la detección de \textit{malware}.

\vspace{1em}

La parte experimental se aborda desde dos perspectivas complementarias. En primer lugar, se evalúa la capacidad de los modelos para la detección de \textit{malware} mediante pruebas de clasificación binaria, determinando si un patrón corresponde a software malicioso o legítimo. En segundo lugar, se analiza la viabilidad de realizar una clasificación multinivel sobre esos mismos patrones, identificando el tipo específico de \textit{malware} al que pertenecen, lo que permite un análisis más detallado y aplicable a entornos de ciberseguridad avanzada.

\section{Modelos utilizados}
\label{sec:modelos_utilizados}

En este proyecto se han empleado diversos algoritmos de aprendizaje automático, seleccionados en función de los criterios mencionados en el capitulo \ref{ch:metodologia} y con el objetivo de representar diferentes enfoques.

Se han utilizado los siguientes modelos implementados en \textit{scikit-learn} y \textit{LightGBM}:

\begin{itemize}
	\item \textit{DecisionTreeClassifier}
	\item \textit{RandomForestClassifier}
	\item \textit{KNeighborsClassifier}
	\item \textit{RidgeClassifier}
	\item \textit{MLPClassifier}
	\item \textit{SVC}
	\item \textit{LGBMClassifier} (de \textit{LightGBM})
\end{itemize}

Todos los modelos se han ajustado y evaluado utilizando \textit{GridSearchCV}, lo que permite explorar sistemáticamente distintas combinaciones de hiperparámetros y asegurar comparaciones consistentes entre los distintos métodos de clasificación. La descripción teórica de estos modelos se presenta en el capítulo \ref{ch:estado_tecnica}.


\section{Procesamiento del conjunto de datos}
\label{sec:proc_dataset}

Dadas las limitaciones \textit{hardware} y la cantidad de datos, aproximadamente 135000 patrones y 2400 atributos por cada patrón, es necesario hacer un procesamiento previo del conjunto de datos. Para ello hemos tenido en cuenta varios enfoques. Por un lado, \textit{BODMAS} nos permite hacer una distinción entre clasificación binaria y clasificación multiclase, pero para ello es necesario reordenar los datos, ya que se encuentran distribuidos en varios archivos. Por otro lado, es necesario reducir la cantidad de datos. A continuación veremos los distintos enfoques.

\subsection{Clasificación multiclase}
\label{subsec:multiclass}

El conjunto de datos seleccionado se divide en varios archivos:

\begin{itemize}
	\item \textit{bodmas.npz}: incluye la matriz de patrones de entrada en formato de matriz de \textit{python} y la matriz de salidas deseadas.
	\item \textit{bodmas\_metadata.csv}: la información relevante para nuestro problema es la columna \textit{sha} que contiene la función \textit{hash} de todo el conjunto de datos.
	\item \textit{bodmas\_malware\_category.csv}: contiene la función \textit{hash} del \textit{malware} y la categoría a la que pertenece.
\end{itemize}

Dado que las distintas categorías se encuentran en formato texto, es necesario codificarlas para poder trabajar con ellas. La codificación elegida ha sido la representada en la tabla \ref{tabla:codificacion_malware}.

\vspace{1em}

\begin{table}[th]
	\centering
	\begin{tabular}{ |m{4cm}|c|c| }
		\hline
		\rowcolor{LightCyan}
		Categoría                   & Codificación & Nº de patrones \\
		\hline
		\textit{benign}             & 0            & 77142 \\
		\textit{trojan}             & 1            & 29972 \\
		\textit{worm}               & 2            & 16697 \\
		\textit{backdoor}           & 3            & 7331  \\
		\textit{downloader}         & 4            & 1031  \\
		\textit{informationstealer} & 5            & 448   \\
		\textit{dropper}            & 6            & 715   \\
		\textit{ransomware}         & 7            & 821   \\
		\textit{rootkit}            & 8            & 3     \\
		\textit{cryptominer}        & 9            & 20    \\
		\textit{pua}                & 10           & 29    \\
		\textit{exploit}            & 11           & 12    \\
		\textit{virus}              & 12           & 192   \\
		\textit{p2p-worm}           & 13           & 16    \\
		\textit{trojan-gamethief}   & 14           & 6     \\
		\hline
	\end{tabular}
	\caption{Codificación de las clases \textit{malware}.}
	\label{tabla:codificacion_malware}
\end{table}

Para obtener una nueva matriz de salidas deseadas que incluya los tipos de \textit{malware}, una vez cargados los datos en sus correspondiente variables de \textit{python}, usamos la función \textit{merge} \cite{merge} perteneciente a la clase \textit{pandas.DataFrame} para incluir en \textit{metadata} los datos de \textit{mw\_category['category']} en las entradas donde coincide la columna \textit{sha}.

\vspace{1em}

Antes de codificar necesitamos darle una etiqueta a los datos vacíos, los cuales significan que esa muestra es benigna. Para ello usamos la función \textit{pandas.DataFrame.fillna} \cite{fillna}, que nos permite completar datos vacíos de distintas formas. Para nuestro caso usamos la etiqueta \textit{benign}. También eliminamos las columnas que no vamos a necesitar, dejando solo la categoría a la que pertenece cada muestra.

\vspace{1em}

Ahora podemos codificar los datos usando la función \textit{pandas.DataFrame.map} \cite{map}. Este método aplica una función que acepta y devuelve un valor escalar a cada elemento del DataFrame, lo que permite asignar un valor numérico a cada clase.

\vspace{1em}

El código utilizado para esta tarea se encuentra en el Anexo \ref{sec:codificacion}.

\subsection{Reducción del conjunto de datos}
\label{subsec:red_dataset}

Reducir el número de datos con el que vamos a trabajar tiene el objetivo de principal de disminuir el tiempo que los algoritmos van a necesitar para procesar la información sin perjudicar la integridad de los datos, ya que los resultados del estudio podrían verse afectados y llevar a unas conclusiones erróneas. Esta tarea se puede enfrentar desde dos planteamientos distintos: condensar el número de patrones o el número de características. Ambos planteamientos se han estudiado de forma teórica en esta memoria en las secciones \ref{subsec:balanceo} y \ref{subsec:red_dim} respectivamente. Las técnicas elegidas son \textit{undersampling} por simplicidad y \textit{PCA} porque según el estudio \textit{A Low Complexity ML-Based Methods for Malware Classification} \cite{red_dim_pca} se obtienen unos resultados algo más precisos que con otros métodos.

\vspace{1em}

El código utilizado se encuentra en el anexo \ref{sec:red_dim}. A continuación se explicarán los pasos seguidos.

\subsubsection{Número de patrones}
\label{subsubsec:num_patrones}

Como ya hemos estudiado en la sección \ref{subsubsec:undersampling}, el submuestreo o \textit{undersampling} en inglés, es una técnica para abordar el desbalance de clases en un conjunto de datos, especialmente cuando una de las clases tiene muchos más patrones que la otra. En nuestro caso, el desbalance no es demasiado grande ya que \textit{BODMAS} contiene 57293 muestras \textit{malware} y 77142 muestras benignas.

\vspace{1em}

El método \textit{RandomUnderSampler} \cite{randundersampler} de la biblioteca \textit{Imbalanced learn} nos permite varias formas de actuar, siendo la que nos interesa para este estudio la que nos permite elegir manualmente el número de patrones de cada clase. Hemos elegido una cantidad de 15000 patrones en por clase.

\subsubsection{Número de características}
\label{subsubsec:num_caract}

Este método, también conocido como reducción de la dimensionalidad, consiste en reducir el número de variables de las que consta el problema. Para aplicar el método matemático-estadístico de análisis de componentes principales, \textit{PCA} por sus siglas en inglés, usamos la clase \textit{PCA} \cite{sklearn_pca} perteneciente a \textit{sklearn.decomposition}. Esta clase nos permite entrenar el modelo y transformar el conjunto de datos tanto para el conjunto de entrenamiento como para el de test. Para ello será necesario separar previamente los datos, ya que \textit{BODMAS} no cuenta con esta división.

\subsubsection{Elección final del nuevo conjunto de datos}
\label{subsubsec:eleccion_dataset}

Para poder decidir como será el conjunto de entrenamiento final se han hecho distintos conjuntos de datos sobre los que se probarán algunos algoritmos. Los conjuntos son los siguientes:

\begin{itemize}
	\item Clasificación binaria con \textit{PCA}.
	\item Clasificación binaria con \textit{PCA} y \textit{Undersampling} con 15000 patrones por clase.
	\item Clasificación multiclase con \textit{PCA}.
\end{itemize}

Los resultados obtenidos se reflejan en las tablas \ref{tabla:binary_pca}, \ref{tabla:binary_under} y \ref{tabla:multi_pca} respectivamente.

\begin{table}[th]
	\centering
	\begin{tabular}{ |c|c|c|c|c|c|c|c| }
		\hline
		\rowcolor{LightCyan}
		Clasificador & Tiempo (s) & \multicolumn{3}{c|}{Entrenamiento} & \multicolumn{3}{c|}{Test} \\
		\hline
		\rowcolor{LightCyan}
		&            & Acc & MS & F1 & Acc & MS & F1 \\
		\hline
		\textit{Decission tree} & 0.885 & 1.000 & 1.000 & 1.000 & 0.972 & 0.971 & 1.000 \\
		\textit{Random forest}  & 25.91 & 1.000 & 1.000 & 1.000 & 0.984 & 0.976 & 1.000 \\
		\textit{K-NN}           & 0.095 & 0.973 & 0.970 & 1.000 & 0.963 & 0.963 & 1.000 \\
		\hline
	\end{tabular}
	\caption{Clasificación binaria con \textit{PCA}.}
	\label{tabla:binary_pca}
\end{table}

\begin{table}[th]
	\centering
	\begin{tabular}{ |c|c|c|c|c|c|c|c| }
		\hline
		\rowcolor{LightCyan}
		Clasificador & Tiempo (s) & \multicolumn{3}{c|}{Entrenamiento} & \multicolumn{3}{c|}{Test} \\
		\hline
		\rowcolor{LightCyan}
		&            & Acc & MS & F1 & Acc & MS & F1 \\
		\hline
		\textit{Decission tree} & 0.184 & 1.000 & 1.000 & 1.000 & 0.945 & 0.936 & 1.000 \\
		\textit{Random forest}  & 4.926 & 1.000 & 1.000 & 1.000 & 0.963 & 0.957 & 1.000 \\
		\textit{K-NN}           & 0.016 & 0.954 & 0.948 & 1.000 & 0.938 & 0.931 & 1.000 \\
		\hline
	\end{tabular}
	\caption{Clasificación binaria con \textit{PCA} y \textit{Undersampling}.}
	\label{tabla:binary_under}
\end{table}

\begin{table}[th]
	\centering
	\begin{tabular}{ |c|c|c|c|c|c|c|c| }
		\hline
		\rowcolor{LightCyan}
		Clasificador & Tiempo (s) & \multicolumn{3}{c|}{Entrenamiento} & \multicolumn{3}{c|}{Test} \\
		\hline
		\rowcolor{LightCyan}
		& & Acc & MS & F1 & Acc & MS & F1 \\
		\hline
		\textit{Decission tree} & 1.059 & 0.999 & 0.895 & 0.999 & 0.939 & 0.000 & 0.976 \\
		\textit{Random forest}  & 30.04 & 0.999 & 0.895 & 0.999 & 0.955 & 0.000 & 0.981 \\
		\textit{K-NN}           & 0.088 & 0.951 & 0.000 & 0.981 & 0.936 & 0.000 & 0.975 \\
		\hline
	\end{tabular}
	\caption{Clasificación multiclase con \textit{PCA}}
	\label{tabla:multi_pca}
\end{table}

En cuanto a la clasificación binaria, hemos decidido usar el conjunto de datos en el que se ha aplicado tanto \textit{PCA} como \textit{undersampling}, ya que, aunque los resultados son similares en ambos conjuntos, el tiempo es considerablemente más bajo y dadas las limitaciones del equipo disponible puede ser beneficioso a la hora de probar algoritmos más complejos.

\vspace{1em}

Para la clasificación multiclase hay varios métodos que podemos usar para reducir el tamaño del conjunto de datos, como el \textit{clustering} o variantes del método de \textit{undersampling} ya utilizado en clasificación binaria. A pesar de ello, estos métodos tienen una mayor complejidad de aplicación y la reducción de las dimensiones no es el objeto de este estudio. Por otro lado, esta decisión puede suponer algunos problemas al usar técnicas como \textit{GridSearchCV} o la validación cruzada, ya que incrementan considerablemente el tiempo de entrenamiento.

\vspace{1em}

También en referencia a la clasificación multiclase, podemos ver en la tabla \ref{tabla:multi_pca} que la métrica de mínima sensibilidad es 0 para todos los casos de test. Como ya se ha explicado en esta memoria, mide cómo de bien se clasifica la clase peor clasificada y un valor de 0 indica que alguna de las clases no se ha clasificado bien. Como podemos ver en la matriz de confusión representada en la imagen \ref{fig:confusion}, algunas de las clases con menos patrones tienen dificultades para obtener una buena clasificación debido a la falta de información en el entrenamiento. Algunos clasificadores tienen la opción de asignar un peso a los patrones de cada clase inversamente proporcional al número de patrones de la clase, de manera que todas las clases tengan el mismo peso en el entrenamiento, pero no se consiguen mejores resultados.

\begin{figure}[H]
	\centering
	% include first image
	\includegraphics[width=1.2\linewidth]{Imagenes/confusion_multiclase}
	\caption[Matriz de confusión para la clasificación multicase]{Matriz de confusión para la clasificación multicase}
	\label{fig:confusion}
\end{figure}

Según el estudio \textit{Malware Behavior Analysis: Learning and Understanding Current Malware Threats} \cite{mba}, algunos de los tipos de \textit{malware} que tenemos con menos patrones, se pueden agrupar en algunas de las clases más representadas de nuestro conjunto de datos. En este estudio se comenta que \textit{p2p-worm} añade un comportamiento específico al comportamiento de un gusano, generando problemas de red y de pérdida de datos. Algo similar pasa con \textit{Gamethief trojan}. De esta forma podemos agrupar estos patrones a sus respectivas clases similares sin perder efectividad a la hora de clasificar y además eliminar así dos de las clases que nos pueden dar problemas por falta de información.

\vspace{1em}

Por otro lado, se han planteado dos formas de solucionar este problema, aunque ambas presentan inconvenientes:

\begin{itemize}
	\item Eliminar las clases menos representadas. Tiene el riesgo de no reconocer un nuevo patrón si es de un tipo distinto de \textit{malware}.
	\item Agruparlas en una nueva clase que represente varios tipos de \textit{malware}. En este caso estamos suponiendo que los patrones agrupados tienen unas características similares.
\end{itemize}

\vspace{1em}

Finalmente hemos decidido agrupar las clases con menos de 30 patrones en una nueva categoría \textit{otros}. Por número de patrones sería recomendable agrupar también la clase \textit{virus}, pero podría tener demasiado peso en la categoría otros y hemos considerado que es lo suficientemente relevante como para estudiarla por separado. En la tabla \ref{tabla:multi_new} podemos ver que, aunque mejoramos la mínima sensibilidad, no se producen unas mejoras significativas en la precisión de clasificación pero dada la alta precisión presentada por los modelos y la mejora en la mínima sensibilidad puede considerarse una buena actualización. Podemos ver la nueva codificación en la tabla \ref{tabla:nueva_codificacion_malware}

\begin{table}[th]
	\centering
	\begin{tabular}{ |m{4cm}|c|c| }
		\hline
		\rowcolor{LightCyan}
		Categoría                   & Codificación & Nº de patrones \\
		\hline
		\textit{benign}             & 0            & 77142 \\
		\textit{trojan}             & 1            & 29978 \\
		\textit{worm}               & 2            & 16713 \\
		\textit{backdoor}           & 3            & 7331  \\
		\textit{downloader}         & 4            & 1031  \\
		\textit{informationstealer} & 5            & 448   \\
		\textit{dropper}            & 6            & 715   \\
		\textit{ransomware}         & 7            & 821   \\
		\textit{virus}              & 8            & 192   \\
		\textit{otros}              & 9            & 64    \\
		\hline
	\end{tabular}
	\caption{Nueva codificación de las clases \textit{malware}.}
	\label{tabla:nueva_codificacion_malware}
\end{table}

\begin{table}[th]
	\centering
	\begin{tabular}{ |c|c|c|c|c|c|c|c| }
		\hline
		\rowcolor{LightCyan}
		Clasificador & Tiempo (s) & \multicolumn{3}{c|}{Entrenamiento} & \multicolumn{3}{c|}{Test} \\
		\hline
		\rowcolor{LightCyan}
		&            & Acc & MS & F1 & Acc & MS & F1 \\
		\hline
		\textit{Decission tree} & 1.220  & 0.998 & 0.992 & 0.998 & 0.938 & 0.670 & 0.975 \\
		\textit{Random forest}  & 31.201 & 0.998 & 0.993 & 0.998 & 0.953 & 0.670 & 0.980 \\
		\textit{K-NN}           & 0.083  & 0.951 & 0.431 & 0.980 & 0.936 & 0.333 & 0.974 \\
		\hline
	\end{tabular}
	\caption{Clasificación multiclase con la nueva codificación.}
	\label{tabla:multi_new}
\end{table}

\vspace{1em}

Por último, se han considerado otras opciones para mejorar la clasificación de las clases minoritarias, pero podrían exceder la complejidad de este proyecto:

\begin{itemize}
	\item Utilizar métodos de sobremuestreo, ya mencionados en la sección \ref{subsubsec:oversampling}, que consisten en aumentar la cantidad de patrones de estas clases de forma sintética.
	\item Utilizar métodos jerárquicos que primero clasifiquen usando la categoría \textit{otros}, para después dividirla en sus diferentes clases y entrenar un modelo específico.
\end{itemize}

\section{Preparación del entorno}
\label{sec:prep_entorno}

En esta sección se describe el entorno de trabajo utilizado por el alumno para la implementación de los modelos y la realización de las pruebas. El entorno se debe preparar de forma correcta, ya que puede afectar a la ejecución de los algoritmos y a la reproducibilidad de los experimentos. A continuación, se explican elementos del entorno como el lenguaje de programación, las bibliotecas y las características del equipo.

\subsection{Herramientas y bibliotecas}
\label{subsec:herramientas}

El desarrollo y la experimentación de este proyecto se han llevado a cabo empleando un conjunto de herramientas y bibliotecas muy utilizadas en la ciencia de datos. \textit{Python} ha sido el lenguaje de programación de este trabajo, ya que ofrece una fácil implementación de modelos, manipulación de datos y visualización de resultados. Su popularidad en se debe a su sintaxis sencilla, escalabilidad y amplia variedad de herramientas y bibliotecas \cite{python_ml}.

\vspace{1em}

En este proyecto se ha utilizado la versión 3.12 de \textit{Python}, elegida principalmente por su compatibilidad con las bibliotecas empleadas, en particular, con \textit{GridSearchCV}, que aprovechan la paralelización de procesos para mejorar el rendimiento. El problema encontrado es que los hilos no se cierran correctamente, es un comportamiento típico asociado a lo que en programación concurrente se denomina \textit{thread leakage} o hilos huérfanos. provoca que la memoria \textit{RAM} y la \textit{CPU} sigan siendo consumidas incluso después de que la ejecución haya terminado. En teoría, este problema no afecta al rendimiento de los clasificadores, pero puede afectar al tiempo de ejecución.

\vspace{1em}

Las bibliotecas utilizadas para construir un modelo y analizar los datos son \textit{Scikit-learn}, \textit{DLOrdinal}, \textit{Matplotlib}, \textit{NumPy}, \textit{Pandas}, \textit{LightGBM} y \textit{Seaborn}.

\subsubsection{\textit{Scikit-learn}}
\label{subsubsec:sklearn}

\textit{Scikit-learn} es un paquete de código abierto en \textit{Python} que ofrece una gran variedad de métodos de aprendizaje automático rápidos y eficientes, gracias a que usan bibliotecas compiladas en lenguajes como \textit{C++}, \textit{C} o \textit{Fortran}. Tiene detrás una comunidad activa que mantiene la documentación, corrige errores y asegura la calidad. Aunque no incluye todos los algoritmos usados en este proyecto, es una herramienta muy recomendable si necesitamos: transformación de datos, aprendizaje supervisado o evaluación de modelos \cite{hao2019scikit}.

\subsubsection{\textit{DLOrdinal}}
\label{subsubsec:dlordinal}

La biblioteca dlordinal incluye muchas de las metodologías más recientes de clasificación ordinal usando técnicas avanzadas de aprendizaje profundo. El enfoque ordinal de esta herramienta tiene el objetivo de aprovechar la información de orden presente en la variable objetivo usando funciones de pérdida, diversas capas de salida y otras estrategias \cite{dlordinal}. El módulo de dlordinal que nos ha resultado de utilidad para este proyecto ha sido la el conjunto de métricas que incluye para evaluar los modelos utilizados, ya que cuenta con algunas de las métricas que finalmente hemos usado: mínima sensibilidad y valor-F.

\subsubsection{\textit{Matplotlib}}
\label{subsubsec:matplotlib}

Para una mejor visualización de los datos obtenidos en los modelos utilizados, se ha usado la biblioteca Matplotlib, ya que incluye una gran cantidad de recursos para la representación gráfica de la información \cite{matplotlib}. Se ha usado, en combinación con \textit{Seaborn}, descrita en la sección \ref{subsubsec:seaborn}.

\subsubsection{\textit{NumPy}}
\label{subsubsec:numpy}

La biblioteca \textit{NumPy} tiene como objetivo principal dar soporte a la creación de vectores y matrices de grandes dimensiones, junto con una colección de funciones matemáticas con las que operar \cite{numpy}. Ha sido de gran utilidad en el desarrollo del proyecto, ya que el conjunto de datos con el que se ha trabajado es de un tamaño considerable, aunque no ha sido necesario hacer uso de las funciones que proporciona porque la mayor parte de los cálculos necesarios se hacen de manera interna en los modelos utilizados.

\subsubsection{\textit{Pandas}}
\label{subsubsec:pandas}

Pandas es una herramienta muy potente para el manejo, análisis y manipulación de datos. Incluye una amplia variedad de herramientas para: leer y escribir datos, reestructuración y segmentación, inserción y eliminación de columnas, mezcla y unión de datos y muchas funcionalidades más \cite{pandas}. Varias de ellas se han utilizado durante el desarrollo y la preparación del conjunto de datos.

\subsubsection{\textit{LightGBM}}
\label{subsubsec:lightgbm}

La biblioteca \textit{Light Gradient-Boosting Machine} por su nombre en inglés, es una infraestructura de aprendizaje automático basada en modelos de árboles de decisión \cite{lgbm}. Se puede usar en diferentes tareas, pero la importante para el análisis realizado el la de clasificación. Los principales algoritmos soportados son: \textit{Gradient Boosting Decision Trees (GBDT)}, el cual utiliza \textit{LGBMClassifier}, clasificador usado durante la experimentación, \textit{Dropouts meet Multiple Additive Regression Trees (Dart)} y \textit{Gradient-based One-Side Sampling (Goss)} \cite{lgbm_alg}.

\subsubsection{\textit{Seaborn}}
\label{subsubsec:seaborn}

Basada en \textit{Matplotlib}, \textit{Seaborn} proporciona una interfaz de alto nivel para generar gráficos estadisticos \cite{seaborn}. Es posible usar ambas bibliotecas de forma combinada para una mayor capacidad de visualización. Mientras que \textit{Matplotlib} ofrece un control detallado sobre cada elemento de la figura, \textit{Seaborn} simplifica la creación de visualizaciones complejas, incorporando estilos predefinidos y funciones específicas para el análisis de datos.

\subsection{Hardware}
\label{subsec:hw_usado}

El entrenamiento y evaluación de los modelos se ha realizado en el equipo del estudiante con las siguientes características: procesador \textit{Intel Core i7-4712MQ}, tarjeta gráfica \textit{NVIDIA GeForce 920M}, 16 GB de \textit{RAM} DDR3 y almacenamiento compuesto por un \textit{SSD Crucial MX500} de 250 GB y un \textit{HDD} de 1 TB. Este hardware permite la paralelización de los algoritmos en múltiples núcleos del procesador, lo que reduce significativamente los tiempos de entrenamiento, pero se encuentra muy limitado respecto al conjunto utilizado para clasificación multiclase y modelos más costosos como puede ser \textit{SVM}.

\subsection{Protocolo de experimentación y validación}
\label{subsec:protocolo_exper}

En esta sección se establecen las condiciones de evaluación del rendimiento de los modelos mencionados en la sección \ref{sec:modelos_utilizados} y se explican los procedimientos seguidos, las técnicas de validación usadas y los criterios que permiten medir de forma objetiva la calidad de las predicciones. Todo esto tiene objetivo de minimizar posibles sesgos, evitar el sobreajuste y obtener conclusiones fiables.

\subsubsection{Diseño experimental}
\label{subsubsec:diseño}

Inicialmente se han planteado tres formas de estructurar el diseño experimental y como se evaluarán posteriormente las pruebas. La primera ha sido comparar distintos modelos para cada tipo de clasificación. La segunda, comparar, clasificación binaria y multiclase para cada clasificador. Por último, se ha planteado la posibilidad de una combinación de ambas comparaciones. Este último caso se ha descarta porque, aunque puede ser interesante la comparación combinada por proporcionar una amplia visión del problema, duplica la carga de trabajo y puede exceder la complejidad del proyecto.

\vspace{1em}

La segunda opción planteada puede servir para comparar el rendimiento de uno o varios modelos según la naturaleza del problema y realizar un análisis de coste computacional. Son aspectos interesantes a estudiar, pero no entran dentro de los objetivos de este estudio.

\vspace{1em}

Finalmente se ha seleccionado la primera opción. Aunque el problema de la detección de malware puede enfocarse tanto para la simple detección de un programa malicioso como para identificar a que tipo pertenece, los problemas de clasificación binaria y multiclase tienen enfoques muy diferentes. Por otro lado, el conjunto de datos usado para clasificación multiclase contiene varias clases con muy pocos patrones y la comparación podría no ser justa.

\subsubsection{Validación de resultados}
\label{subsubsec:validacion}

Para evitar sesgos y resultados poco concluyentes se han empleado varias técnicas.

\begin{itemize}
	\item \textbf{Validación cruzada}: Se ha usado el parámetro \textit{cv} de \textit{GridSearchCV}. En general se han usado 5, aunque en algunos casos ha sido necesario ajustarlo por tiempo.
	\item \textbf{Validación cruzada estratificada adaptativa}: la función \textit{cv()} que encontramos en el Anexo \ref{sec:func_cv} ajusta el numero de particiones en caso de que una clase tenga menos muestras que particiones indicadas.
	\item \textbf{Particion entrenamiento/prueba}: se ha dividido el conjunto de datos en un 75-25 para entrenamiento y pruebas respectivamente usando la variable \textit{random\_state} con la semilla usada en las pruebas.
	\item \textbf{Repetición con semillas aleatorias}: para repetir los experimentos y tener una visión más amplia.
	\item \textbf{Ajuste de pesos de clase}: mediante \texttt{class\_weight = "balanced"} en los clasificadores en los que se encuentra disponible.
\end{itemize}

A pesar de todas estas técnicas, es bastante probable que las clases extremadamente minoritarias del conjunto de datos para la clasificación multiclase pueden tener una influencia muy limitada.

\subsubsection{Reproducibilidad}
\label{subsubsec:reproducibilidad}

Durante el desarrollo del código y de las pruebas, se han adoptado diferentes medidas para garantizar que las comparaciones entre modelos sean justas.

\begin{enumerate}
	\item \textbf{Fijación de semillas:}\\
		Se ha hecho uso de una semilla controlada dentro de un bucle para repetir el experimento. Con ella se controla:

		\begin{itemize}
			\item La partición aleatoria de test y entrenamiento.
			\item la inicialización interna de los clasificadores que aceptan \textit{randon\_state}.
		\end{itemize}
	\item \textbf{Número de repeticiones:}\\
		Si bien el número de repeticiones es ajustable dentro del código utilizado, para asegurar una justa comparación y por las limitaciones del equipo, se han usado 10 semillas en todos los experimentos. Esto permite obtener la media y la desviación típica de las métricas y reducir la variabilidad.
	\item \textbf{Control de parámetros:}\\
		Los hiperparámetros se optimizan con \textit{GridSearchCV} usando la misma rejilla para todas las semillas para poder tener una comparación coherente.
\end{enumerate}

Estas medidas permiten obtener los mismo resultados si se usan las mismas semillas, configuraciones y conjunto de datos.

\subsubsection{Control de parámetros}
\label{subsubsec:control}

Para la optimización de hiperparámetros hemos usado búsqueda en rejilla de \textit{GridSearchCV}. Esta técnica hace pruebas con todas las combinaciones posibles de los parámetros proporcionados y usa validación cruzada para garantizar la robustez de los resultados. El problema con esta técnica es el elevado número de pruebas, ya que se prueban todas las combinaciones de parámetros posibles en cada uno de los conjuntos de la validación cruzada, lo que eleva el tiempo necesario de manera considerable.

\vspace{1em}

Una opción considerada y probada para evitar esta limitación es la búsqueda aleatoria de \textit{RandomizedSearchCV}, que permite establecer un número máximo de combinaciones a probar y puede reducir considerablemente el número de combinaciones evaluadas. El inconveniente que ha surgido con esta técnica es que al disponer de un \textit{Hardware} muy limitado, la cantidad de combinaciones usadas es pequeña y limitar aun más con la búsqueda aleatoria puede suponer que los resultados sean menos representativos.

\vspace{1em}

La rejilla se ha establecido para cada modelo en función de las limitaciones del equipo, el tiempo necesario para el entrenamiento de cada modelo y cuanto influye ese parámetro en el tiempo de entrenamiento y el peso que tiene en los resultados.

\subsubsection{Criterios de evaluación}
\label{subsubsec:evaluacion}

Para evaluar la efectividad de los modelos implementados, se han utilizado las siguientes métricas, cuya descripción teórica se encuentra en la sección \ref{subsec:2_metricas}:

\begin{itemize}
	\item \textbf{\textit{Accuracy}}: proporción de predicciones correctas sobre el total de patrones.
	\item \textbf{Mínima Sensibilidad}: sensibilidad de la clase peor clasificada.
	\item \textbf{\textit{F1-score}}: media armónica entre precisión y sensibilidad. En este documento se ha hecho referencia a ella como valor-F.
\end{itemize}

\section{Implementación y pruebas}
\label{sec:implementacion}

En esta sección se describe, principalmente, la estructura del código empleado para realizar los experimentos y el procedimiento seguido dentro del mismo para entrenar y evaluar los modelos seleccionados. Además se va a tratar la preparación del conjunto de datos, es decir, cómo se cargan y cómo se divide la información para realizar el entrenamiento y las pruebas. Por último, se tratará la forma que hemos seguido para presentar los resultados y las métricas que se han mencionado en la sección \ref{subsubsec:evaluacion}

\subsection{Estructura del código}
\label{subsec:estructura_codigo}
% TODO: Se describe cómo se organizan los scripts y módulos, indicando la función de cada uno en el flujo de entrenamiento, evaluación y análisis de resultados.



\subsection{Procedimiento de entrenamiento y evaluación}
\label{subsec:procedimiento}

El planteamiento seguido para entrenar los diferentes modelos ha sido usar \textit{GridSearchCV} para ajustar los modelos de clasificación con los mejores parámetros posibles. Para obtener una visión más amplia y más justa del problema, se ha repetido el entrenamiento, con las mismas 10 semillas para todos los modelos. Con esto conseguimos que el experimento sea controlado y reproducible, ya que para un mismo modelo, una misma semilla y la misma rejilla de parámetros, obtendremos siempre los mismos resultados. Una vez calculadas las métricas seleccionadas en la sección \ref{subsubsec:evaluacion}, se calcula la media y la desviación típica de todas ellas para usarlas como valor final de comparación entre modelos.

\subsection{Preparación y uso de los conjuntos de datos}
\label{subsec:datos_experimentales}

Además del tratamiento previo del conjunto de datos realizado en la sección \ref{sec:proc_dataset}, es necesario procesar la información antes de entrenar. Con la función \textit{load}, cargamos el conjunto de datos en dos matrices de \textit{Numpy}, la matriz de información y la matriz de clases. La matriz de patrones de entrada se normaliza haciendo uso de la clase \textit{MinMaxScaler} del módulo \textit{preprocessing} de \textit{Scikit-Learn}. Por último, haciendo uso de la función \textit{train\_test\_split}, dividimos el conjunto de datos en test y entrenamiento. Esto se hace dentro del bucle y para cada semilla con el objetivo de tener una evaluación más robusta, ya que permite tener una división distinta y controlada para cada semilla.

\subsection{Métricas y análisis de resultados}
\label{subsec:metricas_pruebas}

Para calcular las métricas se han usado las funciones \textit{accuracy\_off1} y \textit{minimum\_sensitivity} para las métricas valor-F y mínima sensibilidad respectivamente. Estas se encuentran disponibles en el módulo \textit{metrics} de la librería \textit{dlordinal}. Para calcular la exactitud o \textit{accuracy} del entrenamiento, se ha usado la función \textit{accuracy\_score} disponible en el módulo \textit{metrics} de la librería \textit{Scikit-Learn}. Finalmente, una vez calculados los resultados para todas las semillas, se guardan en un objeto \textit{DataFrame} de \textit{Pandas} con el formato que se muestra en el ejemplo del Anexo \ref{sec:info}. Haciendo uso de los métodos \textit{mean} y \textit{std} de esta clase, se obtiene la media y la desviación típica de todas las semillas.

\chapter{Resultados y discusión}
\label{ch:resultados}

En este capítulo se presentan los resultados obtenidos tras aplicar los distintos algoritmos de clasificación sobre los conjuntos de datos preparados en las fases anteriores. El objetivo principal es evaluar el rendimiento de cada modelo bajo diferentes escenarios y métricas, con el fin de identificar sus fortalezas y limitaciones en la detección de \textit{malware}.

\vspace{1em}

Para ello, se analizan tanto los experimentos realizados en el problema binario, donde se distingue entre \textit{software} malicioso y benigno, como en el problema multiclase, en el que se busca identificar el tipo específico de amenaza. Los clasificadores se evalúan considerando como métricas: la sensibilidad mínima y la precisión de los modelos.

\vspace{1em}

Asimismo, se discute la capacidad de generalización de cada modelo, observando las diferencias de rendimiento entre los conjuntos de entrenamiento y prueba, así como la influencia de la variabilidad introducida por el desbalanceo de clases. Con este análisis se pretende ofrecer una visión comparativa que facilite la elección del modelo más adecuado en un contexto práctico de detección de amenazas.

\newpage
\section{Clasificación binaria}
\label{sec:clas_binaria}

En primer lugar se expondrán los resultados obtenidos en clasificación binaria.

\subsection{Árboles de decisión}
\label{subsec:dt_bin}

La tabla \ref{tabla:dt_bin} muestra los resultados de la clasificación binaria utilizando el modelo \texttt{DecisionTreeClassifier}, evaluado en 10 ejecuciones distintas con diferentes estados.

\begin{table}[H]
	\centering
	\begin{tabular}{ |c|c|c|c|c| }
		\hline
		\rowcolor{LightCyan}
		 & \multicolumn{2}{c|}{Entrenamiento} & \multicolumn{2}{c|}{Generalización} \\
		\hline
		\rowcolor{LightCyan}
		Semilla & Acc & MS & Acc & MS \\
		\hline
		0    & \textbf{1.000} & \textbf{1.000} & \textit{0.951} & 0.942           \\
		1    & \textit{1.000} & \textit{1.000} & 0.944          & 0.935           \\
		2    & 1.000          & 1.000          & 0.942          & 0.935           \\
		3    & 1.000          & 1.000          & 0.948          & 0.938           \\
		4    & 1.000          & 1.000          & \textbf{0.953} & \textbf{0.946}  \\
		5    & 0.998          & 0.997          & 0.947          & 0.936           \\
		6    & 0.998          & 0.997          & 0.947          & 0.941           \\
		7    & 1.000          & 1.000          & 0.949          & \textit{0.946}  \\
		8    & 0.999          & 0.998          & 0.948          & 0.937           \\
		9    & 1.000          & 1.000          & 0.950          & 0.940           \\
		Mean & 0.999          & 0.999          & 0.948          & 0.940           \\
		STD  & 0.001          & 0.001          & 0.003          & 0.004           \\
		\hline
	\end{tabular}
	\caption{Resultados en entrenamiento y generalización para las distintas semillas en clasificación binaria con \texttt{DecisionTreeClassifier}.}
	\label{tabla:dt_bin}
\end{table}

Para el entrenamiento, la precisión (\textit{Acc}), la sensibilidad mínima (\textit{MS}) y el \textit{Valor-F1} alcanzan valores muy próximos a 1 en todas las ejecuciones. En cuanto al test, la precisión oscila entre 0.942 y 0.953, mientras que la sensibilidad mínima se sitúa entre 0.935 y 0.946.

\vspace{1em}

la figura \ref{fig:dt_bin} compara la distribución de los valores de \textit{accuracy} en entrenamiento y en test para el clasificador basado en árboles de decisión. Se observa que en entrenamiento son consistentemente cercanos a 1, sin apenas variabilidad, lo que indica que el modelo es capaz de ajustarse casi perfectamente a los datos de entrenamiento.

\newpage

En contraste, los valores en test muestran una ligera caída, con una variabilidad mayor que en entrenamiento. Esta diferencia refleja que el modelo generaliza de forma aceptable, aunque la brecha respecto al rendimiento en entrenamiento sugiere la existencia de cierto sobreajuste.

\vspace{1em}

El uso combinado de \texttt{violinplot} y \texttt{boxplot} permite apreciar tanto la concentración de los valores en torno a la media como la dispersión entre diferentes ejecuciones. En este caso, el test mantiene una distribución compacta, sin valores atípicos extremos, lo que aporta robustez a la evaluación del modelo.

\begin{figure}[H]
	\centering
	\includegraphics[width=1\linewidth]{Imagenes/dt_bin}
	\caption[\texttt{Boxplot} con \texttt{violinplot} para árboles de decisión]{\texttt{Boxplot} con \texttt{violinplot} para árboles de decisión.}
	\label{fig:dt_bin}
\end{figure}

\newpage
\subsection{\textit{Random forest}}
\label{subsec:rf_bin}

En la tabla \ref{tabla:rf_bin} se muestran los resultados de la clasificación binaria utilizando el modelo \texttt{RandomForestClassifier} para diferentes estados aleatorios.

\begin{table}[H]
	\centering
	\begin{tabular}{ |c|c|c|c|c| }
		\hline
		\rowcolor{LightCyan}
		 & \multicolumn{2}{c|}{Entrenamiento} & \multicolumn{2}{c|}{Generalización} \\
		\hline
		\rowcolor{LightCyan}
		 Semilla & Acc & MS & Acc & MS \\
		\hline
		0    & 0.980          & 0.928          & 0.939          & 0.524          \\
		1    & 0.980          & 0.929          & 0.939          & 0.500          \\
		2    & 0.980          & 0.927          & \textbf{0.941} & 0.429          \\
		3    & 0.979          & 0.923          & 0.939          & 0.444          \\
		4    & 0.982          & 0.931          & 0.939          & 0.500          \\
		5    & 0.980          & 0.929          & 0.937          & \textbf{0.609} \\
		6    & 0.978          & 0.920          & 0.938          & 0.550          \\
		7    & 0.980          & 0.929          & 0.939          & 0.562          \\
		8    & 0.982          & 0.934          & 0.938          & 0.489          \\
		9    & \textbf{0.989} & \textbf{0.956} & 0.936          & 0.400          \\
		Mean & 0.981          & 0.931          & 0.939          & 0.501          \\
		STD  & 0.003          & 0.010          & 0.001          & 0.064          \\
		\hline
	\end{tabular}
	\caption{Resultados en entrenamiento y generalización para las distintas semillas en clasificación binaria con \texttt{RandomForestClassifier}.}
	\label{tabla:rf_bin}
\end{table}

En el conjunto de entrenamiento, las métricas presentan valores altos y consistentes: la precisión oscila entre 0.978 y 0.989, la métrica de sensibilidad mínima entre 0.920 y 0.956.

\vspace{1em}

En el conjunto de prueba, la precisión mantiene valores muy estables alrededor de 0.936–0.941. La sensibilidad mínima muestra una mayor variabilidad, con valores que van desde 0.400 hasta 0.609.

\begin{figure}[H]
	\centering
	\includegraphics[width=1\linewidth]{Imagenes/rf_bin}
	\caption[\texttt{Boxplot} con \texttt{violinplot} para \texttt{RandomForestClassifier}]{\texttt{Boxplot} con \texttt{violinplot} para \texttt{RandomForestClassifier}.}
	\label{fig:rf_bin}
\end{figure}

El modelo muestra un rendimiento muy alto y consistente en precisión tanto en el entrenamiento como en el test. En la figura \ref{fig:rf_bin}, se observa que la distribución de \textit{accuracy} es estrecha, con valores muy concentrados en torno a 0.98 en entrenamiento y 0.94 en test, lo que indica estabilidad en ambas fases.

\newpage
Sin embargo, la métrica de sensibilidad mínima (\textit{MS}) introduce un matiz importante: aunque en entrenamiento se mantiene elevada con poca dispersión, en test presenta una gran variabilidad. El rango de valores oscila entre 0.400 y 0.609 según la semilla utilizada, lo que se traduce en distribuciones más amplias en los gráficos. Esto indica que, en algunos casos, el modelo no logra identificar de forma adecuada los ejemplos más difíciles de una de las clases, comprometiendo la robustez de la clasificación en escenarios concretos.

\subsection{\textit{K-NN}}
\label{subsec:knn_bin}

En la tabla \ref{tabla:knn_bin} se muestran los resultados obtenidos con el clasificador \texttt{KNeighborsClassifier} bajo diferentes semillas.

\begin{table}[H]
	\centering
	\begin{tabular}{ |c|c|c|c|c| }
		\hline
		\rowcolor{LightCyan}
		 & \multicolumn{2}{c|}{Entrenamiento} & \multicolumn{2}{c|}{Generalización} \\
		\hline
		\rowcolor{LightCyan}
		 Semilla & Acc & MS & Acc & MS \\
		\hline
		0    & 1.000          & 1.000          & 0.947          & 0.935          \\
		1    & 1.000          & 1.000          & 0.949          & 0.938          \\
		2    & 1.000          & 1.000          & 0.939          & 0.926          \\
		3    & 1.000          & 1.000          & 0.949          & 0.937          \\
		4    & 1.000          & 1.000          & 0.949          & 0.936          \\
		5    & 1.000          & 1.000          & 0.946          & 0.932          \\
		6    & 1.000          & 1.000          & 0.946          & 0.931          \\
		7    & 1.000          & 1.000          & 0.944          & 0.934          \\
		8    & 1.000          & 1.000          & 0.947          & 0.935          \\
		9    & \textbf{1.000} & \textbf{1.000} & \textbf{0.949} & \textbf{0.940} \\
		Mean & 1.000          & 1.000          & 0.947          & 0.934          \\
		STD  & 0.000          & 0.000          & 0.003          & 0.004          \\
		\hline
	\end{tabular}
	\caption{Resultados en entrenamiento y generalización para las distintas semillas en clasificación binaria con \texttt{KNeighborsClassifier}.}
	\label{tabla:knn_bin}
\end{table}

En la fase de entrenamiento, todas las semillas reportan valores de \textit{Acc} y \textit{MS} iguales a 1.000, lo que refleja una completa uniformidad en las métricas. La media confirma este comportamiento perfecto, con desviaciones estándar nulas en las tres métricas.

\newpage
En la fase de generalización, la precisión presenta valores que oscilan entre 0.939 y 0.949, con una media de 0.947 y una desviación estándar reducida de 0.003. \textit{MS} toma valores entre 0.926 y 0.940, con una media de 0.934 y una desviación estándar de 0.004, mostrando ligeras variaciones según la semilla utilizada.

\vspace{1em}

La figura \ref{fig:knn_bin} muestra cómo para el entrenamiento, todos los valores de \textit{accuracy} se encuentran exactamente en 1.000, lo que se refleja en el \texttt{violinplot} como una única línea y en el \texttt{boxplot} como una caja colapsada. Esto indica que el modelo clasifica perfectamente todos los patrones del conjunto de entrenamiento en cada ejecución.

\vspace{1em}

Para la generalización, los valores oscilan ligeramente entre 0.939 y 0.949. El \texttt{violinplot} muestra una distribución muy estrecha alrededor de la media, y el \texttt{boxplot} confirma que la mediana es cercana a 0.947, con una ligera variabilidad reflejada por los bigotes.

\vspace{1em}

Se aprecia un entrenamiento perfecto y una generalización muy alta y consistente, con poca dispersión en los resultados de test, aunque podemos ver una ligera caída en las métricas.

\vspace{1em}

Para ver más clara la caída en las métricas anteriores, podemos hacer uso de las matrices de confusión reflejadas en la figura \ref{figT:knn_mat}. En ella podemos ver que la clasificación en entrenamiento es perfecta, acertando en todos los patrones, mientras que el para la generalización los resultados son peores por un aumento significativo de los falsos positivos y los falsos negativos. Esto podría significar un ligero sobreajuste del modelo.

\begin{figure}[H]
	\centering
	\includegraphics[width=1\linewidth]{Imagenes/knn_bin}
	\caption[\texttt{Boxplot} con \texttt{violinplot} para \textit{K-NN}]{\texttt{Boxplot} con \texttt{violinplot} para \textit{K-NN}.}
	\label{fig:knn_bin}
\end{figure}

\begin{figure}[H]
	\begin{subfigure}{.5\textwidth}
		\centering
		\includegraphics[width=1\linewidth]{Imagenes/knn_bin_mat_train}
		\caption{Matriz de confusión del entrenamiento en la primera semilla.}
		\label{fig:sub-first}
	\end{subfigure}
	\begin{subfigure}{.5\textwidth}
		\centering
		\includegraphics[width=1\linewidth]{Imagenes/knn_bin_mat_test}
		\caption{Matriz de confusión de la generalización en la primera semilla.}
		\label{fig:sub-second}
	\end{subfigure}
	\caption[Matriz de confusión en \textit{K-NN}]{Matriz de confusión en \textit{K-NN}.}
	\label{figT:knn_mat}
\end{figure}

\newpage
\subsection{Máquinas de vectores de soporte}
\label{subsec:svm_bin}

En la tabla \ref{fig:svm_bin} podemos ver cómo para el conjunto de entrenamiento, la precisión se mantiene en valores cercanos a 0.76 en todas las ejecuciones, mientras que (\textit{MS}) muestra variaciones entre 0.656 y 0.704.

\vspace{1em}

En el conjunto de generalización, todas las métricas presentan unos resultados muy similares a los de entrenamiento. Esto se puede apreciar claramente observando la desviación estándar (STD), que indica una baja variabilidad en todas las métricas.

\begin{table}[H]
	\centering
	\begin{tabular}{ |c|c|c|c|c| }
		\hline
		\rowcolor{LightCyan}
		 & \multicolumn{2}{c|}{Entrenamiento} & \multicolumn{2}{c|}{Generalización} \\
		\hline
		\rowcolor{LightCyan}
		 Semilla & Acc & MS & Acc & MS \\
		\hline
		0    & 0.757          & 0.656          & 0.764          & 0.672          \\
		1    & 0.761          & 0.671          & \textbf{0.769} & 0.681          \\
		2    & 0.766          & 0.702          & 0.757          & 0.699          \\
		3    & 0.762          & 0.700          & 0.766          & \textbf{0.704} \\
		4    & 0.760          & 0.684          & 0.768          & 0.696          \\
		5    & 0.761          & 0.663          & 0.753          & 0.655          \\
		6    & 0.762          & 0.702          & 0.762          & 0.683          \\
		7    & 0.763          & 0.699          & 0.759          & 0.697          \\
		8    & \textbf{0.766} & \textbf{0.704} & 0.758          & 0.693          \\
		9    & 0.760          & 0.662          & 0.758          & 0.666          \\
		Mean & 0.762          & 0.684          & 0.762          & 0.685          \\
		STD  & 0.003          & 0.020          & 0.005          & 0.016          \\
		\hline
	\end{tabular}
	\caption{Resultados en entrenamiento y generalización para las distintas semillas en clasificación binaria con \texttt{SVC}.}
	\label{tabla:svm_bin}
\end{table}

Se muestran valores muy consistentes tanto en el conjunto de entrenamiento como en el de test. En la figura \ref{fig:svm_bin}, las distribuciones para ambos conjuntos aparecen concentradas en torno al 0.76, sin grandes variaciones entre diferentes semillas. Esto queda reforzado por la baja dispersión que se observa en el \texttt{boxplot} y por la forma compacta del \texttt{violinplot}, que refleja que no existen valores extremos significativos.

\vspace{1em}

Aunque no se han obtenido los mejores resultados con las máquinas de vectores de soporte, es interesante la estabilidad que consiguen a la hora de generalizar. Esta cercanía entre ambas distribuciones indica que el modelo mantiene un rendimiento estable al generalizar sobre datos no vistos.

\begin{figure}[H]
	\centering
	\includegraphics[width=1\linewidth]{Imagenes/svm_bin}
	\caption[\texttt{Boxplot} con \texttt{violinplot} para \textit{SVC}]{\texttt{Boxplot} con \texttt{violinplot} para \textit{SVC}.}
	\label{fig:svm_bin}
\end{figure}

\newpage
\subsection{\textit{Ridge}}
\label{subsec:ridge_bin}

En los resultados obtenidos con el clasificador \texttt{RidgeClassifier}, las métricas de entrenamiento muestran valores estables. La precisión se mantiene en un rango estrecho entre 0.645 y 0.652, con una media de 0.649 y una desviación estándar de 0.002. La mínima sensibilidad presenta mayor variabilidad, con valores comprendidos entre 0.549 y 0.573, alcanzando una media de 0.564 y una desviación estándar de 0.008.

\vspace{1em}

En el conjunto de generalización, la precisión presenta una media de 0.648, muy próxima a la de entrenamiento, y oscila entre 0.639 y 0.655, con una desviación estándar de 0.005. La mínima sensibilidad alcanza una media de 0.561, con valores entre 0.530 y 0.573 y una desviación estándar de 0.014.

\begin{table}[H]
	\centering
	\begin{tabular}{ |c|c|c|c|c| }
		\hline
		\rowcolor{LightCyan}
		 & \multicolumn{2}{c|}{Entrenamiento} & \multicolumn{2}{c|}{Generalización} \\
		\hline
		\rowcolor{LightCyan}
		 Semilla & Acc & MS & Acc & MS \\
		\hline
		0    & 0.649          & 0.549          & 0.648          & 0.530          \\
		1    & 0.645          & 0.558          & \textbf{0.655} & 0.569          \\
		2    & \textbf{0.652} & \textbf{0.573} & 0.645          & 0.564          \\
		3    & 0.649          & 0.567          & 0.653          & 0.570          \\
		4    & 0.651          & 0.573          & 0.651          & \textbf{0.573} \\
		5    & 0.647          & 0.562          & 0.648          & 0.558          \\
		6    & 0.648          & 0.556          & 0.650          & 0.573          \\
		7    & 0.651          & 0.571          & 0.650          & 0.573          \\
		8    & 0.651          & 0.564          & 0.639          & 0.551          \\
		9    & 0.650          & 0.563          & 0.645          & 0.551          \\
		Mean & 0.649          & 0.564          & 0.648          & 0.561          \\
		STD  & 0.002          & 0.008          & 0.005          & 0.014          \\
		\hline
	\end{tabular}
	\caption{Resultados en entrenamiento y generalización para las distintas semillas en clasificación binaria con \texttt{RidgeClassifier}.}
	\label{tabla:ridge_bin}
\end{table}

En la figura \ref{fig:ridge_bin}, se observa que ambos conjuntos presentan valores muy próximos. Las cajas muestran una dispersión reducida, con rangos estrechos tanto en entrenamiento como en test. Esto se traduce en que el modelo ofrece resultados consistentes sin grandes fluctuaciones entre diferentes semillas.

\vspace{1em}

La forma del \texttt{violinplot} refleja estabilidad en el rendimiento, con pequeñas variaciones entre ejecuciones.

\begin{figure}[H]
	\centering
	\includegraphics[width=1\linewidth]{Imagenes/ridge_bin}
	\caption[\texttt{Boxplot} con \texttt{violinplot} para \texttt{RidgeClassifier}]{\texttt{Boxplot} con \texttt{violinplot} para \texttt{RidgeClassifier}.}
	\label{fig:ridge_bin}
\end{figure}

\newpage
\subsection{Perceptrón multicapa}
\label{subsec:mlp_bin}

En la tabla \ref{tabla:mlp_bin} se muestran los valores de precisión y mínima sensibilidad tanto en entrenamiento como en generalización para las distintas semillas utilizadas.

\begin{table}[H]
	\centering
	\begin{tabular}{ |c|c|c|c|c| }
		\hline
		\rowcolor{LightCyan}
		 & \multicolumn{2}{c|}{Entrenamiento} & \multicolumn{2}{c|}{Generalización} \\
		\hline
		\rowcolor{LightCyan}
		 Semilla & Acc & MS & Acc & MS \\
		\hline
		0    & 0.783          & 0.771          & 0.789          & 0.778          \\
		1    & 0.788          & 0.736          & \textbf{0.792} & 0.740          \\
		2    & 0.788          & 0.750          & 0.782          & 0.739          \\
		3    & 0.733          & 0.605          & 0.737          & 0.609          \\
		4    & 0.767          & 0.759          & 0.769          & 0.760          \\
		5    & \textbf{0.790} & 0.736          & 0.783          & 0.730          \\
		6    & 0.777          & \textbf{0.772} & 0.783          & \textbf{0.781} \\
		7    & 0.774          & 0.767          & 0.770          & 0.763          \\
		8    & 0.778          & 0.704          & 0.772          & 0.705          \\
		9    & 0.788          & 0.762          & 0.784          & 0.751          \\
		Mean & 0.776          & 0.736          & 0.776          & 0.736          \\
		STD  & 0.017          & 0.051          & 0.016          & 0.050          \\
		\hline
	\end{tabular}
	\caption{Resultados en entrenamiento y generalización para las distintas semillas en clasificación binaria con \texttt{MLPClassifier}.}
	\label{tabla:mlp_bin}
\end{table}

Los resultados de \textit{accuracy} en entrenamiento oscilan entre 0.733 y 0.790, mientras que en generalización se sitúan entre 0.737 y 0.792.

\vspace{1em}

En cuanto a MS, los valores en entrenamiento varían desde 0.605 hasta 0.772, y en generalización desde 0.609 hasta 0.781.

\vspace{1em}

La fila de medias muestra que tanto en entrenamiento como en generalización los promedios de Acc y MS coinciden en 0.776 y 0.736, respectivamente. Las desviaciones típicas reflejan mayor variabilidad en MS frente a la precisión.

\vspace{1em}

La figura \ref{fig:mlp_bin} muestra la distribución de los valores de precisión en entrenamiento y en generalización para las distintas semillas utilizadas.

\begin{figure}[H]
	\centering
	\includegraphics[width=1\linewidth]{Imagenes/mlp_bin}
	\caption[\texttt{Boxplot} con \texttt{violinplot} para \texttt{MLPClassifier}]{\texttt{Boxplot} con \texttt{violinplot} para \texttt{MLPClassifier}.}
	\label{fig:mlp_bin}
\end{figure}

En entrenamiento, la caja es relativamente compacta, aunque se observa un valor que queda por debajo de la mayoría de las mediciones. Este caso podría considerarse ruido, pero sería necesaria una muestra de resultados mayor para poder asegurarlo.

\vspace{1em}

En generalización, la distribución es muy similar, con valores que oscilan entre 0.73 y 0.79 y una mediana también próxima a 0.78. La dispersión es comparable a la observada en el entrenamiento, mostrando que los resultados se mantienen consistentes entre ambos conjuntos.

\vspace{1em}

El \texttt{violinplot} refleja que en ambas situaciones la densidad principal se concentra en torno a la franja de 0.77–0.79, indicando que la mayoría de ejecuciones producen resultados en ese rango.

\subsection{\textit{Light gradient boosting machine}}
\label{subsec:lgbm_bin}

Los valores de precisión y mínima sensibilidad obtenidos en entrenamiento y generalización para las distintas semillas se muestran en la tabla \ref{tabla:lgbm_bin}

\begin{table}[H]
	\centering
	\begin{tabular}{ |c|c|c|c|c| }
		\hline
		\rowcolor{LightCyan}
		 & \multicolumn{2}{c|}{Entrenamiento} & \multicolumn{2}{c|}{Generalización} \\
		\hline
		\rowcolor{LightCyan}
		 Semilla & Acc & MS & Acc & MS \\
		\hline
		0    & 0.984          & 0.981          & 0.953          & \textbf{0.952} \\
		1    & 0.984          & 0.980          & 0.951          & 0.947          \\
		2    & 0.985          & 0.983          & 0.949          & 0.946          \\
		3    & 0.985          & 0.982          & 0.952          & 0.951          \\
		4    & 0.984          & 0.981          & 0.950          & 0.945          \\
		5    & 0.985          & 0.981          & 0.949          & 0.948          \\
		6    & 0.985          & 0.982          & 0.952          & 0.949          \\
		7    & 0.986          & 0.984          & 0.948          & 0.947          \\
		8    & 0.984          & 0.979          & 0.953          & 0.952          \\
		9    & \textbf{0.989} & \textbf{0.989} & \textbf{0.953} & 0.950          \\
		Mean & 0.985          & 0.982          & 0.951          & 0.949          \\
		STD  & 0.002          & 0.003          & 0.002          & 0.002          \\
		\hline
	\end{tabular}
	\caption{Resultados en entrenamiento y generalización para las distintas semillas en clasificación binaria con \texttt{LGBMClassifier}}
	\label{tabla:lgbm_bin}
\end{table}

En el conjunto de entrenamiento, los valores de \textit{accuracy} se sitúan entre 0.984 y 0.989, con una media de 0.985 y una desviación estándar de 0.002. MS toma valores entre 0.979 y 0.989, con una media de 0.982 y desviación estándar de 0.003.

\vspace{1em}

En generalización, la precisión oscila entre 0.948 y 0.953, con una media de 0.951 y desviación estándar de 0.002. Para MS, los resultados están entre 0.945 y 0.952, con una media de 0.949 y desviación estándar de 0.002.

\begin{figure}[H]
	\centering
	\includegraphics[width=1\linewidth]{Imagenes/lgbm_bin}
	\caption[\texttt{Boxplot} con \texttt{violinplot} para \texttt{LGBMClassifier}]{\texttt{Boxplot} con \texttt{violinplot} para \texttt{LGBMClassifier}.}
	\label{fig:lgbm_bin}
\end{figure}

En la figura \ref{fig:lgbm_bin}, vemos que el modelo muestra un rendimiento muy consistente, con distribución de precisión estrecha y centrada en torno a valores elevados tanto en entrenamiento como en generalización. La cercanía de ambas distribuciones indica que no existe un sobreajuste significativo, ya que el comportamiento en entrenamiento y en test es muy similar.

\newpage
Además, la baja dispersión en las métricas sugiere que el modelo es estable frente a las distintas semillas de inicialización, lo que respalda su fiabilidad al aplicarse en diferentes ejecuciones. En general, se trata de un modelo robusto, con buen equilibrio entre ajuste a los datos de entrenamiento y capacidad de generalización.

\subsection{Discusión de los resultados}
\label{subsec:discusion}

Las tablas \ref{tabla:resumen_modelos} y \ref{tabla:resumen_std}, resumen los valores medios y las desviaciones típicas de las métricas Acc y MS en entrenamiento y generalización para los distintos modelos implementados.

En la tabla \ref{tabla:resumen_modelos}, se observa que varios modelos alcanzan valores altos y consistentes de precisión, con pequeñas variaciones entre entrenamiento y generalización. Sin embargo, existen diferencias notables en la estabilidad de la métrica de mínima sensibilidad, lo que permite distinguir entre metodologías más o menos equilibradas a la hora de clasificar correctamente todas las clases.

\begin{table}[H]
	\centering
	\begin{tabular}{|c|c|c|c|c|}
		\hline
		\rowcolor{LightCyan}
		Modelo & \multicolumn{2}{c|}{Entrenamiento} & \multicolumn{2}{c|}{Generalización} \\
		\hline
		\rowcolor{LightCyan}
		Clasificador & Acc & MS & Acc & MS \\
		\hline
		\texttt{DecisionTreeClassifier} & 0.981 & 0.931 & 0.939 & 0.501 \\
		\texttt{RandomForestClassifier} & 0.981 & 0.931 & 0.939 & 0.501 \\
		\texttt{KNeighborsClassifier}   & 1.000 & 1.000 & 0.947 & 0.934 \\
		\texttt{SVC}                    & 0.762 & 0.684 & 0.762 & 0.685 \\
		\texttt{RidgeClassifier}        & 0.649 & 0.564 & 0.648 & 0.561 \\
		\texttt{MLPClassifier}          & 0.776 & 0.736 & 0.776 & 0.736 \\
		\texttt{LGBMClassifier}         & 0.985 & 0.982 & 0.951 & 0.949 \\
		\hline
	\end{tabular}
	\caption{Resumen de resultados medios en entrenamiento y generalización para cada modelo}
	\label{tabla:resumen_modelos}
\end{table}

La tabla \ref{tabla:resumen_std} refleja la variabilidad de los resultados frente al uso de diferentes semillas de inicialización. Mientras que algunos modelos poca variación, otros presentan una dispersión algo más elevada, especialmente en la métrica de mínima sensibilidad, lo que señala una mayor dependencia de la partición de datos y menor consistencia en la clasificación.

\begin{table}[H]
	\centering
	\begin{tabular}{|c|c|c|c|c|}
		\hline
		\rowcolor{LightCyan}
		Modelo & \multicolumn{2}{c|}{Entrenamiento} & \multicolumn{2}{c|}{Generalización} \\
		\hline
		\rowcolor{LightCyan}
		& Acc & MS & Acc & MS \\
		\hline
		\texttt{DecisionTreeClassifier} & 0.004 & 0.016 & 0.014 & 0.016 \\
		\texttt{RandomForestClassifier} & 0.004 & 0.016 & 0.014 & 0.016 \\
		\texttt{KNeighborsClassifier}   & 0.000 & 0.000 & 0.003 & 0.004 \\
		\texttt{SVC}                    & 0.003 & 0.020 & 0.005 & 0.016 \\
		\texttt{RidgeClassifier}        & 0.002 & 0.008 & 0.005 & 0.014 \\
		\texttt{MLPClassifier}          & 0.017 & 0.051 & 0.016 & 0.050 \\
		\texttt{LGBMClassifier}         & 0.002 & 0.003 & 0.002 & 0.002 \\
		\hline
	\end{tabular}
	\caption{Resumen de desviaciones típicas en entrenamiento y generalización para cada modelo}
	\label{tabla:resumen_std}
\end{table}

Observando estos resultados, podemos destacar ligeramente las siguientes metodologías:

\begin{itemize}
	\item \texttt{LGBMClassifier}: Presenta valores medios muy elevados en entrenamiento y en generalización y sus desviaciones típicas son mínimas, lo que demuestra alta estabilidad y consistencia. Es un modelo con gran capacidad de generalización y muy bajo riesgo de sobreajuste.
	\item \texttt{KNeighborsClassifier}: Obtiene resultados muy altos y equilibrados entre entrenamiento y generalización y sus desviaciones típicas son extremadamente bajas, con comportamiento muy estable. Aunque no alcanza los valores máximos absolutos de otros métodos, la estabilidad y consistencia lo convierten en una opción muy competitiva.
\end{itemize}

En la gráfica representada en la figura \ref{fig:vs_bin} podemos ver que entre estas dos opciones, el clasificador \texttt{LGBMClassifier} obtiene un rendimiento ligeramente superior.

\begin{figure}[H]
	\centering
	\includegraphics[width=1\linewidth]{Imagenes/vs_bin}
	\caption[Comparativa entre los modelos \texttt{KNegihborsClassifier} y \texttt{LGBMClassifier}]{Comparativa entre los modelos \texttt{KNegihborsClassifier} y \texttt{LGBMClassifier}.}
	\label{fig:vs_bin}
\end{figure}

\newpage
\section{Clasificación multiclase}
\label{sec:clas_multi}

A continuación se comentarán los resultados de los diferentes modelos para tener una visión general de como se comportan en la clasificación multiclase disponible en nuestro conjunto de datos.

\subsection{Árboles de decisión}
\label{subsec:dt_multi}

En la tabla \ref{tabla:dt_multi} se muestran los valores de accuracy y mínima sensibilidad.

\begin{table}[H]
	\centering
	\begin{tabular}{ |c|c|c|c|c| }
		\hline
		\rowcolor{LightCyan}
		 & \multicolumn{2}{c|}{Entrenamiento} & \multicolumn{2}{c|}{Generalización} \\
		\hline
		\rowcolor{LightCyan}
		 Estado aleatorio & Acc & MS & Acc & MS \\
		\hline
		0    & 0.980          & 0.928          & 0.939          & 0.524          \\
		1    & 0.980          & 0.929          & 0.939          & 0.500          \\
		2    & 0.980          & 0.927          & \textbf{0.941} & 0.429          \\
		3    & 0.979          & 0.923          & 0.939          & 0.444          \\
		4    & 0.982          & 0.931          & 0.939          & 0.500          \\
		5    & 0.980          & 0.929          & 0.937          & \textbf{0.609} \\
		6    & 0.978          & 0.920          & 0.938          & 0.550          \\
		7    & 0.980          & 0.929          & 0.939          & 0.562          \\
		8    & 0.982          & 0.934          & 0.938          & 0.489          \\
		9    & \textbf{0.989} & \textbf{0.956} & 0.936          & 0.400          \\
		Mean & 0.981          & 0.931          & 0.939          & 0.501          \\
		STD  & 0.003          & 0.010          & 0.001          & 0.064          \\
		\hline
	\end{tabular}
	\caption{Resultados en entrenamiento y generalización para las distintas semillas en clasificación binaria con  \texttt{DecisionTreeClassifier}.}
	\label{tabla:dt_multi}
\end{table}

La precisión en entrenamiento oscila entre 0.978 y 0.989, mientras que en test se sitúan entre 0.936 y 0.941. En cuanto a mínima sensibilidad, los valores en entrenamiento varían desde 0.920 hasta 0.956, y en test desde 0.400 hasta 0.609.

\vspace{1em}

Los promedios de \textit{accuracy} son 0.981 en entrenamiento y 0.939 en test, mientras que los promedios de MS son 0.931 en entrenamiento y 0.501 en test.

\begin{figure}[H]
	\centering
	\includegraphics[width=1\linewidth]{Imagenes/dt_multi}
	\caption[\texttt{Boxplot} con \texttt{violinplot} para árboles de decisión en clasificación multiclase]{\texttt{Boxplot} con \texttt{violinplot} para árboles de decisión en clasificación multiclase.}
	\label{fig:dt_multi}
\end{figure}

En la figura \ref{fig:dt_multi} vemos como el modelo muestra un comportamiento consistente, con distribuciones relativamente concentradas en torno a valores altos tanto en entrenamiento como en test. Las formas de los \texttt{violinplot} indican que la mayoría de las ejecuciones se agrupan cerca de la mediana, con pocas variaciones extremas.

\vspace{1em}

La comparación entre las distribuciones de entrenamiento y test refleja que las métricas de generalización se mantienen cercanas a las de entrenamiento, lo que sugiere estabilidad en el rendimiento del modelo. La dispersión reducida también indica que los resultados son poco sensibles a la elección de la semilla, lo que respalda la fiabilidad del modelo en distintas ejecuciones. En conjunto, se observa un modelo equilibrado y robusto, con buen ajuste a los datos y capacidad de generalización.

\vspace{1em}

Por otro lado, en los resultados de la tabla \ref{tabla:dt_multi}, apreciamos una caída significativa entre los resultados de MS en entrenamiento y generalización, lo que puede indicar ciertos problemas a la hora de clasificar patrones no conocidos.

\subsection{\textit{Random forest}}
\label{subsec:rf_multi}

En la tabla \ref{tabla:rf_multi} se observan los resultados obtenidos para cada semilla de inicialización en entrenamiento y test.

\begin{table}[H]
	\centering
	\begin{tabular}{ |c|c|c|c|c| }
		\hline
		\rowcolor{LightCyan}
		 & \multicolumn{2}{c|}{Entrenamiento} & \multicolumn{2}{c|}{Generalización} \\
		\hline
		\rowcolor{LightCyan}
		 Estado aleatorio & Acc & MS & Acc & MS \\
		\hline
		0    & 0.981          & 0.926          & 0.951          & 0.524          \\
		1    & 0.981          & 0.926          & 0.953          & \textbf{0.735} \\
		2    & 0.981          & 0.926          & 0.954          & 0.429          \\
		3    & 0.980          & 0.923          & 0.954          & 0.500          \\
		4    & 0.981          & 0.926          & \textbf{0.954} & 0.500          \\
		5    & \textbf{0.981} & \textbf{0.927} & 0.952          & 0.638          \\
		6    & 0.980          & 0.923          & 0.952          & 0.550          \\
		7    & 0.981          & 0.927          & 0.954          & 0.500          \\
		8    & 0.981          & 0.926          & 0.953          & 0.471          \\
		9    & 0.981          & 0.926          & 0.953          & 0.400          \\
		Mean & 0.981          & 0.926          & 0.953          & 0.525          \\
		STD  & 0.000          & 0.002          & 0.001          & 0.098          \\
		\hline
	\end{tabular}
	\caption{Resultados en entrenamiento y generalización para las distintas semillas en clasificación multiclase con  \texttt{RandomForestClassifier}.}
	\label{tabla:rf_multi}
\end{table}

Los valores de accuracy se mantienen altos y consistentes en entrenamiento, con una media de 0.981 y desviación estándar muy baja (0.000), indicando un comportamiento uniforme entre las diferentes ejecuciones. La mínima sensibilidad muestra cierta variabilidad entre semillas, con valores que oscilan y una desviación estándar de 0.002 en entrenamiento.

\vspace{1em}

En los datos de test, el Acc promedio es de 0.953, con desviación estándar de 0.001, mientras que la mínima sensibilidad presenta una mayor dispersión (STD = 0.098).

\begin{figure}[H]
	\centering
	\includegraphics[width=1\linewidth]{Imagenes/rf_multi}
	\caption[\texttt{Boxplot} con \texttt{violinplot} para \texttt{RandomForestClassifier} en clasificación multiclase]{\texttt{Boxplot} con \texttt{violinplot} para \texttt{RandomForestClassifier} en clasificación multiclase.}
	\label{fig:rf_multi}
\end{figure}

El modelo presenta un desempeño elevado en \textit{accuracy}, con valores de entrenamiento muy estables y prácticamente constantes en todas las ejecuciones. En la figura \ref{fig:rf_multi}, la distribución de los resultados de entrenamiento se concentra de forma marcada en torno a un mismo valor, sin apenas dispersión.

\vspace{1em}

En los datos de test, la precisión se sitúa próxima a la de entrenamiento, con un ligero descenso que se mantiene consistente en todas las semillas. La forma del \texttt{violinplot} refleja poca variabilidad, y el \texttt{boxplot} confirma que la mediana y los cuartiles se encuentran muy próximos entre sí.

\vspace{1em}

En conjunto, el comportamiento sugiere un modelo que mantiene estabilidad entre entrenamiento y generalización, pero no asegura un comportamiento homogéneo por clases: la MS en test es más baja e inestable, lo que apunta a dificultades para recuperar consistentemente las clases más difíciles.

\subsection{\textit{K-NN}}
\label{subsec:knn_multi}


En la tabla \ref{tabla:knn_multi}, podemos ver como en el entrenamiento, la métrica Acc se mantiene constante en 0.994 en todos los casos, mientras que MS varía entre 0.794 y 0.849. En el conjunto de prueba, Acc oscila entre 0.939 y 0.943, y MS presenta valores comprendidos entre 0.357 y 0.529.

\vspace{1em}

Para la media, tenemos un Acc de 0.994 y un MS de 0.815 en entrenamiento, y un Acc de 0.941 junto a un MS de 0.451 en prueba. Por su parte, las desviaciones estándar en entrenamiento son de 0.000 para Acc y 0.016 para MS, mientras que en prueba alcanzan 0.001 para Acc y 0.067 para MS.
\begin{table}[H]
	\centering
	\begin{tabular}{ |c|c|c|c|c| }
		\hline
		\rowcolor{LightCyan}
		 & \multicolumn{2}{c|}{Entrenamiento} & \multicolumn{2}{c|}{Generalización} \\
		\hline
		\rowcolor{LightCyan}
		 Estado aleatorio & Acc & MS & Acc & MS \\
		\hline
		0    & 0.994          & 0.811          & 0.940          & 0.524          \\
		1    & 0.994          & 0.794          & \textbf{0.943} & 0.500          \\
		2    & 0.994          & 0.811          & 0.940          & 0.357          \\
		3    & 0.994          & 0.815          & 0.942          & 0.389          \\
		4    & 0.994          & 0.810          & 0.941          & 0.375          \\
		5    & 0.994          & 0.807          & 0.940          & 0.435          \\
		6    & 0.994          & 0.802          & 0.941          & 0.500          \\
		7    & \textbf{0.994} & \textbf{0.849} & 0.940          & 0.500          \\
		8    & 0.994          & 0.817          & 0.939          & \textbf{0.529} \\
		9    & 0.994          & 0.834          & 0.940          & 0.400          \\
		Mean & 0.994          & 0.815          & 0.941          & 0.451          \\
		STD  & 0.000          & 0.016          & 0.001          & 0.067          \\
		\hline
	\end{tabular}
	\caption{Resultados en entrenamiento y generalización para las distintas semillas en clasificación multiclase con \texttt{KNeighborsClassifier}.}
	\label{tabla:knn_multi}
\end{table}

Los resultados muestran que el modelo obtiene un rendimiento muy alto y estable en el conjunto de entrenamiento y un valor de MS cercano a 0.815 en promedio, con muy poca variación. Esto indica que el modelo se ajusta de forma consistente a los datos de entrenamiento.

\vspace{1em}

En el conjunto de prueba, la exactitud también es elevada aunque más baja, pero con estabilidad. Sin embargo, el MS presenta valores más bajos y con mayor dispersión.

\vspace{1em}

En conjunto, el modelo es consistente es precisión, aunque la métrica MS indica que existen diferencias en cómo se comporta según la partición de los datos. Como MS es más alto y estable en entrenamiento, esto podría indicar que el modelo se ajusta muy bien a los datos de entrenamiento, pero al enfrentarse a datos no vistos pierde algo de consistencia.

\newpage
\subsection{Máquinas de vectores de soporte}
\label{subsec:svm_multi}

En este caso, el proceso de entrenamiento presentó una mayor complejidad y dificultad para obtener resultados comparables con los de otros modelos evaluados, principalmente debido a las limitaciones del equipo utilizado. El elevado tiempo requerido para el entrenamiento sin ajuste de parámetros, junto con los resultados poco satisfactorios obtenidos para las dos semillas empleadas ---con una precisión aproximada del 20\%---, motivaron la decisión de no continuar con las máquinas de vectores de soporte para la clasificación multiclase. No obstante, estos resultados no indican que el modelo sea inadecuado para el problema planteado, sino que tiene una mayor exigencia en cuanto a los recursos necesarios para su entrenamiento.

\subsection{\textit{Ridge}}
\label{subsec:ridge_multi}

La tabla \ref{tabla:ridge_multi} muestra los resultados de \texttt{RidgeClassifier} en diez ejecuciones con distintos estados aleatorios.

\vspace{1em}

La Precisión en entrenamiento varía entre 0.170 y 0.195, con un promedio de 0.182 y una desviación estándar de 0.009 mientras que en prueba fluctúa entre 0.166 y 0.191, con promedio 0.182 y desviación estándar 0.009.La métrica MS es 0.000 en todas las ejecuciones, tanto para entrenamiento como para prueba.

\vspace{1em}

A pesar de que hay estabilidad y capacidad de generalización en el modelo, los resultados son extremadamente pobres. Estos resultados pueden atribuirse a la naturaleza lineal del RidgeClassifier, que limita su capacidad para adaptarse a problemas de alta complejidad o con relaciones no lineales entre las características como puede ser el caso del \textit{malware}.

\begin{table}[H]
	\centering
	\begin{tabular}{ |c|c|c|c|c| }
		\hline
		\rowcolor{LightCyan}
		 & \multicolumn{2}{c|}{Entrenamiento} & \multicolumn{2}{c|}{Generalización} \\
		\hline
		\rowcolor{LightCyan}
		 Estado aleatorio & Acc & MS & Acc & MS \\
		\hline
		0    & 0.189          & 0.000          & 0.186          & 0.000          \\
		1    & \textbf{0.195} & \textbf{0.000} & \textbf{0.191} & \textbf{0.000} \\
		2    & 0.173          & 0.000          & 0.176          & 0.000          \\
		3    & 0.172          & 0.000          & 0.172          & 0.000          \\
		4    & 0.185          & 0.000          & 0.189          & 0.000          \\
		5    & 0.187          & 0.000          & 0.189          & 0.000          \\
		6    & 0.170          & 0.000          & 0.166          & 0.000          \\
		7    & 0.186          & 0.000          & 0.191          & 0.000          \\
		8    & 0.187          & 0.000          & 0.187          & 0.000          \\
		9    & 0.171          & 0.000          & 0.173          & 0.000          \\
		Mean & 0.182          & 0.000          & 0.182          & 0.000          \\
		STD  & 0.009          & 0.000          & 0.009          & 0.000          \\
		\hline
	\end{tabular}
	\caption{Resultados en entrenamiento y generalización para las distintas semillas en clasificación multiclase con \texttt{RidgeClassifier}}
	\label{tabla:ridge_multi}
\end{table}

\newpage
\subsection{Perceptrón multicapa}
\label{subsec:mlp_multi}

En la tabla \ref{tabla:mlp_multi} vemos como la precisión en entrenamiento varía entre 0.679 y 0.730, con promedio de 0.717 y desviación estándar de 0.014, mientras que en generalización oscila entre 0.681 y 0.735, con promedio de 0.718 y desviación estándar de 0.014.

\begin{table}[H]
	\centering
	\begin{tabular}{ |c|c|c|c|c| }
		\hline
		\rowcolor{LightCyan}
		 & \multicolumn{2}{c|}{Entrenamiento} & \multicolumn{2}{c|}{Generalización} \\
		\hline
		\rowcolor{LightCyan}
		 Estado aleatorio & Acc & MS & Acc & MS \\
		\hline
		0    & 0.725          & 0.000          & 0.722          & 0.000          \\
		1    & 0.724          & 0.000          & 0.724          & 0.000          \\
		2    & 0.724          & 0.000          & 0.723          & 0.000          \\
		3    & 0.679          & 0.000          & 0.681          & 0.000          \\
		4    & \textbf{0.730} & \textbf{0.000} & \textbf{0.735} & \textbf{0.000} \\
		5    & 0.721          & 0.000          & 0.717          & 0.000          \\
		6    & 0.724          & 0.000          & 0.723          & 0.000          \\
		7    & 0.711          & 0.000          & 0.711          & 0.000          \\
		8    & 0.719          & 0.000          & 0.720          & 0.000          \\
		9    & 0.716          & 0.000          & 0.718          & 0.000          \\
		Mean & 0.717          & 0.000          & 0.718          & 0.000          \\
		STD  & 0.014          & 0.000          & 0.014          & 0.000          \\
		\hline
	\end{tabular}
	\caption{Resultados en entrenamiento y generalización para las distintas semillas en clasificación multiclase con \texttt{MLPClassifier}}
	\label{tabla:mlp_multi}
\end{table}

El modelo muestra un rendimiento consistente y relativamente alto en precisión, como podemos ver en el gráfico representado en la figura \ref{fig:mlp_multi}, tanto en entrenamiento como en prueba, lo que indica estabilidad y buena capacidad de generalización. A pesar de esto, la métrica MS se mantiene en cero, lo que evidencia que el modelo no logra capturar completamente la diversidad de clases ni ciertos patrones específicos del conjunto de datos. La cercanía entre el desempeño en entrenamiento y prueba sugiere que no existe sobreajuste significativo; sin embargo, las limitaciones en MS reflejan que el modelo todavía enfrenta dificultades para manejar todos los aspectos del problema multiclase, posiblemente debido la la cantidad de clases con pocos patrones.

\begin{figure}[H]
	\centering
	\includegraphics[width=1\linewidth]{Imagenes/mlp_multi}
	\caption[\texttt{Boxplot} con \texttt{violinplot} para \texttt{MLPClassifier} en clasificación multiclase]{\texttt{Boxplot} con \texttt{violinplot} para \texttt{MLPClassifier} en clasificación multiclase.}
	\label{fig:mlp_multi}
\end{figure}

\newpage
\subsection{\textit{Light gradient boosting machine}}
\label{subsec:lgbm_multi}

La tabla \ref{tabla:lgbm_multi}, muestra los resultados de precisión y mínima sensibilidad \texttt{LGBMClassifier} en diez ejecuciones con distintos estados aleatorios.

\begin{table}[H]
	\centering
	\begin{tabular}{ |c|c|c|c|c|c| }
		\hline
		\rowcolor{LightCyan}
		 & \multicolumn{2}{c|}{Entrenamiento} & \multicolumn{2}{c|}{Test} \\
		\hline
		\rowcolor{LightCyan}
		 Estado aleatorio & Acc & MS & Acc & MS \\
		\hline
		0    & 0.938          & 0.821          & 0.916          & 0.600          \\
		1    & 0.936          & 0.820          & 0.916          & \textbf{0.735} \\
		2    & 0.890          & 0.749          & 0.884          & 0.357          \\
		3    & 0.323          & 0.000          & 0.327          & 0.000          \\
		4    & 0.888          & 0.747          & 0.880          & 0.500          \\
		5    & \textbf{0.938} & \textbf{0.828} & \textbf{0.917} & 0.565          \\
		6    & 0.893          & 0.758          & 0.881          & 0.550          \\
		7    & 0.936          & 0.821          & 0.917          & 0.562          \\
		8    & 0.891          & 0.750          & 0.880          & 0.588          \\
		9    & 0.893          & 0.760          & 0.883          & 0.467          \\
		Mean & 0.853          & 0.706          & 0.840          & 0.492          \\
		STD  & 0.187          & 0.250          & 0.181          & 0.198          \\
		\hline
	\end{tabular}
	\caption{Resultados en entrenamiento y generalización para las distintas semillas en clasificación multiclase con \texttt{LGBMClassifier}}
	\label{tabla:lgbm_multi}
\end{table}

El Acc en entrenamiento presenta valores muy variables entre ejecuciones, mientras que en prueba también se observa variabilidad, aunque ligeramente menor. La mínima sensibilizad en entrenamiento y prueba oscila entre valores altos y cero, reflejando diferencias importantes entre ejecuciones. La desviación típica de todas las métricas es bastante alta, situándose en 0.25 para MS en entrenamiento.

\begin{figure}[H]
	\centering
	\includegraphics[width=1\linewidth]{Imagenes/lgbm_multi}
	\caption[\texttt{Boxplot} con \texttt{violinplot} para \texttt{LGBMClassifier} en clasificación multiclase]{\texttt{Boxplot} con \texttt{violinplot} para \texttt{LGBMClassifier} en clasificación multiclase.}
	\label{fig:lgbm_multi}
\end{figure}

Como podemos ver en la figura \ref{fig:lgbm_multi}, este modelo muestra un rendimiento alto y relativamente consistente en la mayoría de ejecuciones, tanto en entrenamiento como en prueba. La excepción corresponde a una ejecución aislada, que podría considerarse ruido en una muestra mayor, cuyos resultados están muy por debajo del resto. Podemos interpretar que no representa el comportamiento típico del modelo.

\vspace{1em}

Si no tenemos en cuenta esta muestra, el modelo tiene buena generalización, además de estabilidad entre entrenamiento y prueba, aunque algunas diferencias menores sugieren que puede haber un ligero sobreajuste. La métrica MS indica que el modelo no captura de manera uniforme todos los patrones de las clases.

\subsection{Discusión de los resultados}
\label{subsec:discusion_multi}

Las tablas \ref{tabla:resumen_mean} y \ref{tabla:resumen_std}, resumen los valores medios y las desviaciones típicas de las métricas Acc y MS en entrenamiento y generalización para los distintos modelos implementados.

\vspace{1em}

En la tabla \ref{tabla:resumen_mean} se presentan las medias de las métricas de exactitud y mínima sensibilidad tanto en entrenamiento como en generalización para cada modelo. Se observa que algunos modelos alcanzan valores altos y consistentes de exactitud en ambas fases, mientras que otros presentan valores notablemente más bajos, especialmente en mínima sensibilidad. Esta información permite comparar el rendimiento promedio de cada metodología y proporciona una visión general de cuáles modelos tienden a clasificar correctamente todas las clases de manera más equilibrada.

\begin{table}[H]
	\centering
	\begin{tabular}{|c|c|c|c|c|}
		\hline
		\rowcolor{LightCyan}
		Modelo & \multicolumn{2}{c|}{Entrenamiento} & \multicolumn{2}{c|}{Generalización} \\
		\hline
		\rowcolor{LightCyan}
		& Acc & MS & Acc & MS \\
		\hline
		\texttt{KNeighborsClassifier}   & 0.994 & 0.815 & 0.941 & 0.451 \\
		\texttt{RidgeClassifier}        & 0.182 & 0.000 & 0.182 & 0.000 \\
		\texttt{MLPClassifier}          & 0.717 & 0.000 & 0.718 & 0.000 \\
		\texttt{LGBMClassifier}         & 0.853 & 0.706 & 0.840 & 0.492 \\
		\texttt{DecisionTreeClassifier} & 0.981 & 0.931 & 0.939 & 0.501 \\
		\texttt{RandomForestClassifier} & 0.981 & 0.926 & 0.953 & 0.525 \\
		\texttt{SVC}                    & --    & --    & --    & --    \\
		\hline
	\end{tabular}
	\caption{Resumen de medias de métricas en entrenamiento y generalización para cada modelo}
	\label{tabla:resumen_mean}
\end{table}

\newpage
En la tabla \ref{tabla:resumen_std_multi} se muestran las desviaciones estándar de las métricas de exactitud y mínima sensibilidad. Algunos modelos presentan desviaciones muy bajas, reflejando consistencia y uniformidad en sus resultados, mientras que otros muestran variaciones más altas, especialmente en mínima sensibilidad. Estos datos permiten identificar qué modelos son más confiables, así como aquellos cuya capacidad de predicción puede depender de la semilla o partición de los datos.

\begin{table}[H]
	\centering
	\begin{tabular}{|c|c|c|c|c|}
		\hline
		\rowcolor{LightCyan}
		Modelo & \multicolumn{2}{c|}{Entrenamiento} & \multicolumn{2}{c|}{Generalización} \\
		\hline
		\rowcolor{LightCyan}
		& Acc & MS & Acc & MS \\
		\hline
		\texttt{KNeighborsClassifier}   & 0.000 & 0.016 & 0.001 & 0.067 \\
		\texttt{RidgeClassifier}        & 0.009 & 0.000 & 0.009 & 0.000 \\
		\texttt{MLPClassifier}          & 0.014 & 0.000 & 0.014 & 0.000 \\
		\texttt{LGBMClassifier}         & 0.187 & 0.250 & 0.181 & 0.198 \\
		\texttt{DecisionTreeClassifier} & 0.003 & 0.010 & 0.001 & 0.064 \\
		\texttt{RandomForestClassifier} & 0.000 & 0.002 & 0.001 & 0.098 \\
		\texttt{SVC}                    & --    & --    & --    & --    \\
		\hline
	\end{tabular}
	\caption{Resumen de desviaciones típicas en entrenamiento y generalización para cada modelo}
	\label{tabla:resumen_std_multi}
\end{table}

Analizando todos los modelos evaluados, se observa que el \texttt{RandomForestClassifier} y el \texttt{KNeighborsClassifier} son los que presentan el mejor rendimiento. El primero de ellos destaca por su alta precisión y mínima sensibilidad tanto en entrenamiento como en generalización, con desviaciones estándar bajas, lo que indica resultados consistentes y confiables. El segundo, muestra también una exactitud elevada y resultados estables, con una mínima sensibilidad aceptable. Aunque su MS es ligeramente inferior a la de RandomForest, mantiene un buen equilibrio entre rendimiento y consistencia

\vspace{1em}

En la gráfica representada en la figura \ref{fig:vs_multi} podemos ver que entre estas dos opciones, el clasificador \texttt{RandomForestClassifier} obtiene un rendimiento ligeramente superior.

\begin{figure}[H]
	\centering
	\includegraphics[width=1\linewidth]{Imagenes/vs_multi}
	\caption[Comparativa entre los modelos \texttt{KNegihborsClassifier} y \texttt{LGBMClassifier}]{Comparativa entre los modelos \texttt{KNegihborsClassifier} y \texttt{LGBMClassifier}.}
	\label{fig:vs_multi}
\end{figure}

\chapter{Conclusiones y recomendaciones}
\label{ch:conclusiones}

En este último capítulo del proyecto se expondrán las conclusiones, recomendaciones y mejoras de este proyecto.

\section{Conclusiones de implementación}
\label{sec:conc_implementacion}



\section{Conclusiones de investigación}
\label{sec:conc_investigacion}



\section{Recomendaciones}
\label{sec:recomendaciones}




% --> poner otros capítulos. Por ejemplo:
	%\input{2_objetivos}
	%\input{3_antecedentes}
	%\input{4_materiales}
	%\input{5_diseno}
	%\input{6_pruebas}
	%\chapter{Ejemplos de código Latex}
\label{ch:codigos_ejemplo}
Aquí también se puede poner texto.

\section{Acrónimos}
Ejemplo de como poner un acrónimo \acrshort{iot}. Estos están definidos en el archivo ``anexo\_acronimos.text''.


\section{Comillas}
Mira el código para que veas como se hace bien el ``entrecomillado''. No se usan las comillas situadas en la tecla del dígito 2.

\section{Listas}
En esta sección se muestra como crear dos tipos de listas.

\subsection{Lista Enumerada}
\begin{enumerate}
\item Color rojo
\item Color verde
\item Color amarillo
\end{enumerate}



\subsection{Lista NO-Enumerada}
\begin{itemize}
\item Color rojo
\item Color verde
\item Color amarillo
\end{itemize}

\section{Como incluir FIGURAS}

\subsection{Figura única}
\label{sec:figura_unica}

\begin{figure}[H]
  \centering
  % include first image
  \includegraphics[width=.8\linewidth]{Imagenes/logo_uco}  
  \caption[Titulo de la figura para el índice de figuras. En este no se deben poner referencias a citas]{Título de la figura para el texto}
  \label{fig:figura_unica}
\end{figure}

\subsection{Figura con subfiguras}

En la Figura \ref{figT:subfiguras} se puede aprecia una figura compuesta por dos subfiguras.

\begin{figure}[H]
\begin{subfigure}{.5\textwidth}
  \centering
  % include first image
  \includegraphics[width=.8\linewidth]{Imagenes/logo_uco}  
  \caption{Título subfigura 1}
  \label{fig:sub-first}
\end{subfigure}
\begin{subfigure}{.5\textwidth}
  \centering
  % include second image
  \includegraphics[width=.8\linewidth]{Imagenes/logo_uco}  
  \caption{Título subfigura 2}
  \label{fig:sub-second}
\end{subfigure}
\caption{Titulo principal figura}
\label{figT:subfiguras}
\end{figure}

\subsection{Figura con subfiguras en distintas líneas}
\begin{figure}[H]
\begin{subfigure}{.5\textwidth}
  \centering
  % include first image
  \includegraphics[width=.8\linewidth]{Imagenes/logo_uco}  
  \caption{Título subfigura 1}
  \label{fig:sub-first}
\end{subfigure}
\begin{subfigure}{.5\textwidth}
  \centering
  % include second image
  \includegraphics[width=.8\linewidth]{Imagenes/logo_uco}  
  \caption{Título subfigura 2}
  \label{fig:sub-second}
\end{subfigure}

\begin{subfigure}{.5\textwidth}
  \centering
  % include first image
  \includegraphics[width=.8\linewidth]{Imagenes/logo_uco}  
  \caption{Título subfigura 3}
  \label{fig:sub-first}
\end{subfigure}
\begin{subfigure}{.5\textwidth}
  \centering
  % include second image
  \includegraphics[width=.8\linewidth]{Imagenes/logo_uco}  
  \caption{Título subfigura 4}
  \label{fig:sub-second}
\end{subfigure}

\caption{Figuras en multiples lineas}
\label{figT:subfiguras2}
\end{figure}


\section{Fórmulas Matemáticas}

\subsection{Formulas referenciables}
\begin{equation} \label{eq:erl} a = bq + r \end{equation} donde \eqref{eq:erl} es verdadera si $a$ y $b$ son enteros con $b \neq c$. 

\subsection{Fórmulas que no se van a referenciar}
\begin{align*} 
X = 0,42 \ \tau_{625} + 0.35 \ \tau_{550} + 0.21 \ \tau_{445} \\
Y = 0,20 \ \tau_{625} + 0.63 \ \tau_{550} + 0.17 \ \tau_{495} \\
Z = 0,24 \ \tau_{495} + 0.94 \ \tau_{445} 
\end{align*}

Y con estos valores se calculan las coordenadas \textit{x} e \textit{y}:

\begin{align*} 
x = X / (X + Y + Z) \quad y = Y / (X + Y + Z)
\end{align*}

\section{Diagramas usando TikZ}
\url{https://es.overleaf.com/learn/latex/LaTeX_Graphics_using_TikZ%3A_A_Tutorial_for_Beginners_(Part_1)%E2%80%94Basic_Drawing}

\subsection{Ejemplo diagrama de flujo}

\begin{diagrama}
\centering
 
    \begin{tikzpicture}[node distance=2cm]

    \node (start) [startstop] {Start};
    \node (in1) [io, below of=start] {Input};
    \node (pro1) [process, below of=in1] {Process 1};
    \node (dec1) [decision, below of=pro1, yshift=-0.5cm] {Decision 1};

    \node (pro2a) [process, below of=dec1, yshift=-0.5cm] {Process 2a text text text text text text text};

    \node (pro2b) [process, right of=dec1, xshift=2cm] {Process 2b};
    \node (out1) [io, below of=pro2a] {Output};
    \node (stop) [startstop, below of=out1] {Stop};

    \draw [arrow] (start) -- (in1);
    \draw [arrow] (in1) -- (pro1);
    \draw [arrow] (pro1) -- (dec1);
    \draw [arrow] (dec1) -- node[anchor=east] {yes} (pro2a);
    \draw [arrow] (dec1) -- node[anchor=south] {no} (pro2b);
    \draw [arrow] (pro2b) |- (pro1);
    \draw [arrow] (pro2a) -- (out1);
    \draw [arrow] (out1) -- (stop);

    \end{tikzpicture}

	\caption{Ejemplo diagrama}
	\label{diagrama:ejemplo_diagrama}
\end{diagrama}

\section{Referencia a distintas secciones y figuras}
En la sección \ref{sec:figura_unica} se puede ver la Figura \ref{fig:figura_unica}.

En el capítulo \ref{ch:presupuesto} se puede ver el código para crear una tabla. La Tabla \ref{tabla:presupuesto} muestra un presupuesto.
	%\chapter{Resultados y discusión}
\label{ch:resultados}

% TODO: ESCRIBIR INTRODUCCION

\section{Clasificación binaria}
\label{sec:clas_binaria}

% TODO: escribir introducción para clasificaicon binaria, este texto es de experimentacion
En esta fase se lleva a cabo la implementación práctica del estudio, haciendo uso de los modelos de aprendizaje automático implementados principalmente en la librería \textit{Scikit-Learn} de \textit{python} y descritos en el capítulo \ref{ch:metodologia}. Para ello se configuran los entornos necesarios para su entrenamiento y evaluación, se establecen las métricas de rendimiento, los procedimientos de prueba y los escenarios de experimentación que permitirán obtener resultados consistentes y comparables. El objetivo es verificar, mediante pruebas controladas, la efectividad de cada método en la detección de \textit{malware}.

La parte experimental de este proyecto se estudiará desde dos enfoques distintos. Por un lado se evaluarán los modelos seleccionados en la detección de \textit{malware}, es decir, se realizarán pruebas de clasificación binaria donde se estudiará sin un un patrón corresponde a un programa malicioso o no. Por otro, se estudiará si, para estos mismos patrones, es posible realizar una clasificación más exhaustiva y reconocer con que tipo de \textit{malware} se corresponde cada patrón.

\subsection{Árboles de decisión}
\label{subsec:dt_bin}

\begin{table}[th]
	\centering
	\begin{tabular}{ |c|c|c|c|c|c|c| }
		\hline
		\rowcolor{LightCyan}
		 & \multicolumn{3}{c|}{Entrenamiento} & \multicolumn{3}{c|}{Test} \\
		\hline
		\rowcolor{LightCyan}
		 Estado aleatorio & Acc & MS & F1 & Acc & MS & F1 \\
		\hline
		0 & 1.000 & 1.000 & 1.000 & 0.951 & 0.942 & 1.000 \\
		1 & 1.000 & 1.000 & 1.000 & 0.944 & 0.935 & 1.000 \\
		2 & 1.000 & 1.000 & 1.000 & 0.942 & 0.935 & 1.000 \\
		3 & 1.000 & 1.000 & 1.000 & 0.948 & 0.938 & 1.000 \\
		4 & 1.000 & 1.000 & 1.000 & 0.953 & 0.946 & 1.000 \\
		5 & 0.998 & 0.997 & 1.000 & 0.947 & 0.936 & 1.000 \\
		6 & 0.998 & 0.997 & 1.000 & 0.947 & 0.941 & 1.000 \\
		7 & 1.000 & 1.000 & 1.000 & 0.949 & 0.946 & 1.000 \\
		8 & 0.999 & 0.998 & 1.000 & 0.948 & 0.937 & 1.000 \\
		9 & 1.000 & 1.000 & 1.000 & 0.950 & 0.940 & 1.000 \\
		Mean & 0.999 & 0.999 & 1.000 & 0.948 & 0.940 & 1.000 \\
		STD & 0.001 & 0.001 & 0.000 & 0.003 & 0.004 & 0.000 \\
		\hline
	\end{tabular}
	\caption{Clasificación binaria con \textit{DecisionTreeClassifier}}
	\label{tabla:dt_bin}
\end{table}


\subsection{\textit{Random forest}}
\label{subsec:rf_bin}

\input{Tablas/6_desarrollo/rf_bin}

\subsection{\textit{K-NN}}
\label{subsec:knn_bin}

\begin{table}[th]
	\centering
	\begin{tabular}{ |c|c|c|c|c|c|c| }
		\hline
		\rowcolor{LightCyan}
		 & \multicolumn{3}{c|}{Entrenamiento} & \multicolumn{3}{c|}{Test} \\
		\hline
		\rowcolor{LightCyan}
		 Estado aleatorio & Acc & MS & F1 & Acc & MS & F1 \\
		\hline
		0 & 1.000 & 1.000 & 1.000 & 0.947 & 0.935 & 1.000 \\
		1 & 1.000 & 1.000 & 1.000 & 0.949 & 0.938 & 1.000 \\
		2 & 1.000 & 1.000 & 1.000 & 0.939 & 0.926 & 1.000 \\
		3 & 1.000 & 1.000 & 1.000 & 0.949 & 0.937 & 1.000 \\
		4 & 1.000 & 1.000 & 1.000 & 0.949 & 0.936 & 1.000 \\
		5 & 1.000 & 1.000 & 1.000 & 0.946 & 0.932 & 1.000 \\
		6 & 1.000 & 1.000 & 1.000 & 0.946 & 0.931 & 1.000 \\
		7 & 1.000 & 1.000 & 1.000 & 0.944 & 0.934 & 1.000 \\
		8 & 1.000 & 1.000 & 1.000 & 0.947 & 0.935 & 1.000 \\
		9 & 1.000 & 1.000 & 1.000 & 0.949 & 0.940 & 1.000 \\
		Mean & 1.000 & 1.000 & 1.000 & 0.947 & 0.934 & 1.000 \\
		STD & 0.000 & 0.000 & 0.000 & 0.003 & 0.004 & 0.000 \\
		\hline
	\end{tabular}
	\caption{Clasificación binara con \textit{KNeighborscClassifier}}
	\label{tabla:knn_bin}
\end{table}


\subsection{Máquinas de vectores de soporte}
\label{subsec:svm_bin}

\begin{table}[th]
	\centering
	\begin{tabular}{ |c|c|c|c|c|c|c| }
		\hline
		\rowcolor{LightCyan}
		 & \multicolumn{3}{c|}{Entrenamiento} & \multicolumn{3}{c|}{Test} \\
		\hline
		\rowcolor{LightCyan}
		 Estado aleatorio & Acc & MS & F1 & Acc & MS & F1 \\
		\hline
		0 & 0.757 & 0.656 & 1.000 & 0.764 & 0.672 & 1.000 \\
		1 & 0.761 & 0.671 & 1.000 & 0.769 & 0.681 & 1.000 \\
		2 & 0.766 & 0.702 & 1.000 & 0.757 & 0.699 & 1.000 \\
		3 & 0.762 & 0.700 & 1.000 & 0.766 & 0.704 & 1.000 \\
		4 & 0.760 & 0.684 & 1.000 & 0.768 & 0.696 & 1.000 \\
		5 & 0.761 & 0.663 & 1.000 & 0.753 & 0.655 & 1.000 \\
		6 & 0.762 & 0.702 & 1.000 & 0.762 & 0.683 & 1.000 \\
		7 & 0.763 & 0.699 & 1.000 & 0.759 & 0.697 & 1.000 \\
		8 & 0.766 & 0.704 & 1.000 & 0.758 & 0.693 & 1.000 \\
		9 & 0.760 & 0.662 & 1.000 & 0.758 & 0.666 & 1.000 \\
		Mean & 0.762 & 0.684 & 1.000 & 0.762 & 0.685 & 1.000 \\
		STD & 0.003 & 0.020 & 0.000 & 0.005 & 0.016 & 0.000 \\
		\hline
	\end{tabular}
	\caption{Clasificación binaria con \textit{SVC}}
	\label{tabla:svm_bin}
\end{table}


\subsection{\textit{Ridge}}
\label{subsec:ridge_bin}

\begin{table}[th]
	\centering
	\begin{tabular}{ |c|c|c|c|c|c|c| }
		\hline
		\rowcolor{LightCyan}
		 & \multicolumn{3}{c|}{Entrenamiento} & \multicolumn{3}{c|}{Test} \\
		\hline
		\rowcolor{LightCyan}
		 Estado aleatorio & Acc & MS & F1 & Acc & MS & F1 \\
		\hline
		0 & 0.649 & 0.549 & 1.000 & 0.648 & 0.530 & 1.000 \\
		1 & 0.645 & 0.558 & 1.000 & 0.655 & 0.569 & 1.000 \\
		2 & 0.652 & 0.573 & 1.000 & 0.645 & 0.564 & 1.000 \\
		3 & 0.649 & 0.567 & 1.000 & 0.653 & 0.570 & 1.000 \\
		4 & 0.651 & 0.573 & 1.000 & 0.651 & 0.573 & 1.000 \\
		5 & 0.647 & 0.562 & 1.000 & 0.648 & 0.558 & 1.000 \\
		6 & 0.648 & 0.556 & 1.000 & 0.650 & 0.573 & 1.000 \\
		7 & 0.651 & 0.571 & 1.000 & 0.650 & 0.573 & 1.000 \\
		8 & 0.651 & 0.564 & 1.000 & 0.639 & 0.551 & 1.000 \\
		9 & 0.650 & 0.563 & 1.000 & 0.645 & 0.551 & 1.000 \\
		Mean & 0.649 & 0.564 & 1.000 & 0.648 & 0.561 & 1.000 \\
		STD & 0.002 & 0.008 & 0.000 & 0.005 & 0.014 & 0.000 \\
		\hline
	\end{tabular}
	\caption{Clasificación binaria con \textit{RidgeClassifier}}
	\label{tabla:ridge_bin}
\end{table}


\subsection{Redes neuronales: Perceptrón multicapa}
\label{subsec:mlp_bin}

\begin{table}[th]
	\centering
	\begin{tabular}{ |c|c|c|c|c|c|c| }
		\hline
		\rowcolor{LightCyan}
		 & \multicolumn{3}{c|}{Entrenamiento} & \multicolumn{3}{c|}{Test} \\
		\hline
		\rowcolor{LightCyan}
		 Estado aleatorio & Acc & MS & F1 & Acc & MS & F1 \\
		\hline
		0 & 0.783 & 0.771 & 1.000 & 0.789 & 0.778 & 1.000 \\
		1 & 0.788 & 0.736 & 1.000 & 0.792 & 0.740 & 1.000 \\
		2 & 0.788 & 0.750 & 1.000 & 0.782 & 0.739 & 1.000 \\
		3 & 0.733 & 0.605 & 1.000 & 0.737 & 0.609 & 1.000 \\
		4 & 0.767 & 0.759 & 1.000 & 0.769 & 0.760 & 1.000 \\
		5 & 0.790 & 0.736 & 1.000 & 0.783 & 0.730 & 1.000 \\
		6 & 0.777 & 0.772 & 1.000 & 0.783 & 0.781 & 1.000 \\
		7 & 0.774 & 0.767 & 1.000 & 0.770 & 0.763 & 1.000 \\
		8 & 0.778 & 0.704 & 1.000 & 0.772 & 0.705 & 1.000 \\
		9 & 0.788 & 0.762 & 1.000 & 0.784 & 0.751 & 1.000 \\
		Mean & 0.776 & 0.736 & 1.000 & 0.776 & 0.736 & 1.000 \\
		STD & 0.017 & 0.051 & 0.000 & 0.016 & 0.050 & 0.000 \\
		\hline
	\end{tabular}
	\caption{Clasificación binaria con \textit{MLPClassifier}}
	\label{tabla:mlp_bin}
\end{table}


\subsection{\textit{Light Gradient Boosting Machine}}
\label{subsec:lgbm_bin}

\begin{table}[th]
	\centering
	\begin{tabular}{ |c|c|c|c|c|c|c| }
		\hline
		\rowcolor{LightCyan}
		 & \multicolumn{3}{c|}{Entrenamiento} & \multicolumn{3}{c|}{Test} \\
		\hline
		\rowcolor{LightCyan}
		 Estado aleatorio & Acc & MS & F1 & Acc & MS & F1 \\
		\hline
		0 & 0.984 & 0.981 & 1.000 & 0.953 & 0.952 & 1.000 \\
		1 & 0.984 & 0.980 & 1.000 & 0.951 & 0.947 & 1.000 \\
		2 & 0.985 & 0.983 & 1.000 & 0.949 & 0.946 & 1.000 \\
		3 & 0.985 & 0.982 & 1.000 & 0.952 & 0.951 & 1.000 \\
		4 & 0.984 & 0.981 & 1.000 & 0.950 & 0.945 & 1.000 \\
		5 & 0.985 & 0.981 & 1.000 & 0.949 & 0.948 & 1.000 \\
		6 & 0.985 & 0.982 & 1.000 & 0.952 & 0.949 & 1.000 \\
		7 & 0.986 & 0.984 & 1.000 & 0.948 & 0.947 & 1.000 \\
		8 & 0.984 & 0.979 & 1.000 & 0.953 & 0.952 & 1.000 \\
		9 & 0.989 & 0.989 & 1.000 & 0.953 & 0.950 & 1.000 \\
		Mean & 0.985 & 0.982 & 1.000 & 0.951 & 0.949 & 1.000 \\
		STD & 0.002 & 0.003 & 0.000 & 0.002 & 0.002 & 0.000 \\
		\hline
	\end{tabular}
	\caption{Clasificación binaria con \textit{LGBMClassifier}}
	\label{tabla:lgbm_bin}
\end{table}


\section{Clasificación multiclase}
\label{sec:clas_multi}

\subsection{Árboles de decisión}
\label{subsec:dt_multi}

\begin{table}[th]
	\centering
	\begin{tabular}{ |c|c|c|c|c|c|c| }
		\hline
		\rowcolor{LightCyan}
		 & \multicolumn{3}{c|}{Entrenamiento} & \multicolumn{3}{c|}{Test} \\
		\hline
		\rowcolor{LightCyan}
		 Estado aleatorio & Acc & MS & F1 & Acc & MS & F1 \\
		\hline
		0 & 0.980 & 0.928 & 0.989 & 0.939 & 0.524 & 0.972 \\
		1 & 0.980 & 0.929 & 0.990 & 0.939 & 0.500 & 0.973 \\
		2 & 0.980 & 0.927 & 0.989 & 0.941 & 0.429 & 0.974 \\
		3 & 0.979 & 0.923 & 0.989 & 0.939 & 0.444 & 0.973 \\
		4 & 0.982 & 0.931 & 0.990 & 0.939 & 0.500 & 0.974 \\
		5 & 0.980 & 0.929 & 0.990 & 0.937 & 0.609 & 0.971 \\
		6 & 0.978 & 0.920 & 0.988 & 0.938 & 0.550 & 0.971 \\
		7 & 0.980 & 0.929 & 0.990 & 0.939 & 0.562 & 0.972 \\
		8 & 0.982 & 0.934 & 0.991 & 0.938 & 0.489 & 0.971 \\
		9 & 0.989 & 0.956 & 0.992 & 0.936 & 0.400 & 0.972 \\
		Mean & 0.981 & 0.931 & 0.990 & 0.939 & 0.501 & 0.972 \\
		STD & 0.003 & 0.010 & 0.001 & 0.001 & 0.064 & 0.001 \\
		\hline
	\end{tabular}
	\caption{Clasificación multiclase con \textit{DecisionTreeClassifier}}
	\label{tabla:dt_multi}
\end{table}


\subsection{\textit{Random forest}}
\label{subsec:rf_multi}

\begin{table}[th]
	\centering
	\begin{tabular}{ |c|c|c|c|c|c|c| }
		\hline
		\rowcolor{LightCyan}
		 & \multicolumn{3}{c|}{Entrenamiento} & \multicolumn{3}{c|}{Test} \\
		\hline
		\rowcolor{LightCyan}
		 Estado aleatorio & Acc & MS & F1 & Acc & MS & F1 \\
		\hline
		0 & 0.981 & 0.926 & 0.990 & 0.951 & 0.524 & 0.977 \\
		1 & 0.981 & 0.926 & 0.990 & 0.953 & 0.735 & 0.978 \\
		2 & 0.981 & 0.926 & 0.990 & 0.954 & 0.429 & 0.978 \\
		3 & 0.980 & 0.923 & 0.990 & 0.954 & 0.500 & 0.978 \\
		4 & 0.981 & 0.926 & 0.990 & 0.954 & 0.500 & 0.979 \\
		5 & 0.981 & 0.927 & 0.990 & 0.952 & 0.638 & 0.977 \\
		6 & 0.980 & 0.923 & 0.990 & 0.952 & 0.550 & 0.977 \\
		7 & 0.981 & 0.927 & 0.991 & 0.954 & 0.500 & 0.978 \\
		8 & 0.981 & 0.926 & 0.990 & 0.953 & 0.471 & 0.978 \\
		9 & 0.981 & 0.926 & 0.990 & 0.953 & 0.400 & 0.978 \\
		Mean & 0.981 & 0.926 & 0.990 & 0.953 & 0.525 & 0.978 \\
		STD & 0.000 & 0.002 & 0.000 & 0.001 & 0.098 & 0.001 \\
		\hline
	\end{tabular}
	\caption{Clasificación multiclase con \textit{RandomForestClassifier}}
	\label{tabla:rf_multi}
\end{table}


\subsection{\textit{K-NN}}
\label{subsec:knn_multi}

\begin{table}[th]
	\centering
	\begin{tabular}{ |c|c|c|c|c|c|c| }
		\hline
		\rowcolor{LightCyan}
		 & \multicolumn{3}{c|}{Entrenamiento} & \multicolumn{3}{c|}{Test} \\
		\hline
		\rowcolor{LightCyan}
		 Estado aleatorio & Acc & MS & F1 & Acc & MS & F1 \\
		\hline
		0 & 0.994 & 0.811 & 0.997 & 0.940 & 0.524 & 0.976 \\
		1 & 0.994 & 0.794 & 0.996 & 0.943 & 0.500 & 0.977 \\
		2 & 0.994 & 0.811 & 0.997 & 0.940 & 0.357 & 0.977 \\
		3 & 0.994 & 0.815 & 0.996 & 0.942 & 0.389 & 0.978 \\
		4 & 0.994 & 0.810 & 0.997 & 0.941 & 0.375 & 0.976 \\
		5 & 0.994 & 0.807 & 0.996 & 0.940 & 0.435 & 0.976 \\
		6 & 0.994 & 0.802 & 0.996 & 0.941 & 0.500 & 0.976 \\
		7 & 0.994 & 0.849 & 0.996 & 0.940 & 0.500 & 0.976 \\
		8 & 0.994 & 0.817 & 0.996 & 0.939 & 0.529 & 0.976 \\
		9 & 0.994 & 0.834 & 0.996 & 0.940 & 0.400 & 0.976 \\
		Mean & 0.994 & 0.815 & 0.996 & 0.941 & 0.451 & 0.976 \\
		STD & 0.000 & 0.016 & 0.000 & 0.001 & 0.067 & 0.001 \\
		\hline
	\end{tabular}
	\caption{Clasificación multiclase con \textit{KNeighborsClasiffier}}
	\label{tabla:knn_multi}
\end{table}


\subsection{Máquinas de vectores de soporte}
\label{subsec:svm_multi}

En este caso, el proceso de entrenamiento presentó una mayor complejidad y dificultad para obtener resultados comparables con los de otros modelos evaluados, principalmente debido a las limitaciones del equipo utilizado. El elevado tiempo requerido para el entrenamiento sin ajuste de parámetros, junto con los resultados poco satisfactorios obtenidos para las dos semillas empleadas ---con una precisión aproximada del 20\%---, motivaron la decisión de no continuar con las máquinas de vectores de soporte para la clasificación multiclase. No obstante, estos resultados no indican que el modelo sea inadecuado para el problema planteado, sino que tiene una mayor exigencia en cuanto a los recursos necesarios para su entrenamiento.

\subsection{\textit{Ridge}}
\label{subsec:ridge_multi}

\begin{table}[th]
	\centering
	\begin{tabular}{ |c|c|c|c|c|c|c| }
		\hline
		\rowcolor{LightCyan}
		 & \multicolumn{3}{c|}{Entrenamiento} & \multicolumn{3}{c|}{Test} \\
		\hline
		\rowcolor{LightCyan}
		 Estado aleatorio & Acc & MS & F1 & Acc & MS & F1 \\
		\hline
		0 & 0.189 & 0.000 & 0.301 & 0.186 & 0.000 & 0.299 \\
		1 & 0.195 & 0.000 & 0.308 & 0.191 & 0.000 & 0.307 \\
		2 & 0.173 & 0.000 & 0.284 & 0.176 & 0.000 & 0.287 \\
		3 & 0.172 & 0.000 & 0.283 & 0.172 & 0.000 & 0.280 \\
		4 & 0.185 & 0.000 & 0.297 & 0.189 & 0.000 & 0.305 \\
		5 & 0.187 & 0.000 & 0.300 & 0.189 & 0.000 & 0.303 \\
		6 & 0.170 & 0.000 & 0.280 & 0.166 & 0.000 & 0.274 \\
		7 & 0.186 & 0.000 & 0.299 & 0.191 & 0.000 & 0.303 \\
		8 & 0.187 & 0.000 & 0.300 & 0.187 & 0.000 & 0.301 \\
		9 & 0.171 & 0.000 & 0.282 & 0.173 & 0.000 & 0.284 \\
		Mean & 0.182 & 0.000 & 0.293 & 0.182 & 0.000 & 0.294 \\
		STD & 0.009 & 0.000 & 0.010 & 0.009 & 0.000 & 0.012 \\
		\hline
	\end{tabular}
	\caption{Clasificación multiclase con \textit{RidgeClassifier}}
	\label{tabla:ridge_multi}
\end{table}


\subsection{Redes neuronales: Perceptrón multicapa}
\label{subsec:mlp_multi}

\begin{table}[th]
	\centering
	\begin{tabular}{ |c|c|c|c|c|c|c| }
		\hline
		\rowcolor{LightCyan}
		 & \multicolumn{3}{c|}{Entrenamiento} & \multicolumn{3}{c|}{Test} \\
		\hline
		\rowcolor{LightCyan}
		 Estado aleatorio & Acc & MS & F1 & Acc & MS & F1 \\
		\hline
		0 & 0.725 & 0.000 & 0.885 & 0.722 & 0.000 & 0.883 \\
		1 & 0.724 & 0.000 & 0.901 & 0.724 & 0.000 & 0.900 \\
		2 & 0.724 & 0.000 & 0.885 & 0.723 & 0.000 & 0.885 \\
		3 & 0.679 & 0.000 & 0.885 & 0.681 & 0.000 & 0.888 \\
		4 & 0.730 & 0.000 & 0.904 & 0.735 & 0.000 & 0.902 \\
		5 & 0.721 & 0.000 & 0.888 & 0.717 & 0.000 & 0.884 \\
		6 & 0.724 & 0.000 & 0.889 & 0.723 & 0.000 & 0.888 \\
		7 & 0.711 & 0.000 & 0.885 & 0.711 & 0.000 & 0.884 \\
		8 & 0.719 & 0.000 & 0.910 & 0.720 & 0.000 & 0.910 \\
		9 & 0.716 & 0.000 & 0.886 & 0.718 & 0.000 & 0.885 \\
		Mean & 0.717 & 0.000 & 0.892 & 0.718 & 0.000 & 0.891 \\
		STD & 0.014 & 0.000 & 0.009 & 0.014 & 0.000 & 0.009 \\
		\hline
	\end{tabular}
	\caption{Clasificación multiclase con \textit{MLPClassifier}}
	\label{tabla:mlp_multi}
\end{table}


\subsection{\textit{Light Gradient Boosting Machine}}
\label{subsec:lgbm_multi}

\begin{table}[th]
	\centering
	\begin{tabular}{ |c|c|c|c|c|c|c| }
		\hline
		\rowcolor{LightCyan}
		 & \multicolumn{3}{c|}{Entrenamiento} & \multicolumn{3}{c|}{Test} \\
		\hline
		\rowcolor{LightCyan}
		 Estado aleatorio & Acc & MS & F1 & Acc & MS & F1 \\
		\hline
		0 & 0.938 & 0.821 & 0.965 & 0.916 & 0.600 & 0.953 \\
		1 & 0.936 & 0.820 & 0.964 & 0.916 & 0.735 & 0.953 \\
		2 & 0.890 & 0.749 & 0.941 & 0.884 & 0.357 & 0.936 \\
		3 & 0.323 & 0.000 & 0.460 & 0.327 & 0.000 & 0.460 \\
		4 & 0.888 & 0.747 & 0.940 & 0.880 & 0.500 & 0.936 \\
		5 & 0.938 & 0.828 & 0.967 & 0.917 & 0.565 & 0.955 \\
		6 & 0.893 & 0.758 & 0.943 & 0.881 & 0.550 & 0.935 \\
		7 & 0.936 & 0.821 & 0.964 & 0.917 & 0.562 & 0.952 \\
		8 & 0.891 & 0.750 & 0.941 & 0.880 & 0.588 & 0.933 \\
		9 & 0.893 & 0.760 & 0.942 & 0.883 & 0.467 & 0.936 \\
		Mean & 0.853 & 0.706 & 0.903 & 0.840 & 0.492 & 0.895 \\
		STD & 0.187 & 0.250 & 0.156 & 0.181 & 0.198 & 0.153 \\
		\hline
	\end{tabular}
	\caption{Clasificación multiclase con \textit{LGBMClassifier}}
	\label{tabla:lgbm_multi}
\end{table}




%--------------------%
% DOCUMENTOS TÉNICOS %
%--------------------%

%\chapter*{PRESUPUESTO}
\label{ch:presupuesto}

\addcontentsline{toc}{chapter}{Presupuesto} % si queremos que aparezca en el índice
\markboth{PRESUPUESTO}{PRESUPUESTO} % encabezado

En esta seccion, se presentan los gastos asociados al desarrollo del presente proyecto, los cuales se refieren exclusivamente a los elementos que conforman la parte hardware del sistema.

%\clearpage

\definecolor{LightCyan}{rgb}{0.88,1,1}

\begin{table}[ht]
	\centering
	\begin{tabular}{ |m{7cm}|c|r|r| }
		\hline
		\rowcolor{LightCyan}
		Descripción & Cantidad & Precio (\euro) & Importe (\euro)\\
		\hline
		Pantalla LCD Quimat 20x4 & 1 & 10,69 & 10,69\\
		\hline
		Baterías 9 V & 3 & 2,10 & 6,36\\
		\hline
		Amplificador operacional TL082IP & 3 & 0,61 & 1,84\\
		\hline
		Regulador tensión L7905CV  & 1 & 0,40 &  0,40\\
		\hline
		Resistencias $20 M\Omega$ & 1 & 0,60 & 0,60\\
		\hline
		Resistencias $5M \Omega$ & 1 & 0,85 & 0,85\\
		\hline
		Condensadores $2.2 \mu F$ & 1 & 0,37 & 0,37 \\
		\hline
		Condensadores $1 \mu F$ & 1 & 0,20 & 0,20\\
		\hline
		Condensadores $1 \rho F$ & 3 & 0,35 & 1,05\\
		\hline
		Transistor BJT 2N3904 & 6 & 0,40 & 2,4 \\
		\hline
		Placas de prototipado & 2 & 3,00 & 6,00\\
		\hline
		\textbf{Total} &  &  & \textbf{59,90 \euro}\\
		\hline
	\end{tabular}
	\caption{Presupuesto del proyecto.}
	\label{tabla:presupuesto}
\end{table}




%------------------%
%	BIBLIOGRAFIA   %
%------------------%

%Estilo de la bibliografia
\bibliographystyle{unsrtnat}
%\bibliographystyle{elsarticle-num}
\bibliography{referencias.bib}

%------------------%
%      ANEXOS      %
%------------------%

%\chapter*{PLANOS}

\addcontentsline{toc}{chapter}{Planos} % si queremos que aparezca en el índice
\markboth{PLANOS}{PLANOS} % encabezado


\includepdf[scale=0.8,pages=1,angle=90,pagecommand=\section*{Placa de Circuito Impreso (PCB)}\label{planos:PCB}]{DocumentacionTecnica/ejemplo_pcb.png}


\includepdf[scale=0.8,pages=1,angle=90,pagecommand=\section*{Esquema Eléctrico}\label{planos:esquema_electrico}]{DocumentacionTecnica/Ejemplo_Esquema_Electrico.pdf}

\appendix

% \chapter{Hojas de Características}

% Pages: Es este parámetro se indican cuantas páginas se quieren poner 1 o 1-2, ...
% Angle: A veces es necesario rotar la página --> angle=90
\includepdf[scale=0.7,pages=1,angle=0,pagecommand=\section{Amplificador Operacional 741}\label{hc:ao_741}]{HojasDeCaracteristicas/ejemplo_lm741.pdf}
\chapter{Código del programa}

El código completo de este proyecto se encuentra en github para su libre consulta \cite{github}

\section{Codificación de las categorías \textit{malware}}
\label{sec:codificacion}

\lstset{style=codestyle, language=Python}
\begin{lstlisting}[frame=single]
X, y        = load('bodmas/bodmas.npz')
metadata    = pd.read_csv('bodmas/bodmas_metadata.csv')
mw_category = pd.read_csv('bodmas/bodmas_malware_category.csv')

# Incluimos los valores de 'category' en metadata cuando coinciden los valoes de 'sha'
mw_category = metadata.merge(mw_category, on = 'sha', how = 'left')

# Rellenamos los huecos como software benigno
mw_category['category'] = mw_category['category'].fillna('benign')

# Eliminamos todas las columnas excepto 'category'
mw_category = mw_category['category']

# Codificamos las categorias de malware
category = {
	'benign': 0, 'trojan': 1, 'worm': 2, 'backdoor': 3,
	'downloader': 4, 'informationstealer': 5, 'dropper': 6,
	'ransomware': 7, 'rootkit': 8, 'cryptominer': 9, 'pua': 10,
	'exploit': 11, 'virus': 12, 'p2p-worm': 13, 'trojan-gamethief': 14
}

mw_category = mw_category.map(category)

y = mw_category.to_numpy()

save('bodmas/bodmas_multiclass.npz', X, y)

\end{lstlisting}

\section{Reducción de la dimensionalidad}
\label{sec:red_dim}

\lstset{style=codestyle, language=Python}
\begin{lstlisting}[frame=single]
def resampling(X, y, n_components = 5, size = 15000, u = False):
	if u:
		rus  = RandomUnderSampler(sampling_strategy = {0: size, 1: size})
		# rus  = RandomUnderSampler(sampling_strategy = 'majority')
		X, y = rus.fit_resample(X, y)

	X_train, X_test, y_train, y_test = train_test_split(
		X, y, test_size = 0.25, random_state = 1
	)

	pca     = PCA(n_components)
	X_train = pca.fit_transform(X_train)
	X_test  = pca.transform(X_test)

	return X_train, X_test, y_train, y_test
\end{lstlisting}

\newpage
\section{Pruebas para la elección del conjunto de datos}
\label{sec:select_dataset}

\lstset{style=codestyle, language=Python}
\begin{lstlisting}[frame=single]
file = {'pca_binary', 'resampling_binary', 'pca_multiclass'}
clf  = None

print('clasificador,dataset,n patrones,n caracteristicas,accuracy,tiempo')

for i in range(3):
	if i == 0: clf = DecisionTreeClassifier()
	elif i == 1: clf = RandomForestClassifier()
	else: clf = KNeighborsClassifier()

	for train_file in file:

		X_train, y_train = load('bodmas/' + train_file + '_train.npz')
		X_test, y_test = load('bodmas/' + train_file + '_test.npz')

		# Entrenar el modelo
		inicio = time.time()
		clf.fit(X_train, y_train)
		tiempo = time.time() - inicio

		# Predecir sobre el conjunto de prueba
		y_pred = clf.predict(X_test)

		# Evaluar
		accuracy = accuracy_score(y_test, y_pred)

		print(f'{i},{train_file},{X_train.shape},{accuracy:.3f},{tiempo:.3f}')

\end{lstlisting}

\newpage
\section{Control de la validación cruzada}
\label{sec:func_cv}

\lstset{style=codestyle, language=Python}
\begin{lstlisting}[frame=single]
def cv(y, crossval):
	y_ = min(pd.DataFrame(y).value_counts())

	if y_ < crossval:
		return y_

	return crossval
\end{lstlisting}

\section{Ejemplo de salida de la información}
\label{sec:info}

\lstset{style=codestyle, language=Python}
\begin{lstlisting}[frame=single]
      acc train  ms train  f1 train  acc test   ms test  f1 test
0      0.648800  0.548712       1.0  0.648133  0.530019      1.0
1      0.645200  0.558267       1.0  0.655200  0.568564      1.0
2      0.652400  0.572655       1.0  0.644667  0.563784      1.0
3      0.648578  0.566829       1.0  0.653467  0.569664      1.0
4      0.650933  0.573087       1.0  0.650933  0.573003      1.0
5      0.647289  0.562228       1.0  0.647867  0.558393      1.0
6      0.647867  0.556000       1.0  0.650267  0.572533      1.0
7      0.650667  0.570983       1.0  0.649600  0.572906      1.0
8      0.650711  0.564155       1.0  0.638800  0.551123      1.0
9      0.649911  0.563205       1.0  0.645200  0.550628      1.0
Mean   0.649236  0.563612       1.0  0.648413  0.561062      1.0
STD    0.002112  0.007797       0.0  0.004713  0.013867      0.0
\end{lstlisting}

% \chapter{Manual de Usuario}


\section{Introducción}



\end{document}
